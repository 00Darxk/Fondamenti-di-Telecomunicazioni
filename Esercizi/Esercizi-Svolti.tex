\documentclass{article}

\usepackage{cancel}
\usepackage{tikz}
\usepackage{amsmath}
\usepackage[includehead,nomarginpar]{geometry}
\usepackage{graphicx}
\usepackage{amsfonts} 
\usepackage{verbatim}
\usepackage{mathrsfs}  
\usepackage{lmodern}
\usepackage{braket}
\usepackage{bookmark}
\usepackage{steinmetz}
\usepackage[italian]{babel}
\usepackage{pgfplots}
\usepackage{fancyhdr}
\usepackage{romanbarpagenumber}
\usepackage{circuitikz}
\allowdisplaybreaks

\pgfplotsset{compat=1.16}
\setlength{\headheight}{12.0pt}
\addtolength{\topmargin}{-12.0pt}

%% inserire grafici

\hypersetup{
    colorlinks=true,
    linkcolor=black,
}
\tikzset{block/.style = {draw, fill=white, very thick, rectangle, minimum height=1cm, minimum width=2cm},  
         square/.style = {draw, fill=white, very thick, rectangle, minimum height=1cm, minimum width=1cm},
         sum/.style= {draw, fill=white, very thick, circle, node distance=0.5cm},  
         cross/.pic = {  \draw[rotate = 45] (-#1,0) -- (#1,0);
                         \draw[rotate = 45] (0,-#1) -- (0, #1);}}


\renewcommand{\contentsname}{Indice}
\newcommand{\intinf}{\displaystyle\int_{-\infty}^{+\infty}}
\newcommand{\rect}{\mathrm{rect}}
\newcommand{\sinc}{\mathrm{sinc}}
\newcommand{\tri}{\mathrm{tri}}
\newcommand{\df}{\mathrm{d}}
\newcommand{\tageq}{\tag{\stepcounter{equation}\theequation}}
\newcommand{\Frac}[2]{\displaystyle\frac{\strut{#1}}{\strut{#2}}}

\fancypagestyle{link}{\fancyhf{}\renewcommand{\headrulewidth}{0pt}\fancyfoot[C]{Sorgente del file LaTeX disponibile al seguente link: \url{https://github.com/00Darxk/Fondamenti-di-Telecomunicazioni}}}

\begin{document}

\title{%
    \textbf{Fondamenti di Telecomunicazioni}  \\ 
    \large Esercizi Svolti di Fondamenti di Telecomunicazioni\\
    \textit{Anno Accademico: 2023/24}}
\author{\textit{Giacomo Sturm}}
\date{\textit{Dipartimento di Ingegneria Civile, Informatica e delle Tecnologie Aeronautiche \\
Università degli Studi ``Roma Tre"}}

\maketitle
\thispagestyle{link}


\clearpage

\pagestyle{fancy}
\fancyhead{}\fancyfoot{}
\fancyhead[C]{\textit{Fondamenti di Telecomunicazioni - Università degli Studi ``Roma Tre"}}
\fancyfoot[C]{\thepage}

\pagenumbering{Roman}
\tableofcontents

\clearpage

\pagenumbering{arabic}

\section{Analisi nel Tempo}

\subsection{Energia e Potenza}

\subsubsection*{Esercizio 1}

Si determina l'energia di una gaussiana di ampiezza $A$:
\begin{gather*}
    E_x=\lim_{\Delta t\to\infty}\displaystyle\int_{-\frac{\Delta t}{2}}^{\frac{\Delta t}{2}}A^2e^{-2\alpha t^2}\df t=A^2\lim_{\Delta t\to\infty}\displaystyle\int_{-\frac{\Delta t}{2}}^{\frac{\Delta t}{2}}e^{-2\alpha t^2}\df t
\end{gather*}
Si considera il cambio di variabile $\tau=\sqrt{2\alpha}t$:
\begin{equation*}
    E_x=A^2\lim_{\Delta t\to\infty}\displaystyle\int_{-\frac{\Delta t}{2}}^{\frac{\Delta t}{2}}\frac{e^{-\tau^2}}{\sqrt{2\alpha}}\df\tau=\frac{A^2}{\sqrt{2\alpha}}\int_{-\infty}^{+\infty}e^{-\tau^2}\df\tau
\end{equation*}
L'integrale ottenuto è l'integrale di Gauss, il suo valore risulta quindi essere: 
\begin{equation*}
    \displaystyle\int_{\mathbb{R}}e^{-t^2}\df t=\sqrt\pi
\end{equation*}
Per cui l'energia di una gaussiana è:
\begin{equation}
    E_x=A^2\displaystyle\sqrt{\frac{\pi}{2\alpha}}
\end{equation}

\subsubsection*{Esercizio 2}

Si determina l'energia e la potenza di un esponenziale complesso. Il segnale ha un modulo unitario $|x(t)|=1$, per cui la sua energia risultante è:
\begin{equation}
    E_x=\lim_{\Delta t\to\infty}\displaystyle\int_{-\frac{\Delta t}{2}}^{\frac{\Delta t}{2}}\df t=\lim_{\Delta t\to\infty}\left(\frac{\Delta t}{2}+\frac{\Delta t}{2}\right)=\infty
\end{equation}
Per cui l'esponenziale complesso non è un segnale di energia. La potenza risulta essere:
\begin{equation}
    P_x=\lim_{\Delta t\to\infty}\displaystyle\left(\frac{1}{\Delta t}\int_{-\frac{\Delta t}{2}}^{\frac{\Delta t}{2}}\df t\right)=\lim_{\Delta t\to\infty}\left(\frac{1}{\Delta t}\cdot\Delta t\right)=1
\end{equation}
L'esponenziale complesso è quindi un segnale di potenza. 

\subsubsection*{Esercizio 3}

Si determina l'energia di un esponenziale unilatero:
\begin{equation}
    E_x=\displaystyle\int_{-\infty}^{+\infty}\left[e^{-\alpha t}u(t)\right]^2\df t=\int_0^{+\infty}e^{-2\alpha t}\df t=-\frac{1}{2\alpha}\left(\cancelto{0}{e^{-\infty}}-\cancelto{1}{e^{0}}\right)=\frac{1}{2\alpha}
\end{equation}
Per cui questo segnale non è né di energia né di potenza. 

\subsubsection*{Esercizio 4}

Si determina la potenza del segnale coseno, di ampiezza $A$ e frequenza naturale $f_0$:
\begin{gather*}
    x(t)=A\cos\left(2\pi f_0t\right)\\
    P_x=\displaystyle A^2f_0\int_{-\frac{1}{2f_0}}^{\frac{1}{2f_0}}\cos^2(2\pi f_0 t)\df t
\end{gather*}
Per esprimere il quadrato del coseno, si considera la formula di bisezione del coseno:
\begin{gather*}
    \cos(2x)=2\cos^2(x)-1\\
    \cos^2(x)=\displaystyle\frac{\cos(2x)+1}{2}\\
    A^2\cos^2(2\pi f_0t)=\displaystyle\frac{A^2}{2}(\cos(4\pi f_0t)+1)
\end{gather*}
Si può esprimere inoltre mediante la notazione complessa delle funzioni trigonometriche:
\begin{gather*}
    A\cos(2\pi f_0t)=\displaystyle\frac{A}{2}(e^{2i\pi f_0t}+e^{-2i\pi f_0t})\\
    |x+y|^2\,\,x,y\in\mathbb{C}\\
    (x+y)\cdot(x+y)^*=(x+y)\cdot(x^*+y^*)\\
    xx^*+xy^*+yx^*+yy^*=|x|^2+|y|^2+xy^*+yx^*\\
    xy^*+yx^*=(a_x+ib_x)\cdot(a_y-ib_y)+(a_x-ib_x)\cdot(a_y+ib_y)=2(a_xa_y+b_xb_y)=2\Re\{x^*y\}\\
    |x|^2+|y|^2+2\Re\{|x|e^{-i\varphi_x}+|y|e^{i\varphi_y}\}\\
    |x|^2+|y|^2+2|x||y|\cos(|\varphi_y-\varphi_x|)\\
    \Bigg|\displaystyle\frac{A}{2}\left(e^{2i\pi f_0t}+e^{-2i\pi f_0t}\right)\Bigg|^2=\frac{A^2}{4}\left(1+1+2\Re\Bigl\{e^{2i\pi f_0t}\cdot\left(e^{-2i\pi f_0t}\right)^*\Bigr\}\right)\\
    A^2\cos^2(2\pi f_0t)=\displaystyle\frac{A^2}{2}(1+\cos(4\pi f_0t))
\end{gather*}
Considerando questa sostituzione, l'integrale diventa:
\begin{equation*}
    P_x=\displaystyle\frac{A^2f_0}{2}\int_{-\frac{1}{2f_0}}^{\frac{1}{2f_0}}(\cancelto{0}{\cos(4\pi f_0 t)}+1)\df t=\frac{A^2f_0}{2}\left(\frac{1}{2f_0}+\frac{1}{2f_0}\right)=\frac{A^2}{2}
\end{equation*}
L'integrale su un periodo del coseno è nullo, poiché è una funzione pari, per cui la componente $\cos(4\pi f_0 t)$ fornisce un contributo nullo. 

\subsection{Convoluzioni, Correlazioni e Sistemi Ingresso-Uscita}

\subsubsection*{Esercizio 5}

Calcolare la convoluzione tra un esponenziale unilatero ed un gradino, 
Si considera $\alpha\in\mathbb{R}^+$. 
\begin{equation*}
    z(t)=e^{-\alpha t}u(t)*u(t)=\displaystyle\int_{-\infty}^{+\infty}e^{-\alpha\tau}u(\tau)u(t-\tau)\df\tau
\end{equation*}
Poiché è presente un gradino nell'integrale, la convoluzione assumerà valori nulli per $t<0$, ciò si può anche individuare graficamente, poiché per gli stessi valori i due 
grafici del gradino e dell'esponenziale unilatero non si sovrappongono. All'aumentare del valore di $t$, il valore del segnale $z(t)$ aumenta sempre più lentamente, poiché i 
contributi dell'esponenziale vanno diminuendo. Poiché entrambi i gradini nell'integrale assumono valori unitari per $t>0$, il valore del segnale in questo intervallo 
dipende dalla funzione integrale dell'esponenziale, da $0$ al valore di $t$ corrente:
\begin{equation*}
    z(t)=\begin{cases}
        0&t<0\\
        \displaystyle\int_0^te^{-\alpha \tau}\df\tau&t\geq0
    \end{cases}
\end{equation*}
Risolvendo l'integrale si ottiene:
\begin{equation*}
    \displaystyle\int_0^te^{-\alpha \tau}\df\tau=\left|-\frac{e^{-\alpha\tau}}{\alpha}\right|^t_0=\frac{1-e^{-\alpha t}}{\alpha}
\end{equation*}
Per cui il segnale convoluzione in forma analitica risulta:
\begin{equation}
    z(t)=\begin{cases}
        0&t<0\\
        \displaystyle\frac{1-e^{-\alpha t}}{\alpha}&t\geq0
    \end{cases}
\end{equation}
Questo segnale tende asintoticamente a $1/\alpha$ per $t\to+\infty$. 

\begin{center}
    \begin{tikzpicture}[scale=2]
        \draw[->,very thick](-1,0)--(1.5,0)node[above]{$\tau$};
        \draw[->,very thick](0,-0.5)--(0,1.5)node[right]{$x(\tau)$};
        \node[below right]at(0,0){$0$};
        \node[above right]at(0,1){$1$};

        \draw[-,very thick]plot[smooth, domain=0:1.5](\x,{e^(-\x)});

        \draw[->,very thick](-1,-2.1)--(1.5,-2.1)node[above]{$\tau$};
        \draw[->,very thick](0,-2.6)--(0,-0.6)node[left]{$y(t-\tau)$};
        \node[below right]at(0,-2.1){$0$};
        \node[above right]at(0,-1.1){$1$};

        \draw[-,very thick](-1,-1.1)--(0.4,-1.1)--(0.4,-2.1)node[below right]{$t$};
        \draw[dashed,very thick](0.4,-1.1)--(0.4,0.67);


        \draw[->,very thick](2,-1)--(4,-1)node[above]{$t$};
        \draw[->,very thick](2.5,-1.5)--(2.5,0.5)node[right]{$z(t)$};
        \node[below right]at(2.5,-1){$0$};
        \node[left]at(2.5,0){$\displaystyle\frac{1}{\alpha}$};
        \draw[dashed](2.5,0)--(4,0);
        \draw[-,very thick]plot[smooth, domain=2.5:4](\x,{-e^(-2*(\x-2.5))});
    \end{tikzpicture}
\end{center}

\subsubsection*{Esercizio 6}

Calcolare l'autoconvoluzione di un esponenziale unilatero
\begin{equation*}
    z(t)=e^{-\alpha t}u(t)*e^{-\alpha t}u(t)=\displaystyle\int_{-\infty}^{+\infty}e^{-\alpha\tau}u(\tau)e^{-\alpha(t-\tau)}u(t-\tau)\df\tau
\end{equation*}
Poiché sono unilateri, non si sovrapporranno per $t<0$, quindi la convoluzione è nulla per quei valori di tempo. Può essere spiegato tramite la proprietà del gradino di 
cambiare i limiti di integrazione, per cui invece di integrare da $-\infty$ a $+\infty$, si integra nell'intervallo dove si trova il gradino $u(\tau)$, ovvero da $0$ a $+\infty$: 
\begin{equation*}
    z(t)=\displaystyle\int_0^{+\infty}e^{-\alpha t}u(t-\tau)\df\tau=e^{-\alpha t}\int_0^{+\infty}u(t-\tau)\df\tau
\end{equation*}
L'area sottesa da un gradino ribaltato e traslato di un fattore $t>0$ da $0$ a $+\infty$ equivale all'area di un rettangolo di altezza $1$ e di base $t$:
\begin{equation*}
    z(t)=e^{-\alpha t}\displaystyle\int_0^t\df\tau=te^{-\alpha t}
\end{equation*}

In forma analitica risulta:
\begin{equation}
    z(t)=\begin{cases}
        0&t<0\\
        te^{-\alpha t}&t\geq0
    \end{cases}
\end{equation}
 

\begin{center}
    \begin{tikzpicture}[scale=2]
        \draw[->,very thick](-1,0)--(1.5,0)node[above]{$\tau$};
        \draw[->,very thick](0,-0.5)--(0,1.5)node[right]{$x(\tau)$};
        \node[below right]at(0,0){$0$};
        \node[above right]at(0,1){$1$};

        \draw[-,very thick]plot[smooth, domain=0:1.5](\x,{e^(-\x)});

        \draw[->,very thick](-1,-2.1)--(1.5,-2.1)node[above]{$\tau$};
        \draw[->,very thick](0,-2.6)--(0,-0.6)node[left]{$y(t-\tau)$};
        \node[below right]at(0,-2.1){$0$};
        \node[above right]at(0,-1.1){$1$};

        \draw[-,very thick]plot[smooth, domain=-1:0.4](\x,{e^(\x-0.67)-2.1});
        \draw[-,very thick](0.4,-2.1)node[below]{$t$}--(0.4,-1.337);
        \draw[dashed,very thick](0.4,-1.337)--(0.4,0.67);


        \draw[->,very thick](2,-1)--(4,-1)node[below]{$t$};
        \draw[->,very thick](2.5,-1.5)--(2.5,0)node[right]{$z(t)$};
        \node[below right]at(2.5,-1){$0$};
        \draw[-,very thick]plot[smooth, domain=2.5:4](\x,{2*(\x-2.5)*e^(-2*(\x-2.5))-1});
    \end{tikzpicture}
\end{center}

\subsubsection*{Esercizio 7}

Calcolare la convoluzione tra due finestre. 
Si considerano due casi, dove le due finestra hanno base uguale, ed un caso dove hanno base differente. Si considera il caso $T_1=T_2$:
\begin{equation*}
    z(t)=\rect\left(\displaystyle\frac{t}{T}\right)*\rect\left(\frac{t}{T}\right)
\end{equation*}

La convoluzione assume valori nulli quando non si sovrappongono, ovvero per un valore $t+\displaystyle\frac{T}{2}<-\frac{T}{2}\to t<-T$. L'area sottesa dal prodotto 
di queste due finestre aumenta linearmente fino a quando non si sovrappongono per $t=0$, dove l'area assume valore massimo $T\cdot 1$, dopo il quale 
decresce linearmente fino a $t=T$. Se una delle due fosse stata traslata di un fattore $t_0$, l'intera convoluzione sarebbe stata traslata dello stesso fattore. Da notare 
che la convoluzione di due segnali pari genera un segnale pari.  
\begin{equation*}
    z(t)=\begin{cases}
        0&t<-T\\
        \displaystyle\int_{-T}^t\df\tau&-T\leq t<0\\
        \displaystyle\int_t^T\df\tau&0\leq t<T\\
        0&t>T
    \end{cases}=\begin{cases}
        0&t<-T\land t>T\\
        T-|t|&-T\leq t<T\\
    \end{cases}\\
    z(t)=T\,\tri \left(\displaystyle\frac{t}{T}\right)\tageq
\end{equation*}
La convoluzione tra due finestre di base uguale risulta in un triangolo di base doppia $2T$, e scalata di un fattore pari alla base $T$. 

\begin{center}
    \begin{tikzpicture}[scale=2]
        \draw[->,very thick](-1,0)--(1.5,0)node[above]{$\tau$};
        \draw[->,very thick](0,-0.5)--(0,1.5)node[right]{$x(\tau)$};
        \node[below right]at(0,0){$0$};
        \node[above right]at(0,1){$1$};

        \draw[-,very thick](-0.5,0)node[below]{$-\displaystyle\frac{T}{2}$}--(-0.5,1)--(0.5,1)--(0.5,0)node[below]{$\displaystyle\frac{T}{2}$};

        \draw[->,very thick](-1,-2.1)--(1.5,-2.1)node[above]{$\tau$};
        \draw[->,very thick](0,-2.6)--(0,-0.6)node[left]{$y(t-\tau)$};
        \node[below right]at(0,-2.1){$0$};
        \node[above right]at(0,-1.1){$1$};

        \draw[-,very thick](0.3,-2.1)node[below]{$t+\displaystyle\frac{T}{2}$}--(0.3,-1.1)--(-0.7,-1.1)--(-0.7,-2.1)node[below]{$t-\displaystyle\frac{T}{2}$};
        \draw[dashed,very thick](0.3,-1.1)--(0.3,1);


        \draw[->,very thick](2,-1)--(5,-1)node[below]{$t$};
        \draw[->,very thick](3.5,-1.5)--(3.5,0.5)node[right]{$z(t)$};
        \node[below right]at(3.5,-1){$0$};
        \draw[-,very thick](2.5,-1)node[below]{$-T$}--(3.5,0)node[above right]{$T$}--(4.5,-1)node[below]{$T$};
    \end{tikzpicture}
\end{center}

Si considera ora il caso dove le due finestre hanno basi $T_1>T_2$:
\begin{equation*}
    z(t)=\rect\left(\displaystyle\frac{t}{T_1}\right)*\rect\left(\frac{t}{T_2}\right)
\end{equation*}

Poiché le due finestre sono simmetriche, si analizzano solo i casi per $t<0$, per poi aggiungere per simmetria l'equazione analitica per $t\geq0$. Le due finestre non si 
sovrappongono per $t+\displaystyle\frac{T_2}{2}<-\frac{T_1}{2}\to t<-\frac{1}{2}(T_1+T_2)$, per cui in quell'intervallo la convoluzione assume valore nullo. Il valore 
della convoluzione aumenta linearmente fino a quando la finestra più piccola trasla fino ad essere completamente interna alla finestra di base $T_1$ per un valore 
$t-\displaystyle\frac{T_2}{2}<-\frac{T_1}{2}\to t<-\frac{1}{2}(T_1-T_2)$. Quando le due finestre si sovrappongono, il valore della convoluzione è costante, e corrisponde 
all'area di un rettangolo di base, base della finestra più piccola $T_2$ ed altezza unitaria $T_2\cdot1$ fino a raggiungere $t_0$. Ribaltando questo segnale ottenuto 
si ottiene il segnale per valori $t\geq0$:
\begin{gather*}
    z(t)=\begin{cases}
        0&t<-\displaystyle\frac{1}{2}(T_1+T_2)\\
        \displaystyle\int_{-\frac{1}{2}(T_1+T_2)}^t\df\tau&-\frac{1}{2}(T_1+T_2)\leq t <-\frac{1}{2}(T_1-T_2)\\
        T_2&-\frac{1}{2}(T_1-T_2)\leq t<\frac{1}{2}(T_1-T_2)\\
        -\displaystyle\int_{-\frac{1}{2}(T_1+T_2)}^t\df\tau&\frac{1}{2}(T_1-T_2)\leq t <\frac{1}{2}(T_1+T_2)\\
        0&t\geq\displaystyle\frac{1}{2}(T_1+T_2)
    \end{cases}\\
    z(t)=\begin{cases}
        0&t<-\displaystyle\frac{1}{2}(T_1+T_2)\land t\geq\frac{1}{2}(T_1+T_2)\\
        \displaystyle\frac{1}{2}(T_1+T_2)-|t| &-\frac{1}{2}(T_1+T_2)\leq t <-\frac{1}{2}(T_1-T_2)\land \frac{1}{2}(T_1-T_2)\leq t <\frac{1}{2}(T_1+T_2)\\
        T_2&-\displaystyle\frac{1}{2}(T_1-T_2)\leq t<\frac{1}{2}(T_1-T_2)
    \end{cases}\tageq
\end{gather*}

\begin{center}
    \begin{tikzpicture}[scale=2]
        \draw[->,very thick](-1,0)--(1.5,0)node[above]{$\tau$};
        \draw[->,very thick](0,-0.5)--(0,1.5)node[right]{$x(\tau)$};
        \node[below right]at(0,0){$0$};
        \node[above right]at(0,1){$1$};

        \draw[-,very thick](-0.7,0)node[below]{$-\displaystyle\frac{T_1}{2}$}--(-0.7,1)--(0.7,1)--(0.7,0)node[below]{$\displaystyle\frac{T_1}{2}$};

        \draw[->,very thick](-1,-2.1)--(1.5,-2.1)node[above]{$\tau$};
        \draw[->,very thick](0,-2.6)--(0,-0.6)node[left]{$y(t-\tau)$};
        \node[below right]at(0,-2.1){$0$};
        \node[above right]at(0,-1.1){$1$};

        \draw[-,very thick](0.3,-2.1)node[below]{$t+\displaystyle\frac{T_2}{2}$}--(0.3,-1.1)--(-0.8,-1.1)--(-0.8,-2.1)node[below]{$t-\displaystyle\frac{T_2}{2}$};
        \draw[dashed,very thick](0.3,-1.1)--(0.3,1);


        \draw[->,very thick](2,-1)--(5,-1)node[below]{$t$};
        \draw[->,very thick](3.5,-1.5)--(3.5,0.5)node[right]{$z(t)$};
        \node[above right]at(3.5,-1){$0$};
        \draw[-,very thick](2.5,-1)node[below left]{$-\displaystyle\frac{1}{2}(T_1+T_2)$}--(3,0)--(3.5,0)node[above right]{$T_2$}--(4,0)--(4.5,-1)node[below right]{$\displaystyle\frac{1}{2}(T_1+T_2)$};
        \draw[dashed, very thick](3,0)--(3,-1)node[below]{$\displaystyle-\frac{1}{2}(T_1-T_2)$};
        \draw[dashed, very thick](4,0)--(4,-1)node[below]{$\displaystyle\frac{1}{2}(T_1-T_2)$};
    \end{tikzpicture}
\end{center}

Il grafico di questa convoluzione rappresenta un trapezio di altezza $T_2$, base maggiore $T_1+T_2$ e base minore $T_1-T_2$ Si nota come l'autoconvoluzione di due finestre 
di base uguali rappresenta un caso speciale di questo trapezio, avente base minore nulla $T-T=0$ e base doppia rispetto alla finestra $T+T=2T$, ciò equivale ad un triangolo 
di altezza $T$ e base $2T$. 

\subsubsection*{Esercizio 8}

Calcolare la convoluzione tra i segnali $x$ e $y$:
\begin{gather*}
    x(t)=\displaystyle t\,\rect\left(\frac{t-T}{2T}\right)\\
    y(t)=\displaystyle\rect\left(\frac{t+T}{2T}\right)\\
    z(t)=x(t)*y(t)=\displaystyle\int_{-\infty}^{+\infty}\tau\,\rect\left(\frac{\tau-T}{2T}\right)\rect\left(\frac{t-\tau+T}{2T}\right)\df\tau\\
    z(t)=\begin{cases}
        0&t<-2T\\
        \displaystyle\int_{0}^{t+2T}\tau \df\tau&-2T\leq t<0\\
        \displaystyle\int_{t}^{2T}\tau \df\tau&0\leq t<2T\\
        0&t\geq2T
    \end{cases}\\
    z(t)=\begin{cases}
        0&t<-2T\land t\geq2T\\
        \displaystyle\frac{1}{2}t^2+2Tt+2T^2&-2T\leq t<0\\
        \displaystyle2T^2-\frac{1}{2}t^2&0\leq t<2T
    \end{cases}\tageq
\end{gather*}

\subsubsection*{Esercizio 9}

Calcolare la convoluzione tra queste due finestre:
\begin{gather*}
    z(t)=z(t)*y(t)\\
    x(t)=\displaystyle\rect\left(\frac{t-3T}{4T}\right)\\
    y(t)=-\displaystyle\rect\left(\frac{t+2T}{2T}\right)\\
    z(t)=\begin{cases}
        0&t<-2T\\
        -\displaystyle\int^{3T+t}-{T}\df\tau&-2T\leq t<0\\
        -2T&0\geq t<2T\\
        -\displaystyle\int_{T+t}^{5T}\df\tau&2T\leq t<4T\\
        0&t\geq 4T
    \end{cases}\\
    z(t)=\begin{cases}
        0&t<-2T \land t\geq 4T\\
        -2T-t&-2T\leq t<0\\
        -2T&0\leq t<2T\\
        -4T+t&2T\leq t<4T
    \end{cases}\tageq
\end{gather*}

\subsubsection*{Esercizio 10}

Calcolare la correlazione tra due finestre:
\begin{equation*}
    \rect\displaystyle\left(\frac{t-\frac{T_1}{2}}{T_1}\right)\otimes\rect\left(\frac{t-\frac{T_2}{2}}{T_2}\right)
\end{equation*}

Si considera la prima finestra di base maggiore $T_1>T_2$. Per facilitare il calcolo si esprime come una convoluzione. I due segnali non si sovrappongono per valori 
$T_2-t<0\to t\geq T_2$, per cui la correlazione è nulla. Cominciano a sovrapporsi da $t<T_2$ fino a quando la finestra più piccola non si trova interamente nella prima 
finestra per $-t<0\to t\geq0$. La finestra più piccola si trova interamente in quella più grande, risultando in un'area di $T_2$, fino ad un valore 
$T_2-t<T_1\to t\geq -(T_1-T_2)$. L'area comincia a scendere fino ad un valore $-t<T_1\to t\geq -T_1$. Per valori più piccoli di $-T_1$ le due finestre non si sovrappongono e la 
correlazione risulta nulla.
\begin{equation*}
    R_{xy}(t)=\begin{cases}
        0&t\geq T_2\\
        \displaystyle\int_t^{T_2}\df t& 0\leq t<T_2\\
        T_2& -(T_1+T_2)\leq t<0\\
        \displaystyle\int_{-T_1}^t\df t& -T_1\leq t<-(T_1-T_2)\\
        0&t<-T_1
    \end{cases}\\
    R_{xy}(t)=\begin{cases}
        0& t<-T_1\land t\geq T_2\\
        T_1+t& T_1\leq t<-(T_1-T_2)\\
        T_2& -(T_1-T_2)\leq t<0\\
        T_2-t &  0\leq t<T_2
    \end{cases}\tageq
\end{equation*}

\begin{center}
    \begin{tikzpicture}[scale=2]
        \draw[->,very thick](-1,0)--(1.5,0)node[above]{$\tau$};
        \draw[->,very thick](0,-0.5)--(0,1.5)node[right]{$x(\tau)$};
        \node[below right]at(0,0){$0$};
        \node[above right]at(0,1){$1$};

        \draw[-,very thick](0,0)--(0,1)--(1.2,1)--(1.2,0)node[below]{$T_1$};

        \draw[->,very thick](-1,-2.1)--(1.5,-2.1)node[above]{$\tau$};
        \draw[->,very thick](0,-2.6)--(0,-0.6)node[left]{$y(t+\tau)$};
        \node[below left]at(0,-2.1){$0$};
        \node[above right]at(0,-1.1){$1$};

        \draw[-,very thick](0.3,-2.1)node[below]{$T_2-t$}--(0.3,-1.1)--(-0.8,-1.1)--(-0.8,-2.1)node[below]{$-t$};
        \draw[dashed,very thick](0.3,-1.1)--(0.3,1);


        \draw[->,very thick](2,-1)--(4.5,-1)node[below]{$t$};
        \draw[->,very thick](3.5,-1.5)--(3.5,0.5)node[right]{$z(t)$};
        \node[above right]at(3.5,-1){$0$};
        \draw[-,very thick](2.5,-1)node[below left]{$-T_1$}--(3,0)--(3.5,0)--(4,-1)node[below]{$T_2$};
        \draw[dashed, very thick](3,0)--(3,-1)node[below]{$-(T_1-T_2)$};
    \end{tikzpicture}
\end{center}

\subsubsection*{Esercizio 11}

Dato il sistema definito dall'equazione $y$:
\begin{gather*}
    y(t)=x(t)+3u(t-1)
\end{gather*}
Si dimostra che non è lineare:
\begin{gather*}
    x_1(t)\to y_1(t)=x_1(t)+3u(t-1)\\
    x_2(t)\to y_2(y)=x_2(t)+3u(t-1)\\
    ax_1(t)+bx_2(t)\to ax_1(t)+bx_2(t)+3u(t-1)\neq ay_1(t)+by_2(t)
\end{gather*}
Si dimostra che non è tempo invariante:
\begin{gather*}
    x(t-\tau)\to x(t-\tau)+3u(t-1)\neq y(t-\tau)=x(t-\tau)+3u(t-\tau-1)
\end{gather*}
Sicuramente non è un filtro. Mentre è causale, poiché il gradino è ritardato, e quindi l'uscita non dipende da valori futuri.

\subsubsection*{Esercizio 12}

Calcolare l'uscita di un filtro discreto di risposta impulsiva $h[n]$, con un entrata parametrizzata $x[n]$:
\begin{gather*}
    h[n]=\displaystyle\frac{1}{4}\delta[n+1]+\frac{1}{2}\delta[n]+\frac{1}{4}\delta[n-1]\\
    x[n]=\cos(2\pi\phi n)\\
    \phi_1=0\to x_1[n]=1\\
    y_1[n]=h[n]*x_1[n]=\displaystyle\left[\frac{1}{4}\delta[n+1]+\frac{1}{2}\delta[n]+\frac{1}{4}\delta[n-1]\right]*x_1[n]\\
    \displaystyle\frac{1}{4}x_1[n+1]+\frac{1}{2}x_1[n]+\frac{1}{4}x_1[n-1]=\frac{1}{4}+\frac{1}{2}+\frac{1}{4}=1\\
    y_1[n]=1\tageq\\
    \phi_2=\displaystyle\frac{1}{4}\to x_2[n]=\cos\left(\displaystyle\frac{\pi n}{2}\right)\\
    y_2[n]=\left[\displaystyle\frac{1}{4}\delta[n+1]+\frac{1}{2}\delta[n]+\frac{1}{4}\delta[n-1]\right]*\cos\left(\frac{\pi n}{2}\right)\\
    \displaystyle\frac{1}{4}\cos\left(\frac{\pi (n+1)}{2}\right)+\frac{1}{2}\cos\left(\frac{\pi n}{2}\right)+\frac{1}{4}\cos\left(\frac{\pi (n-1)}{2}\right)\\
    \displaystyle\frac{1}{4}\cos\left(\frac{\pi (n+1)}{2}\right)=-\frac{1}{4}\cos\left(\frac{\pi (n-1)}{2}\right)\,\forall k\in\mathbb{Z}\\
    y_2[n]=\displaystyle\frac{1}{2}\cos\left(\frac{\pi n}{2}\right)\tageq\\
    \phi_3=\displaystyle\frac{1}{2}\to x_3[n]=\cos\left(\displaystyle{\pi n}\right)\\
    y[n]=\displaystyle\frac{1}{4}\cos(\pi (n+1))+\frac{1}{2}\cos(\pi n)+\frac{1}{4}\cos(\pi(n-1))=0\tageq
\end{gather*}

\subsubsection*{Esercizio 13}

Si considera un generico schema di un circuito sommatore, un tipo di sistema controllore a feedforward. Calcolarne l'uscita:
\begin{center}
\begin{tikzpicture}[scale=2]
    \node[sum](+)at(2,0){$+$};
        \draw[->,ultra thick](0,0)node[above]{$x[n]$}--(+.180);
        \filldraw[black](0.5,0)circle(1pt);

        \node[square](t)at(1,-0.5){$T$};
        \draw[->,ultra thick](0.5,0)--(0.5,-0.5)--(t.180);

        \node[sum](x)at(1.5,-0.5){$\times$};
        \draw[->,ultra thick](t.0)--(x.180);
        \draw[->,ultra thick](1.5,-1)node[right]{$a$}--(x.270);

        \draw[->,ultra thick](x.0)--(2,-0.5)--(+.270);
        \draw[->,ultra thick](+.0)--(2.5,0)node[above]{$y[n]$};
\end{tikzpicture}
\end{center}
Per ottenere l'uscita si somma il segnale originario allo stesso segnale ritardato di un campione e moltiplicato per un fattore $a$:
\begin{gather}
    y[n]=x[n]+ax[n-1]\\
    h[n]=\delta[n]+a\delta[n-1]
\end{gather}
\begin{center}
\begin{tikzpicture}[scale=2]
    \node[block](s){$h[n]$};
        \draw[->,ultra thick](-1,0)--(s.180)node[above left]{$x[n]$};
        \draw[->,ultra thick](s.0)node[above right]{$y[n]$}--(1,0);
\end{tikzpicture}
\end{center}

\clearpage

\section{Analisi in Frequenza}

\subsection{Serie di Fourier}

\subsubsection*{Esercizio 14}

Dato il segnale $x(t)$, calcolare i suoi coefficienti di Fourier:
\begin{equation*}
    x(t)=\left|\cos\left(\displaystyle\frac{2\pi t}{T}\right)\right|
\end{equation*}

Si determinano i coefficienti dell'espansione di Fourier tramite la definizione:
\begin{gather*}
    c_k=\displaystyle\frac{2}{T}\int_{-\frac{T}{4}}^{\frac{T}{4}}\cos\left(\displaystyle\frac{2\pi t}{T}\right)e^{-i\frac{4\pi kt}{T}}\df t\\
    \displaystyle\frac{2}{T}\int_{-\frac{T}{4}}^{\frac{T}{4}}\cos\left(\displaystyle\frac{2\pi t}{T}\right)\left[\cos\left(\frac{4\pi kt}{T}\right)-i\sin\left(\frac{4\pi kt}{T}\right)\right]\df t\\
    \displaystyle\frac{2}{T}\int_{-\frac{T}{4}}^{\frac{T}{4}}\cos\left(\frac{2\pi t}{T}\right)\cos\left(\frac{4\pi kt}{T}\right)\df t
    {-i\int_{-\frac{T}{4}}^{\frac{T}{4}}\cos\left(\frac{2\pi t}{T}\right)\sin\left(\frac{4\pi kt}{T}\right)\df t}
\end{gather*}
Poiché è una funzione pari, il primo integrale corrisponde al doppio dell'integrale su solo una metà dell'intervallo, mentre il secondo integrale, essendo una funzione 
dispari, assume valore nullo su un intervallo simmetrico:
\begin{equation*}
    \displaystyle\frac{4}{T}\int_{0}^{\frac{T}{4}}\cos\left(\frac{2\pi t}{T}\right)\cos\left(\frac{4\pi kt}{T}\right)\df t-i\cdot0
\end{equation*}


Per le formule di prostaferesi si ottiene una somma di due coseni. Integrandoli separatamente si ottiene:
\begin{gather*}
    c_k=\displaystyle\frac{4}{T}\frac{1}{2}\int_{-\frac{T}{4}}^{\frac{T}{4}}\cos\left(\frac{2(2k+1)\pi t}{T}\right)+\cos\left(\frac{2(2k-1)\pi t}{T}\right)\df t\\
    \displaystyle\left[\frac{2}{T}\frac{T}{2(2k+1)\pi}\sin\left(\frac{2(2k+1)\pi t}{T}\right)+
    \frac{2}{T}\frac{T}{2(2k-1)\pi}\sin\left(\frac{2(2k-1)\pi t}{T}\right)\right]_{0}^{\frac{T}{4}}\\
    \left[\displaystyle\frac{\sin(2(2k+1)\pi t/T)}{(2k+1)\pi}+\frac{\sin(2(2k-1)\pi t/T)}{(2k-1)\pi}\right]_0^{\frac{T}{4}}\\
    c_k=\displaystyle\frac{\sin(2(2k+1)\pi/2)}{(2k+1)\pi}+\frac{\sin(2(2k-1)\pi/2)}{(2k-1)\pi}
\end{gather*}

L'argomento del seno, lo rende tale che assume solo valori pari a $\pm1$. Per valori pari di $k$, il primo seno assume valore di $1$, mentre l'altro $-1$, accade l'opposto per 
valori dispari di $k$. 
I due seni assumono sempre valori discordi e unitari, quindi si possono esprimere come: 
\begin{gather}
    c_k=\displaystyle\frac{1}{\pi}\frac{(-1)^k}{2k+1}-\frac{1}{\pi}\frac{(-1)^k}{2k-1}=-\frac{2}{\pi}\frac{1}{4k^2-1}(-1)^k
\end{gather}

\subsubsection*{Esercizio 15}

Dato un segnale periodico $x(t)$ calcolare i suoi coefficienti di Fourier:
\begin{gather*}
    x(t)=2\cos\displaystyle\left(\frac{2\pi t}{T}+\frac{\pi}{4}\right)+6\sin\left(\frac{2\pi t}{3T}\right)
\end{gather*}
Il primo passaggio nella rappresentazione di Fourier corrisponde all'individuazione del periodo del segnale $x$. Il periodo della somma di due segnali periodici corrisponde 
al minimo comune multiplo tra il periodo dei due segnali. In questo caso il periodo del segnale $x$ è $T'=3T$. Per cui si può riscrivere come:
\begin{gather*}
    x(t)=\displaystyle2\cos\left(\frac{6\pi t}{T'}+\frac{\pi}{4}\right)+6\sin\left(\frac{2\pi t}{T'}\right)\\
    \displaystyle e^{i\left(\frac{6\pi t}{T'}+\frac{\pi}{4}\right)}+e^{-i\left(\frac{6\pi t}{T'}+\frac{\pi}{4}\right)}+\frac{3}{i}e^{i\frac{2\pi t}{T'}}-\frac{3}{i}e^{-i\frac{2\pi t}{T'}}
\end{gather*} 
Per confronto diretto si ottengono i seguenti coefficienti:
\begin{gather}
    c_k=
    \begin{cases}
        0&\forall k\neq\pm1\land\pm3\\
        \displaystyle\Frac{3}{ i}=-3i=-3e^{\frac{\pi}{2}} &k=1\\
        -\Frac{3}{i}=3i=3e^{\frac{\pi}{2}}&k=-1\\
        \displaystyle e^{\frac{\pi}{4}}=\Frac{1}{\sqrt{2}}(1+i)&k=3\\
        \displaystyle e^{-\frac{\pi}{4}}=\Frac{1}{\sqrt{2}}(1-i)&k=-3
    \end{cases}
\end{gather}

\subsubsection*{Esercizio 16}

Dato un segnale $x(t)$, calcolare i suoi coefficienti di Fourier:
\begin{gather*}
    x(t)=\sin\displaystyle(2\pi t)+\cos(4\pi t+\phi)
\end{gather*}
Questo segnale ha un periodo $T=1$. Per confronto diretto si ottengono dei coefficienti:
\begin{gather*}
    x(t)=\displaystyle \frac{e^{2i\pi t}-e^{-2i\pi t}}{2i}+\frac{e^{i\phi}e^{4i\pi t}+e^{-i\phi}e^{-4i\pi t}}{2}\\
    c_k=
    \begin{cases}
        0&\forall k\neq\pm1\land\pm2\\
        \displaystyle\Frac{1}{2i}=-\frac{i}{2} &k=1\\
        -\displaystyle\Frac{1}{2i}=\frac{i}{2}&k=-1\\
        \displaystyle\Frac{e^{i\phi}}{2}&k=2\\
        \displaystyle\Frac{e^{-i\phi}}{2}&k=-2
    \end{cases}\tageq
\end{gather*}

\subsubsection*{Esercizio 17}

Dato il segnale $x(t)$, calcolare i suoi coefficienti di Fourier:
\begin{gather*}
    x(t)=\sin^2\displaystyle\left(\frac{2\pi t}{T}+\phi\right)=\frac{1}{2}\left(1-\cos\left(\frac{4\pi t}{T}+2\phi\right)\right)
\end{gather*}
Questo segnale ha periodo $T/2$. Per confronto diretto si ottengono i coefficienti:
\begin{gather*}
    x(t)=\displaystyle\frac{1}{2}-\frac{1}{4}e^{2i\phi}e^{i\frac{4\pi t}{T}}-\frac{1}{4}e^{-2i\phi}e^{-i\frac{4\pi t}{T}}\\
    c_k=\begin{cases}
        0&\forall k\neq0\land\pm2\\
        \displaystyle\Frac{1}{2}&k=0\\
        -\displaystyle\Frac{1}{4}e^{2i\phi}&k=1\\
        -\displaystyle\Frac{1}{4}e^{-2i\phi}&k=-1
    \end{cases}\tageq
\end{gather*}

\subsubsection*{Esercizio 18}

Data un'onda quadra $z(t)$, che assume valori di $1$ e $-1$ periodicamente, con periodo $T$, calcolare i suoi coefficienti di Fourier. 
Può essere espressa come una differenza tra due onde quadre $x(t)$ e $y(t)$ in 
opposizione di fase che assumono valori di $1$ e $0$ con periodo $T$. Una delle quali ha una base di lunghezza $\tau$, mentre l'altra lunghezza di $T-\tau$ e traslata di un 
semi-periodo $T/2$:
\begin{equation*}
    z(t)=\displaystyle\sum_{k=-\infty}^{+\infty}\left[\rect\left(\frac{t-kT}{\tau}\right)-\rect\left(\frac{t-nT-\frac{T}{2}}{T-\tau}\right)\right]
\end{equation*}
\begin{center}
    \begin{tikzpicture}[scale=2]
        \draw[->](-2.5,0)--(2.5,0)node[above]{$t$};
            \draw[->](0,-1)--(0,1)node[right]{$z(t)$};
            \draw[-](-2.5,0.5)--(-2,0.5)--(-2,-0.5)--(-0.5,-0.5)--(-0.5,0.5)--(0.5,0.5)--(0.5,-0.5)--(2,-0.5)--(2,0.5)--(2.5,0.5);
            \node[above right]at(0,0.5){$1$};
            \node[below right]at(0,-0.5){$-1$};
    \end{tikzpicture}
\end{center}
Si calcolano ora i coefficienti dell'espansione di Fourier:
\begin{gather*}
    c_k=\displaystyle\frac{1}{T}\int_{-\frac{T}{2}}^{\frac{T}{2}}\left[\rect\left(\frac{t-kT}{\tau}\right)-\rect\left(\frac{t-kT-\frac{T}{2}}{T-\tau}\right)\right]\df t\\
    \displaystyle\frac{1}{T}\int_{-\frac{\tau}{2}}^{\frac{\tau}{2}}\rect\left(\frac{t-kT}{\tau}\right)\df t-\frac{1}{T}\int_{-\frac{T-\tau}{2}}^{\frac{T-\tau}{2}}\rect\left(\frac{t-kT-\frac{T}{2}}{T-\tau}\right)\df t\\
    \displaystyle\frac{e^{-i\pi k\frac{\tau}{T}}-e^{i\pi k\frac{\tau}{T}}}{-2i}\frac{1}{\pi k}\frac{\tau}{T}-\frac{e^{-i2\pi k\frac{T-\tau/2}{T}}-e^{i2\pi k\frac{T-\tau/2}{T}}}{-2i}\frac{1}{\pi k}\frac{\tau}{T}
\end{gather*}
Per $k=0$ si ottiene una forma indeterminata, per cui bisogna risolvere l'integrale considerando il valor medio assunto dal segnale nell'intervallo $\left[-T/2,T/2\right]$. 
\begin{gather*}
    \displaystyle\frac{\tau}{T}\sinc\left(\frac{k\tau}{T}\right)-\frac{e^{i\pi k\frac{\tau}{T}}-e^{-i\pi k\frac{\tau}{T}}}{-2i}\frac{1}{\pi k}\\
    c_k=\displaystyle\frac{2\tau}{T}\sinc\left(\frac{k\tau}{T}\right)
\end{gather*}


Il segnale originale può essere espresso come un'onda quadra doppia, di stesso periodo $T$ e base $\tau$, e traslata verso il basso:
\begin{equation*}
    z(t)=\displaystyle\sum_{k=-\infty}^{+\infty}\left[2\,\rect\left(\frac{t-kT}{\tau}\right)\right]-1=2x(t)-1
\end{equation*}
In questo modo si effettuano meno calcoli per determinare i coefficienti di Fourier, usufruendo della proprietà di linearità bisogna tenere conto che il fattore costante $-1$, 
assume un valore non nullo solo per $k=0$. Si considerano i coefficienti del segnale $x$ come $c_k$, mentre del segnale costante $d_k$, per cui in questo caso i coefficienti 
del segnale $z$ si esprimono come: 
\begin{equation}
    a_k=\begin{cases}
        2c_k+d_k &k=0\\
        2c_k&k\neq0
    \end{cases}=\begin{cases}
        \displaystyle\frac{\strut 2\tau}{\strut T}-1&k=0\\
        \displaystyle\frac{\strut 2\tau}{\strut T}\sinc\left(\frac{ k\tau}{ T}\right)&k\neq0
    \end{cases}
\end{equation}

\subsubsection*{Esercizio 19}

Si considera un'onda quadra di periodo $T$ traslata di un fattore $\tau$, corrispondente alla lunghezza della sua base. Dati i coefficienti di un onda quadra non traslata 
$c_k$, calcolare i suoi coefficienti di Fourier. 

I coefficienti $d_k$ del segnale traslato si possono esprimere direttamente come:
\begin{equation}
    d_k=\displaystyle\frac{\tau}{T}\sinc\left(\frac{k\tau}{T}\right)e^{-i\frac{2\pi k\tau}{T}}
\end{equation}

In caso il periodo sia esattamente il doppio della base $T=2\tau$:
\begin{equation}
    d_k=\displaystyle\frac{1}{2}\sinc\left(\frac{k}{2}\right)e^{-i\pi k}=\frac{1}{2}\frac{e^{i\frac{\pi k}{2}}-e^{-i\frac{\pi k}{2}}}{i\pi k}e^{-i\pi k}=
    \frac{e^{-i\frac{\pi k}{2}}-e^{-i\frac{3\pi k}{2}}}{2i\pi k}
\end{equation}
Per cui i coefficienti dell'onda quadra di armoniche di ordine pari risultano nulli. 

\subsubsection*{Esercizio 20}

Dato il segnale dente di sega, calcolare i suoi coefficienti di Fourier:
\begin{center}
    \begin{tikzpicture}[scale=2]
        \draw[->](-2.5,0)--(2.5,0)node[above]{$t$};
        \draw[->](0,-1)--(0,1)node[right]{$x(t)$};
        \draw[-](-2.5,-0.5)--(-1.5,0.5)--(-1.5,-0.5)--(-0.5,0.5)--(-0.5,0)node[above right]{$\displaystyle-\frac{T}{2}$}--(-0.5,-0.5)--(0.5,0.5)--(0.5,0)node[above right]{$\displaystyle\frac{T}{2}$}--(0.5,-0.5)--(1.5,0.5)--(1.5,-0.5)--(2.5,0.5);
        \draw[dashed](-2.5,0.5)--(0,0.5)node[above right]{$\displaystyle\frac{T}{2}$}--(2.5,0.5);
        \draw[dashed](-2.5,-0.5)--(0,-0.5)node[below right]{$-\displaystyle\frac{T}{2}$}--(2.5,-0.5);
    \end{tikzpicture}
\end{center}

Si calcolano i coefficienti di Fourier tramite la definizione, e si calcola l'integrale per parti: 
\begin{gather*}
    c_k=\displaystyle\frac{1}{T}\int_{-\frac{T}{2}}^{\frac{T}{2}}te^{-i\frac{2\pi kt}{T}}\df t=\displaystyle\left[-\frac{1}{T}\frac{T}{2i\pi k}te^{-i\frac{2\pi kt}{T}}\right]_{-\frac{T}{2}}^{\frac{T}{2}}-
    \frac{1}{T}\int_{-\frac{T}{2}}^{\frac{T}{2}}-\frac{T}{2i\pi k}e^{-i\frac{2\pi kt}{T}\df t}\\
    \displaystyle-\frac{T}{2i\pi k}\left(\frac{e^{-i\pi k}+e^{i\pi k}}{2}\right)+\frac{1}{2i\pi k}\left[\frac{T}{2i\pi k}e^{-i\frac{2\pi kt}{T}}\right]_{-\frac{T}{2}}^{\frac{T}{2}}\\
    \displaystyle-\frac{T}{2i\pi k}\left(\frac{e^{i(-\pi k)}+e^{-i(-\pi k)}}{2}\right)-\frac{2iT}{4\pi^2k^2}\left(\frac{e^{i(-\pi k)}-e^{-i(-\pi k)}}{2i}\right)\\
    \displaystyle-\frac{T}{2i\pi k}\cos\left(-\pi k\right)+\frac{2iT}{4\pi^2k^2}\sin(-\pi k)
\end{gather*}
La componente sinusoidale è nulla per ogni valore di $k$, per cui il coefficiente generico si esprime come:
\begin{gather}
    c_k=\displaystyle-\frac{T}{2i\pi k}\cos(-\pi k)=\frac{Ti}{2\pi k}\cos(\pi k)=\frac{Ti}{2\pi k}(-1)^k \;\;k\neq0
\end{gather}
Il coefficiente $c_0$ è nullo, poiché rappresenta il valor medio del periodo. In questo caso essendo un segnale dispari, il valor medio nel periodo è nullo:
\begin{equation}
    c_0=\displaystyle\frac{1}{T}\int_{-\frac{T}{2}}^{\frac{T}{2}}t\,\df t=0
\end{equation}

\subsubsection*{Esercizio 21}

Dato il seguente segnale dente di sega, calcolare i suoi coefficienti di Fourier:
\begin{center}
    \begin{tikzpicture}[scale=2]
        \draw[->](-3,-0.5)--(2,-0.5)node[above]{$t$};
        \draw[->](-0.5,-1.25)--(-0.5,0.75)node[right]{$x(t)$};
        \draw[-](-2.5,-0.5)--(-1.5,0.5)--(-1.5,-0.5)--(-0.5,0.5)--(-0.5,0)--(-0.5,-0.5)--(0.5,0.5)--(0.5,0)--(0.5,-0.5)node[below right]{$T$}--(1.5,0.5)--(1.5,-0.5);
        \draw[dashed](-3,0.5)--(0,0.5)node[above right]{$T$}--(2,0.5);
    \end{tikzpicture}
\end{center}
Si può esprimere come il precedente segnale analizzato, traslato nel tempo e nello spazio di uno stesso fattore $T/2$. La traslazione nello spazio può essere sia un ritardo 
che un anticipo, poiché si ottiene lo stesso segnale risultante:
\begin{equation*}
    y(t)=x\left(t\pm\frac{T}{2}\right)+\displaystyle\frac{T}{2}
\end{equation*}
Si considera in questo caso un ritardo di $T/2$. Conoscendo i coefficienti di $c_k$, si può attuare una traslazione nel tempo, per ottenere i coefficienti della serie di Fourier 
del segnale $y$:
\begin{gather}
    d_k=\begin{cases}
        \displaystyle c_0+\frac{T}{2}=\frac{T}{2} &k=0\\
        c_ke^{-i\frac{2\pi k}{T}\frac{T}{2}}=\displaystyle\frac{T}{2i\pi k}(-1)^ke^{-i\pi k}&k\neq0
    \end{cases}
\end{gather}

\subsubsection*{Esercizio 22}

Dato un segnale treno di triangoli $x(t)$, calcolare i suoi coefficienti di Fourier:
\begin{equation*}
    x(t)=\displaystyle\sum_{k=-\infty}^{+\infty}\tri \left(\frac{2(t-kT)}{\tau}\right)
\end{equation*}
\begin{center}
    \begin{tikzpicture}[scale=2]
        \draw[->](-2.5,0)--(2.5,0)node[above]{$t$};
        \draw[->](0,-0.5)--(0,1.5)node[right]{$x(t)$};

        \draw[-](-2,0)--(-1.5,1)--(-1,0)--(-0.5,0)node[below]{$\displaystyle-\frac{\tau}{2}$}--(0,1)node[above right]{$T$}--(0.5,0)node[below]{$\displaystyle\frac{\tau}{2}$}--(1,0)--(1.5,1)--(2,0);
        \draw[dashed](-2.5,1)--(2.5,1);
    \end{tikzpicture}
\end{center}
Si calcolano i coefficienti tramite la definizione. Poiché è presente un solo triangolo nell'intervallo di integrazione, si può restringere l'intervallo: 
\begin{gather*}
    c_k=\displaystyle\frac{1}{T}\int_{-\frac{T}{2}}^{\frac{T}{2}}\tri \left(\frac{2t}{\tau}\right)e^{-i\frac{2\pi kt}{T}}\df t=\frac{1}{T}\int_{-\frac{\tau}{2}}^{\frac{\tau}{2}}\left(1-\left|\frac{2t}{\tau}\right|\right)e^{-i\frac{2\pi kt}{T}}\df t\\
    \displaystyle\frac{1}{T}\int_{-\frac{\tau}{2}}^{\frac{\tau}{2}}e^{-i\frac{2\pi kt}{T}}\df t+\frac{1}{T}\int_{-\frac{\tau}{2}}^{\frac{\tau}{2}}-\left|\frac{2t}{\tau}\right|e^{-i\frac{2\pi kt}{T}}\df t\\
    \displaystyle\frac{1}{T}\int_{-\frac{\tau}{2}}^{\frac{\tau}{2}}e^{-i\frac{2\pi kt}{T}}\df t-\frac{1}{T}\left(\int_{-\frac{\tau}{2}}^{0}-\frac{2t}{\tau}e^{-i\frac{2\pi kt}{T}}\df t+\int_{0}^{\frac{\tau}{2}}\frac{2t}{\tau}e^{-i\frac{2\pi kt}{T}}\df t\right)
\end{gather*}
Il primo integrale corrisponde all'integrale ai coefficienti di una finestra, ed è stato già precedentemente calcolato, per cui si omette la risoluzione. Si considera la 
sostituzione $t=-t'$ nel secondo integrale:
\begin{gather*}
    \displaystyle\frac{\tau}{T}\sinc\left(\frac{\tau k}{T}\right)-\frac{1}{T}\left(\int_{-\frac{-\tau}{2}}^{0}-\frac{2(-t)}{\tau}e^{-i\frac{2\pi k(-t)}{T}}\df(-t)+\int_{0}^{\frac{\tau}{2}}\frac{2t}{\tau}e^{-i\frac{2\pi kt}{T}}\df t\right)\\
    \displaystyle\frac{\tau}{T}\sinc\left(\frac{\tau k}{T}\right)-\frac{1}{T}\left(-\int_{\frac{\tau}{2}}^{0}\frac{2t}{\tau}e^{i\frac{2\pi kt}{T}}\df t+\int_{0}^{\frac{\tau}{2}}\frac{2t}{\tau}e^{-i\frac{2\pi kt}{T}}\df t\right)\\
    \displaystyle\frac{\tau}{T}\sinc\left(\frac{\tau k}{T}\right)-\frac{1}{T}\left(\int^{\frac{\tau}{2}}_{0}\frac{2t}{\tau}e^{i\frac{2\pi kt}{T}}\df t+\int_{0}^{\frac{\tau}{2}}\frac{2t}{\tau}e^{-i\frac{2\pi kt}{T}}\df t\right)\\
    \displaystyle\frac{\tau}{T}\sinc\left(\frac{\tau k}{T}\right)-\frac{1}{T}\int^{\frac{\tau}{2}}_{0}\frac{4t}{\tau}\frac{e^{i\frac{2\pi kt}{T}}+e^{-i\frac{2\pi kt}{T}}}{2}\df t\\
    \displaystyle\frac{\tau}{T}\sinc\left(\frac{\tau k}{T}\right)-\frac{4}{\tau T}\int_{0}^{\frac{\tau}{2}}t\cos\left(\frac{2\pi kt}{T}\right)\df t
\end{gather*}
Quest'ultimo integrale così ottenuto si risolve mediante integrazione per parti:
\begin{gather*}
    \displaystyle\frac{\tau}{T}\sinc\left(\frac{\tau k}{T}\right)-\frac{4}{\tau T}\left[\frac{T}{2\pi k}t\sin\left(\frac{2\pi kt}{T}\right)\right]^{\frac{\tau}{2}}_0
    +\frac{4}{\tau T}\int_{0}^{\frac{\tau}{2}}\frac{T}{2\pi k}\sin\left(\frac{2\pi kt}{T}\right)\df t\\
    \displaystyle\frac{\tau}{T}\sinc\left(\frac{\tau k}{T}\right)-\frac{\tau}{T}\frac{T}{\pi k\tau}\sin\left(\frac{\pi k\tau}{T}\right)+0+
    \frac{4}{\tau T}\left[-\left(\frac{T}{2\pi k}\right)^2\cos\left(\frac{2\pi kt}{T}\right)\right]^{\frac{\tau}{2}}_0\\
    \displaystyle\frac{\tau}{T}\sinc\left(\frac{\tau k}{T}\right)-\displaystyle\frac{\tau}{T}\sinc\left(\frac{\tau k}{T}\right)-
    \frac{T}{\pi^2k^2\tau}\left[\cos\left(\frac{\pi k\tau}{T}\right)-1\right]\\
    \displaystyle\frac{2T}{\pi^2k^2\tau}\sin^2\left(\frac{\pi kt}{2T}\right)=
    \frac{\tau}{2T}\left[\frac{2T}{\pi k\tau}\sin\left(\frac{\pi kt}{2T}\right)\right]\left[\frac{2T}{\pi k\tau}\sin\left(\frac{\pi kt}{2T}\right)\right]\\
    \displaystyle\frac{\tau}{2T}\left[\sinc\left(\frac{k\tau}{2T}\right)\right]\left[\sinc\left(\frac{k\tau}{2T}\right)\right]\\
    c_k=\displaystyle\frac{\tau}{2T}\sinc^2\left(\frac{k\tau }{2T}\right)\tageq
\end{gather*}
I coefficienti di un treno di triangoli risultano essere dei seni cardinali quadrati. 

\subsection{Trasformata di Fourier}

\subsubsection*{Esercizio 23}

Data una gaussiana traslata nel tempo, calcolare la sua trasformata: 
\begin{gather*}
    x(t)=e^{-\alpha(t-t_0)^2}\\
    X(f)=\displaystyle\sqrt{\frac{\pi}{\alpha}}e^{-\left(\frac{\pi^2f^2}{\alpha}+2i\pi ft_0\right)}\tageq
\end{gather*}

\subsubsection*{Esercizio 24}

Calcolare l'antitrasformata del segnale $X_1(f)$: 
\begin{gather*}
    X_1(f)=\displaystyle\frac{1}{(\alpha+2i\pi f)^2}
\end{gather*}
Per ottenere la trasformata del segnale, si considera il duale della proprietà della derivazione della trasformata. Si considera il segnale esponenziale unilatero e la sua 
trasformata:
\begin{gather*}
    x(t)=e^{-\alpha t}u(t)\\
    X(f)=\displaystyle\frac{1}{\alpha+2i\pi f}\\
    \displaystyle\frac{\df}{\df f}X(f)=\frac{-2i\pi}{(\alpha+2i\pi f)^2}\\
    X_1(f)=\displaystyle-\frac{1}{2i\pi}\frac{\df}{\df f}X(f)\to-\frac{1}{2i\pi}(-2i\pi tx(t))=te^{-\alpha t}u(t)\\
    x_1(t)=te^{-\alpha t}u(t)\tageq
\end{gather*}

\subsubsection*{Esercizio 25}

Calcolare l'energia segnale $x(t)$:
\begin{gather*}
    x(t)=\displaystyle\frac{\sin(\pi t)}{\pi t}\cos(2\pi \alpha t)=\frac{1}{2}\sinc(t)(e^{2i\pi\alpha t}+e^{-2i\pi\alpha t})\\
\end{gather*}
Per il teorema della traslazione in frequenza, la trasformata del segnale $x$ risulta essere:
\begin{gather*}
    X(f)=\displaystyle\frac{1}{2}\rect(f+\alpha)+\frac{1}{2}\rect(f-\alpha)
\end{gather*}
La sua energia si calcola come:
\begin{gather*}
    E_x=\displaystyle\int_{-\infty}^{+\infty}|X(f)|^2\df f\\
    \displaystyle\frac{1}{4}\left(\int_{-\infty}^{+\infty}\left[\rect(f+\alpha)+\rect(f-\alpha)\right]^2\df f\right)\\
    E_x=\begin{cases}
        \displaystyle\frac{1}{4}\int_{-\alpha-1/2}^{-\alpha+1/2}\df f+\frac{1}{4}\int_{\alpha-1/2}^{\alpha+1/2}\df f& \alpha\geq1/2\\
        \displaystyle\frac{1}{4}\int_{-\alpha-1/2}^{\alpha-1/2}\df f+\frac{1}{2}\int_{\alpha-1/2}^{-\alpha+1/2}\df f+\frac{1}{4}\int_{-\alpha+1/2}^{\alpha+1/2}\df f&\alpha<1/2
    \end{cases}\\
    E_x=\begin{cases}
        \displaystyle\frac{\strut 1}{\strut 2}&\alpha\geq1/2\\
        \displaystyle\frac{\strut 2\alpha}{\strut 4}+\frac{\strut 2\alpha}{\strut 4}+1-2\alpha=1-\alpha&\alpha<1/2
    \end{cases}\tageq
\end{gather*}

\subsubsection*{Esercizio 26}

Calcolare la trasformata del segnale $x(t)$: 
\begin{gather*}
    x(t)=e^{3t}u(-t)
\end{gather*}
Tramite la proprietà di ribaltamento e di scala: 
\begin{gather*}
    x_1(t)=e^{-3t}u(t)\\
    X_1(f)=X(-f)=\displaystyle\frac{1}{3+2i\pi f}\\
    X(f)=\displaystyle\frac{1}{3-2i\pi f}\tageq
\end{gather*}

\subsubsection*{Esercizio 27}

Calcolare la trasformata del segnale $x(t)$: 
\begin{gather*}
    x(t)=e^{-2t+4}u(t-2)=e^{-2*(t-2)}u(t-2)
\end{gather*}
Per la proprietà della traslazione nel tempo:
\begin{gather*}
    x_1(t)=e^{-2t}u(t)\\
    X_1(f)=\displaystyle\frac{1}{2+2i\pi f}\\
    X(f)=\displaystyle\frac{1}{2+2i\pi f}e^{-4i\pi f}\tageq
\end{gather*}

\subsubsection*{Esercizio 28}

Calcolare la trasformata del segnale $x(t)$:
\begin{gather*}
    x(t)=e^{-t/2}\cos(100\pi t)u(t)
\end{gather*}
Si applica la proprietà di modulazione:
\begin{gather*}
    x_1(t)=e^{-t/2}u(t)\\
    X_1(f)=\displaystyle\frac{1}{1/2+2i\pi f}\\
    X(f)=\displaystyle\frac{1}{2}\left(X_1(f-50)+X_1(f+50)\right)\\
    X(f)=\displaystyle\frac{1}{2}\frac{1}{1/2+2i\pi (f-50)}+\frac{1}{2}\frac{1}{1/2+2i\pi (f+50)}\\
    X(f)=\displaystyle\frac{1}{1+4i\pi (f-50)}+\frac{1}{1+4i\pi (f+50)}\tageq
\end{gather*}

\subsubsection*{Esercizio 29}

Calcolare l'autoconvoluzione del segnale $x(t)$, passando per la sua trasformata: 
\begin{gather*}
    x(t)=A\,\rect\displaystyle\left(\frac{2(t-3T/4)}{T}\right)-A\,\rect\displaystyle\left(\frac{2(t-T/4)}{T}\right)\\
    X(f)=\displaystyle\frac{AT}{2}\,\sinc\left(\frac{Tf}{2}\right)e^{-3i\pi fT/2}-\displaystyle\frac{AT}{2}\,\sinc\left(\frac{T}{2}f\right)e^{-i\pi fT/2}\\
    X(f)=\displaystyle\frac{AT}{2}\,\sinc\left(\frac{Tf}{2}\right)\left(e^{-3i\pi fT/2}-e^{-i\pi fT/2}\right)\\
    X^2(f)=\displaystyle\frac{A^2T^2}{4}\,\sinc^2\left(\frac{Tf}{2}\right)\left(e^{-3i\pi fT}+e^{-i\pi fT}-2e^{-2i\pi fT}\right)\\
    x(t)*x(t)=\displaystyle\int_{-\infty}^{+\infty}X^2(f)e^{2i\pi ft}\df f\\
    \displaystyle\int_{-\infty}^{+\infty}\left[\frac{A^2T^2}{4}\,\sinc^2\left(\frac{Tf}{2}\right)\left(e^{-3i\pi fT}+e^{-i\pi fT}-2e^{-2i\pi fT}\right)\right]e^{2i\pi ft}\df f
\end{gather*}
Un prodotto nel domino delle frequenze corrisponde ad una convoluzione nel domino del tempo, per cui si può esprimere l'autoconvoluzione come la convoluzione tra un segnale 
triangolo e una combinazione lineare di impulsi:
\begin{gather*}
    X_1(f)=\displaystyle\frac{A^2T^2}{4}\,\sinc^2\left(\frac{Tf}{2}\right)\\
    x_1(t)=\displaystyle\frac{A^2T^2}{4}\frac{2}{T}\,\tri \left(\frac{2t}{T}\right)\\
    X_2(f)=e^{-3i\pi fT}+e^{-i\pi fT}-2e^{-2i\pi fT}\\
    x_2(t)=\delta(t-3T/2)+\delta(t-T/2)-2\delta(t-T)\\
    x(t)*x(t)=\displaystyle\int_{-\infty}^{+\infty}X(f)^2e^{2i\pi ft}\df f=\int_{-\infty}^{+\infty}X_1(f)X_2(f)e^{2i\pi ft}\df f=x_1(t)*x_2(t)\\
    \displaystyle\frac{A^2T^2}{4}\frac{2}{T}\,\tri \left(\frac{2t}{T}\right)*\left[\delta(t-3T/2)+\delta(t-T/2)-2\delta(t-T)\right]\\
    x(t)*x(t)=\displaystyle\frac{A^2T^2}{4}\left(\tri \left(\frac{2}{T}(t-3T/2)\right)+\tri \left(\frac{2}{T}(t-T/2)\right)-2\,\tri \left(\frac{2}{T}(t-T)\right)\right)\tageq
\end{gather*}
%% grafico segnale convoluzione 

\subsubsection*{Esercizio 30}

Calcolare la trasformata del segnale $x(t)$: 
\begin{gather*}
    x(t)=\rect(3t-1/2)=\rect(3(t-1/6))
\end{gather*}
Per la proprietà di scala e di traslazione nel tempo:
\begin{gather}
    X(f)=\displaystyle\frac{1}{3}\sinc\left(\frac{f}{3}\right)e^{-i\pi f/3}
\end{gather}

\subsubsection*{Esercizio 31}

Calcolare la trasformata del segnale $x(t)$, esprimibile come un segnale triangolo moltiplicato per un gradino, oppure una retta moltiplicata per una finestra, si sceglie 
quest'ultima rappresentazione per facilitare il calcolo:
\begin{gather*}
    x(t)=A\,\displaystyle\tri \left(\frac{t+T/2}{T}\right)u(t+T/2)\lor x(t)=A\left(\frac{1}{2}-\frac{t}{T}\right)\rect\left(\frac{t}{T}\right)\\
    \displaystyle\frac{A}{2}\rect\left(\frac{t}{T}\right)-\frac{A}{T}t\,\rect\left(\frac{t}{T}\right)
\end{gather*}
%% grafico segnale
Per la proprietà duale alla derivazione, si calcola il secondo componente: 
\begin{gather*}
    x_1(t)=-2i\pi tx_2(t)\\
    X_1(f)=\displaystyle\frac{\df X_2}{\df f}\\
    x_1(t)=-\displaystyle\frac{A}{T}t\,\rect\left(\frac{t}{T}\right)\\
    x_2(t)=\displaystyle\frac{1}{2i\pi}\frac{A}{T}\rect\left(\frac{t}{T}\right)\\
    X_2(f)=\displaystyle\frac{1}{2i\pi}\frac{A}{T}T\,\sinc(Tf)=\frac{A}{\pi T}\frac{\sin(\pi Tf)}{f}\\
    X_1(f)=\displaystyle\frac{\df X_2}{\df f}=\frac{1}{2i\pi}\frac{A}{\pi T}\frac{\cos(\pi Tf)\pi Tf-\sin(\pi Tf)}{f^2}\\
\end{gather*}
Per cui la trasformata complessiva è:
\begin{gather*}
    X(f)=\displaystyle\frac{AT}{2}\sinc(Tf)+\frac{A}{\pi Tf^2(2\pi i)}\left[\cos(\pi Tf)\pi Tf-\sin (\pi Tf)\right]\\
    \displaystyle\frac{AT}{2\pi Tf}\frac{i}{i}\sin(\pi Tf)+\frac{A}{2i\pi f}\cos(\pi Tf)-\frac{A}{2i\pi f}\frac{\sin(\pi Tf)}{\pi Tf}\\
    \displaystyle\frac{A}{2i\pi f}\frac{i}{i}\left[\cos(\pi Tf)+i\sin(\pi Tf)\right]-\frac{A}{2i\pi f}\frac{i}{i}\sinc( Tf)\\
    X(f)=\displaystyle-\frac{Ai}{2\pi f}\left[e^{i\pi fT}-\sinc(fT)\right]\tageq
\end{gather*}


Calcolare ora questa trasformata usando la proprietà alla derivazione:
\begin{gather*}
    x_1(t)=\displaystyle\frac{A}{2}\rect\left(\frac{t}{T}\right)-\frac{A}{T}t\,\rect\left(\frac{t}{T}\right)\\
    \displaystyle\frac{\df x_1}{\df t}=x_2(t)
\end{gather*}
Il segnale $x_1$ presente una discontinuità in $-T/2$, e rimane costante fino al valore $T/2$:
\begin{gather*}
    x_2(t)=A\delta(t+T/2)-\displaystyle\frac{A}{T}\rect\left(\frac{t}{T}\right)\\
    X_2(f)=Ae^{i\pi fT}-\displaystyle\frac{A}{T}T\sinc(Tf)\\
    X_2(f)=2i\pi fX_1(f)\\
    X_1(f)=\displaystyle\frac{A}{2i\pi f}\frac{i}{i}\left[e^{i\pi fT}-\sinc(Tf)\right]\\
    X_1(f)=\displaystyle-\frac{Ai}{2\pi f}\left[e^{i\pi fT}-\sinc(Tf)\right]\tageq
\end{gather*}

\subsubsection*{Esercizio 32}

Calcolare l'antitrasformata del segnale $X_1(f)$:
\begin{gather*}
    X_1(f)=\displaystyle\frac{1}{2}\left[X(f-f_0)+X(f+f_0)\right]\cos(2\pi f/f_0)
\end{gather*}
Per l'inverso della proprietà di modulazione, e per la proprietà di convoluzione:
\begin{gather*}
    x_1(t)=\left[x(t)\cos(2\pi tf_0)\right]*\left[\displaystyle\frac{1}{2}\delta(t+1/f_0)+\frac{1}{2}\delta(t-1/f_0)\right]\\
    x_1(t)=\displaystyle\frac{1}{2}x\left(t+\frac{1}{f_0}\right)\cos\left[2\pi f_0\left(t+\frac{1}{f_0}\right)\right]+\frac{1}{2}x\left(t-\frac{1}{f_0}\right)\cos\left[2\pi f_0\left(t-\frac{1}{f_0}\right)\right]
\end{gather*}
Poiché il coseno è traslato di un periodo $1/f_0$: 
\begin{gather}
    x_1(t)=\left[\displaystyle\frac{1}{2}x\left(t+\frac{1}{f_0}\right)+\frac{1}{2}x\left(t-\frac{1}{f_0}\right)\right]\cos(2\pi tf_0)
\end{gather}

\subsubsection*{Esercizio 33}

Calcolare la trasformata del segnale $x(t)$:
\begin{gather*}
    x(t)=2\,\rect(2t)-\rect(t)=\rect(2t)-\rect\left[\displaystyle4\left(t-\frac{3}{2}\right)\right]-\rect\left[\displaystyle4\left(t-\frac{3}{8}\right)\right]\\
    X(f)=\displaystyle\frac{1}{2}\sinc\left(\frac{f}{2}\right)-\frac{1}{4}\sinc\left(\frac{f}{4}\right)e^{-3i\pi f/4}-\frac{1}{4}\sinc\left(\frac{f}{4}\right)e^{-3i\pi f}\\
    X(f)=\displaystyle\frac{1}{2}\left[\sinc\left(\frac{f}{2}\right)-\sinc\left(\frac{f}{4}\right)\cos\left(\frac{3\pi f}{4}\right)\right]\tageq
\end{gather*}

\subsubsection*{Esercizio 34}

Calcolare la trasformata del segnale $x(t)$:
\begin{gather*}
    x(t)=\rect\left(\displaystyle\frac{t+T/2}{T}\right)-\rect\left(\displaystyle\frac{t-T/2}{T}\right)
\end{gather*}

\begin{center}
    \begin{tikzpicture}[scale=2]
        \draw[->](-1.5,0)--(1.5,0)node[above]{$t$};
        \draw[->](0,-1)--(0,1)node[right]{$x(t)$};
        \draw[-](-1,0)node[below]{$-T$}--(-1,0.5)--(0,0.5)node[right]{$T$}--(0,-0.5)node[left]{$-T$}--(1,-0.5)--(1,0)node[below right]{$T$};
    \end{tikzpicture}
\end{center}
Per la linearità della trasformata e per la proprietà di traslazione nel tempo la trasformata si ottiene come:
\begin{gather*}
    X(f)=T\left(e^{i\pi fT}-e^{-i\pi fT}\right)\sinc\left(T f\right)=2Ti\sin(\pi fT)\sinc(Tf)\\
    2iT\displaystyle\frac{\sin(\pi fT)}{\pi fT}\sin(\pi fT)=2i\frac{\sin^2(\pi fT)}{\pi f}\\
    X(f)=2i\displaystyle\frac{\sin^"(\pi fT)}{\pi f}\tageq
\end{gather*}

\subsubsection*{Esercizio 35}

Calcolare la trasformata del seguente segnale onda quadra di base $\tau$ e di periodo $T$:
\begin{gather*}
    x(t)=\displaystyle\sum_{k=-\infty}^{+\infty}\rect\left(\frac{t-kT}{\tau}\right)\\
    X(f)=\displaystyle\sum_{k=-\infty}^{+\infty}\frac{\tau}{T}\sinc\left(\frac{\tau k}{T}\right)\delta\left(f-\frac{k}{T}\right)\tageq\\
    c_k=\displaystyle\frac{\tau}{T}\sinc\left(\frac{\tau k}{T}\right)\tageq
\end{gather*}

\subsubsection*{Esercizio 36}

Calcolare la trasformata di un treno di trapezi di base maggiore $2T$, di base minore $T$, di altezza $T$ e di periodo $\overline T=5T/2$:
%% grafico segnale 
\begin{gather*}
    X(f)=\displaystyle\sum_{k=-\infty}^{+\infty}c_k\delta\left(f-\frac{k}{\overline T}\right)
\end{gather*}
Per calcolare i coefficienti della serie, si può esprimere un singolo trapezio come l'autoconvoluzione di un segnale finestra. Inoltre è sufficiente calcolare la 
trasformata di un singolo trapezio per ottenere i coefficienti:
\begin{gather*}
    x_1(t)=2\displaystyle\rect\left(\frac{2t}{3T}\right)\\
    x_2(t)=\displaystyle\rect\left(\frac{2t}{T}\right)\\
    c_k=\displaystyle\frac{1}{\overline T}\int_{-\overline T/2}^{\overline T/2}x_1(t)*x_2(t)e^{-2i\pi kt/\overline T}\df t\\
    c_k=\displaystyle\frac{1}{\overline T}X_1\left(\frac{k}{ T}\right)X_2\left(\frac{k}{ T}\right)\\
    c_k=\displaystyle\frac{2}{\overline T}\left[\frac{3T}{2}\sinc\left(\frac{3T}{2}\frac{k}{ T}\right)\frac{T}{2}\sinc\left(\frac{T}{2}\frac{k}{ T}\right)\right]\\
    c_k=\displaystyle\frac{6T^2}{4}\frac{2}{5T}\sinc\left(\frac{3T}{2}\frac{k}{ T}\right)\sinc\left(\frac{T}{2}\frac{k}{ T}\right)\\
    c_k=\displaystyle\frac{3T}{5}\sinc\left(\frac{3k}{2}\right)\sinc\left(\frac{k}{2}\right)\\
    X(f)=\displaystyle\frac{2}{5T}\sum_{n=-\infty}^{+\infty}\left[\frac{3T}{5}\sinc\left(\frac{3k}{2}\right)\sinc\left(\frac{k}{2}\right)\right]\delta\left(f-\frac{2k}{5T}\right)\tageq
\end{gather*}

\subsubsection*{Esercizio 37}

Calcolare la seguente convoluzione, passando per il dominio della frequenza:
\begin{gather*}
    x(t)=\left[\cos(2\pi f_0t)\sin(2\pi f_0t)\right]*\displaystyle\frac{\sin(3\pi f_0t)}{\pi t}\\
    \left[\cos(2\pi f_0t)\sin(2\pi f_0t)\right]*\displaystyle3f_0\frac{\sin(3\pi f_0t)}{3\pi f_0t}\\
    \left[\cos(2\pi f_0t)\sin(2\pi f_0t)\right]*3f_0\sinc(3f_0 t)
\end{gather*}
Si applica la formula di duplicazione del seno, e si passa per il dominio della frequenza:
\begin{gather*}
    x(t)=\displaystyle\frac{3f_0}{2}\sin(4\pi f_0t)*\sinc(3f_0t)\\
    X(f)=\displaystyle\frac{1}{4i}\left[\delta(f-2f_0)-\delta(f+2f_0)\right]\frac{3f_0}{3f_0}\rect\left(\frac{f}{3f_0}\right)\\
    \displaystyle\frac{1}{4i}\left[\rect\left(-\frac{2f_0}{3f_0}\right)\delta(f-2f_0)-\rect\left(\frac{2f_0}{3f_0}\right)\delta(f+2f_0)\right]\\
    \displaystyle\frac{1}{4i}\rect\left(\frac{2}{3}\right)\left[\delta(f-2f_0)-\delta(f+2f_0)\right]
\end{gather*}
Poiché il valore $2/3$ è esterno all'intervallo di valori non nulli del segnale finestra $[-1/2,1/2]$, la trasformata assume valore nullo. 
\begin{gather}
    x(t)=0
\end{gather}

\subsubsection*{Esercizio 38}

Calcolare il segnale $x(t)$, passando per il dominio della frequenza:
\begin{gather*}
    x(t)=\cos(2\pi f_0t)\left[\sin(2\pi f_0t)*\displaystyle\frac{\sin(3\pi f_0t)}{\pi t}\right]\\
    x_1(t)=\sin(2\pi f_0t)*\displaystyle\frac{\sin(3\pi f_0t)}{\pi t}\\
    X_1(f)=\displaystyle\frac{1}{2i}\left[\delta(f-f_0)-\delta(f+f_0)\right]\cdot \frac{3f_0}{3f_0}\rect\left(\frac{f}{3f_0}\right)\\
    \displaystyle\frac{1}{2i}\delta(f-f_0)\rect\left(-\frac{f_0}{3f_0}\right)-\frac{1}{2i}\delta(f+f_0)\rect\left(\frac{f_0}{3f_0}\right)\\
    \displaystyle\frac{1}{2i}\rect\left(\frac{1}{3}\right)\left[\delta(f-f_0)-\delta(f+f_0)\right]\\
    X_1(f)=\displaystyle\frac{1}{2i}\left[\delta(f-f_0)-\delta(f+f_0)\right]\\
    x_1(t)=\sin(2\pi f_0t)
\end{gather*}
Per la formula di duplicazione del seno: 
\begin{gather}
    x(t)=\cos(2\pi f_0t)\sin(2\pi f_0t)=\displaystyle\frac{\sin(4\pi f_0t)}{2}
\end{gather}

\subsubsection*{Esercizio 39}

Dato il segnale $x(t)$, calcolare il periodo, i coefficienti di Fourier, la potenza, e la sua trasformata:
\begin{gather*}
    x(t)=2e^{2i\pi t/T_0}-e^{-i\pi t/2T_0}=2e^{2i\pi t/T_0}-e^{-2i\pi t/4T_0}
\end{gather*}
Il periodo di un segnale combinazione lineare di altri segnali è il minimo comune multiplo tra i periodi dei segnali considerati:
\begin{gather}
    T=4T_0
\end{gather}
Si calcola la potenza del segnale:
\begin{gather*}
    P_x=\displaystyle\frac{1}{4T_0}\int_{-2T_0}^{2T_0}\left|2e^{2i\pi t/T_0}-e^{-2i\pi t/4T_0}\right|^2\df t\\
    \displaystyle\frac{1}{4T_0}\int_{-2T_0}^{2T_0}\left(4+1+4\cos\left[\frac{2\pi t}{T_0}\left(1+\frac{1}{4}\right)\right]\right)\df t\\
    \displaystyle\frac{20T_0}{4T_0}+\frac{2T_0}{5\pi }\cancelto{0}{\left[\sin\left(\frac{5\pi t}{2T_0}\right)\right]_{-2T_0}^{2T_0}}\\
    P_x=5\tageq
\end{gather*}
Si determinano i coefficienti di Fourier:
\begin{gather*}
    c_k=\displaystyle\frac{1}{4T_0}\int_{-2T_0}^{2T_0}\left(2e^{2i\pi t/T_0}-e^{-2i\pi t/4T_0}\right)e^{-2i\pi kt/4T_0}\df t\\
    c_k=\displaystyle\frac{2}{4T_0}\int_{-2T_0}^{2T_0}e^{2i\pi t(1-k/4)/T_0}\df t+\frac{1}{4T_0}\int_{-2T_0}^{2T_0}2e^{-2i\pi t(1+k)/4T_0}\df t\\
    c_k=\begin{cases}
        0 &\forall k\neq4\lor-1\\
        \displaystyle\frac{\strut 1}{\strut 4T_0}& k=-1\\
        \displaystyle\frac{\strut 1}{\strut 2T_0}&k=4
    \end{cases}\tageq
\end{gather*}
Si ottiene lo stesso risultato, più velocemente, per confronto diretto. La sua trasformata si ottiene sia trasformando la sua serie di Fourier, sia attuando l'integrale 
di trasformazione:
\begin{gather}
    X(f)=\displaystyle\frac{1}{4T_0}\delta\left(f+\frac{1}{4T_0}\right)+\frac{1}{2T_0}\delta\left(f-\frac{1}{T_0}\right)
\end{gather}

\subsubsection*{Esercizio 40}

Calcolare la trasformata del segnale $x$:
\begin{gather*}
    x(t)=\cos\left[\displaystyle\frac{\pi(t-T)}{2T}\right]\cos\left(\frac{2\pi t}{T}\right)=\cos\left[\displaystyle\frac{2\pi(t-T)}{4T}\right]\cos\left(\frac{2\pi t}{T}\right)
\end{gather*}
Questo segnale ha periodo $4T$. Espandendo l'argomento del primo coseno si ottiene:
\begin{gather*}
    x(t)=\displaystyle\cos\left[\frac{2\pi t}{4T}-\frac{\pi}{2}\right]\cos\left(\frac{2\pi t}{T}\right)\\
    \sin\left(\displaystyle\frac{2\pi t}{4T}\right)\cos\left(\frac{2\pi t}{T}\right)
\end{gather*}
Si rappresenta il segnale in forma esponenziale:
\begin{gather*}
    x(t)=\displaystyle\frac{1}{2i}\left(e^{2i\pi t/4T}-e^{-2i\pi t/4T}\right)\cdot\displaystyle\frac{1}{2}\left(e^{2i\pi t/T}+e^{-2i\pi t/T}\right)\\
    \displaystyle\frac{1}{4i}\left[e^{\frac{2\pi t}{T}\left(\frac{1}{4}+1\right)}-e^{\frac{2\pi t}{T}\left(1-\frac{1}{4}\right)}+e^{\frac{2\pi t}{T}\left(\frac{1}{4}-1\right)}-e^{-\frac{2\pi t}{T}\left(1+\frac{1}{4}\right)}\right]\\
    \displaystyle\frac{1}{4i}\left[e^{\frac{2\pi t}{T}\frac{5}{4}}-e^{\frac{2\pi t}{T}\frac{3}{4}}+e^{-\frac{2\pi t}{T}\frac{3}{4}}-e^{-\frac{2\pi t}{T}\frac{5}{4}}\right]\\
    X(f)=\displaystyle\frac{1}{4i}\left[\delta\left(f+\frac{5}{4T}\right)-\delta\left(f+\frac{3}{4T}\right)+\delta\left(f-\frac{3}{4T}\right)-\delta\left(f-\frac{5}{4T}\right)\right]\\
    X(f)=\displaystyle\frac{1}{4i}\left[\delta\left(f+\frac{5}{4T}\right)-\delta\left(f-\frac{5}{4T}\right)\right]-\frac{1}{4i}\left[\delta\left(f+\frac{3}{4T}\right)-\delta\left(f-\frac{3}{4T}\right)\right]\tageq
\end{gather*}
Per cui il segnale originale corrisponde ad una differenza di due seni, considerando operazioni trigonometriche del segnale originario. Alternativamente, per la 
proprietà della convoluzione, si può calcolare come convoluzione tra le due trasformate dei segnali moltiplicati:
\begin{gather*}
    x(t)=\cos\left[\displaystyle\frac{\pi(t-T)}{2T}\right]\cos\left(\frac{2\pi t}{T}\right)=x_1(t)\cdot x_2(t)\\
    X(f)=X_1(f)*X_2(f)\\
    X_1(f)=\displaystyle\frac{1}{2}\left[\delta\left(f+\frac{1}{4T}\right)+\delta\left(f-\frac{1}{4T}\right)\right]e^{-2i\pi fT}\\
    \displaystyle\frac{1}{2}e^{-2i\pi/4}\delta\left(f+\frac{1}{4T}\right)+\frac{1}{2}e^{2i\pi /4}\delta\left(f-\frac{1}{4T}\right)\\
    \displaystyle-\frac{i}{2}\delta\left(f+\frac{1}{4T}\right)+\frac{i}{2}\delta\left(f-\frac{1}{4T}\right)\\
    X_2(f)=\displaystyle\frac{1}{2}\left[\delta\left(f+\frac{1}{T}\right)+\delta\left(f-\frac{1}{T}\right)\right]
\end{gather*}
Si attua quindi una convoluzione nel dominio della frequenza per ottenere la trasformata del segnale originale:
\begin{gather*}
    X(f)=\displaystyle\frac{i}{2}\left[\delta\left(f-\frac{1}{4T}\right)-\delta\left(f+\frac{1}{4T}\right)\right]*\frac{1}{2}\left[\delta\left(f+\frac{1}{T}\right)+\delta\left(f-\frac{1}{T}\right)\right]\\
    X(f)=\displaystyle\frac{i}{4}\left[-\delta\left(f+\frac{5}{4T}\right)+\delta\left(f+\frac{3}{4T}\right)-\delta\left(f-\frac{3}{4T}\right)+\delta\left(f-\frac{5}{4T}\right)\right]\tageq
\end{gather*}

\subsubsection*{Esercizio 41}

Dato un filtro di risposta impulsiva $h(t)$, calcolare la sua funzione di trasferimento, e la risposta ad un gradino:
\begin{gather*}
    h(t)=\delta(t-2T)+\delta(t)-\delta(t-T)\\
    H(f)=e^{-4i\pi fT}+1-e^{-2i\pi fT}\tageq\\
    h(t)*u(t)=\delta(t-2T)+\delta(t)-\delta(t-T)*u(t)=\delta(t-2T)u(t-2T)+\delta(t)u(t)-\delta(t-T)u(t-T)\tageq
\end{gather*}
\begin{center}
    \begin{tikzpicture}[scale=2]
        \draw[->](-0.5,0)--(2,0)node[above]{$t$};
        \draw[->](0,-0.25)--(0,1.5)node[right]{$y(t)$};
        \draw[-](0,1)node[left]{$1$}--(0.5,1)--(0.5,0)node[below]{$T$}--(1,0)node[below]{$2T$}--(1,1)--(2,1);
    \end{tikzpicture}
\end{center}

\subsubsection*{Esercizio 42}

Calcolare l'uscita $Y(f)$ di un sistema di risposta impulsiva $h(t)$ con un'entrata $x(t)$:
\begin{gather*}
    h(t)=\displaystyle\frac{1}{T}\sinc\left(\frac{t}{T}\right)\\
    x(t)=\cos(2\pi f_0t)
\end{gather*}
Per la proprietà di scala:
\begin{gather*}
    H(f)=\rect(Tf)\\
    X(f)=\displaystyle\frac{1}{2}\delta(f-f_0)+\frac{1}{2}\delta(f+f_0)\\
    Y(f)=\frac{1}{2}\rect(Tf)(\delta(f-f_0)+\delta(f+f_0))\tageq
\end{gather*}

\subsection{Trasformata di Fourier Tempo Discreto}

\subsubsection*{Esercizio 43}

Calcolare la trasformata della sequenza $x[n]$:
\begin{gather*}
    x[n]=\left[\displaystyle\frac{\sin(2\pi f_cnT)}{\pi n}\right]^2=4f_c^2T^2\sinc^2(2f_cnT)
\end{gather*}
Questa sequenza si può considerare estratta dal segnale tempo continuo $x$:
\begin{gather*}
    x(t)=4f_c^2T^2\sinc(2f_ct)\\
    X(f)=\displaystyle 2f_cT^2\tri \left(\frac{f}{2f_c}\right)
\end{gather*}

Per cui si esprime la trasformata della sequenza come:
\begin{gather}
    X_c(f)=\displaystyle\frac{1}{T}\sum_{n=-\infty}^{+\infty}X\left(f-\frac{n}{T}\right)=2f_cT\sum_{n=-\infty}^{+\infty}\tri \left(\frac{f-\frac{n}{T}}{2f_c}\right)
\end{gather}
Se $T$ il passo di campionamento non è abbastanza grande, i triangoli si sovrappongono tra di loro. 

\subsubsection*{Esercizio 44}

Calcolare la convoluzione $y$ tempo discreta tra $x$ e $h$, esponenziali unilateri: 
\begin{gather*}
    y[n]=x[n]*h[n]\\
    x[n]=a^nu[n]\\
    h[n]=b^nu[n]\\
    y[n]=\displaystyle\sum_{k=-\infty}^{+\infty}x[k]h[n-k]=\sum_{k=-\infty}^{+\infty}a^ku[k]b^{n-k}u[n-k]=\sum_{k=0}^na^kb^{n-k}\\
    \displaystyle b^n\sum_{k=0}^n\left(\frac{a}{b}\right)^k=b^n\frac{1-\left(\frac{a}{b}\right)^{n+1}}{1-\frac{a}{b}}\\
    y[n]=\displaystyle\frac{b^{n+1}-a^{n+1}}{b-a}u[n]\tageq
\end{gather*}
Si può calcolare analogamente nel dominio della frequenza, poi attuando un antitrasformata: 
\begin{gather*}
    Y_c(f)=X_c(f)H_c(f)\\
    X_c(f)=\displaystyle\sum_{n=0}^{+\infty}a^ne^{-2i\pi nfT}=\sum_{n=0}^{+\infty}\left(ae^{-2i\pi fT}\right)^n=\frac{1}{1-ae^{-2i\pi fT}}\\
    H_c(f)=\displaystyle\frac{1}{1-be^{-2i\pi fT}}
\end{gather*}
Poiché le due sommatorie diventano delle serie geometriche. Si considera ora il loro prodotto:
\begin{gather*}
    Y_c(f)=X_c(f)H_c(f)=\displaystyle\frac{1}{1-ae^{-2i\pi fT}}\frac{1}{1-be^{-2i\pi fT}}
\end{gather*}
Si può decomporre in fratti semplici come:
\begin{gather*}
    Y_c(f)=\displaystyle\frac{A}{1-ae^{-2i\pi fT}}+\frac{B}{1-be^{-2i\pi fT}}\\
    \begin{cases}
        A+B=1\\
        Ab+Ba=0
    \end{cases}\to
    \begin{cases}
        A=1-B\\
        b-Bb+Ba=0
    \end{cases}\to\begin{cases}
        A=\displaystyle\frac{\strut a}{\strut a-b}\\
        B=\displaystyle\frac{\strut b}{\strut b-a}
    \end{cases}
\end{gather*}
Per cui il segnale convoluzione si esprime come:
\begin{gather*}
    Y_c(f)=\displaystyle\frac{a}{a-b}\frac{1}{1-ae^{-2i\pi fT}}+\frac{b}{b-a}\frac{1}{1-be^{-2i\pi fT}}\\
    y[n]=\displaystyle\frac{a}{a-b}a^nu[n]+\displaystyle\frac{b}{b-a}b^nu[n]\tageq
\end{gather*}

\subsubsection*{Esercizio 45}

Calcolare la trasformata di Fourier della sequenza $x[n]$:
\begin{gather*}
    x[n]=\displaystyle\frac{\sin(\pi n/3)}{\pi n}\cos\left(\frac{2\pi n}{6}\right)
\end{gather*}
Si considera $T=1$, e si esprime il segnale da cui è stata campionata la sequenza:
\begin{gather*}
    x(t)=\frac{1}{3}\sinc\left(\frac{t}{3}\right)\cos\left(\frac{2\pi t}{6}\right)
\end{gather*}
Si determina la trasformata di questo segnale tempo continuo:
\begin{gather*}
    X(f)=\displaystyle\rect(3f)*\frac{1}{2}\left[\delta\left(f-\frac{1}{6}\right)+\delta\left(f+\frac{1}{6}\right)\right]\\
    X(f)=\displaystyle\frac{1}{2}\rect\left(3f-\frac{1}{2}\right)+\frac{1}{2}\rect\left(3f+\frac{1}{2}\right)\\
    X_c(f)=\displaystyle\frac{1}{T}\sum_{n=-\infty}^{+\infty}X\left(f-\frac{n}{T}\right)=\frac{1}{T}\sum_{n=-\infty}^{+\infty}\left[\frac{1}{2}\rect\left(3f-\frac{1}{2}-3n\right)+\frac{1}{2}\rect\left(3f+\frac{1}{2}-3n\right)\right]\tageq
\end{gather*}

\subsubsection*{Esercizio 46}

Calcolare la trasformata di Fourier della sequenza $x[n]$:
\begin{gather*}
    x[n]=\left[\displaystyle\frac{\sin(\pi n/3)}{\pi n}\right]^2=\frac{1}{9}\sinc^2\left(\frac{n}{3}\right)\\
    T=1\\
    x(t)=\displaystyle\frac{1}{9}\sinc^2\left(\frac{t}{3}\right)\\
    X(f)=\displaystyle \frac{1}{3}\tri (3f)\\
    X_c(f)=\displaystyle\frac{1}{3}\sum_{n=-\infty}^{+\infty}\tri (3f-n)\tageq
\end{gather*}

\subsection{Campionamento}

\subsubsection*{Esercizio 47}

Dato un segnale coseno di frequenza $f_0$ si determina la frequenza di Nyquist:
\begin{gather*}
    x(t)=\cos(2\pi f_0t)\\
    X(f)=\displaystyle\frac{1}{2}\left[\delta(f-f_0)+\delta(f+f_0)\right]
\end{gather*}
Per cui la frequenza massima corrisponde a $f_0$, per cui la frequenza di Nyquist si ottiene come:
\begin{gather}
    f_c=2f_0
\end{gather}
Il segnale campionato corrisponde a:
\begin{gather*}
    x_c(t)=\displaystyle\sum_{n=-\infty}^{+\infty}x(nT)\delta(t-nT)=\sum_{n=-\infty}^{+\infty}\cos(\pi n)\delta\left(t-\frac{n}{2f_0}\right)
\end{gather*}
Ma utilizzando la frequenza di Nyquist come la frequenza di campionamento gli impulsi coincidono per $f=2f_0$, per cui questa sequenza non individua univocamente il segnale 
coseno, per cui è necessaria una frequenza di campionamento maggiore. Si considera allora:
\begin{gather*}
    f_c=4f_0
\end{gather*}
Si suppone sia presente aliasing nel segnale campionato in frequenza. Per cui si ha una frequenza $f_c<2f_0$, e sono presenti degli impulsi in più nell'intervallo $[-f_0,f_0]$, 
corrispondenti alle repliche antecedenti e posteriori alla replica centrata in $f=0$. Lo spettro si sovrappone, quindi ricomponendo il segnale non si ricostruisce il 
segnale coseno originario. 
Il segnale ricostruito si ottiene antitrasformando il segnale ottenuto in frequenza:
\begin{gather*}
    Y(f)=X_c(f)H(f)=\displaystyle\frac{1}{2}\left(f-\frac{1}{T}-f_0\right)+\frac{1}{2}\left(f-\frac{1}{T}+f_0\right)\\
    y(t)=\cos\left[2\pi \left(f_0-\frac{1}{T}\right)t\right]\tageq
\end{gather*}

\subsubsection*{Esercizio 48}

Dato il segnale campionato $x(t)$, determinare la frequenza di Nyquist: 
\begin{gather*}
    x(t)=1+\left[\displaystyle\frac{\sin(20\pi t)}{\pi t}\right]^2=1+400\,\sinc^2\left(20\pi t\right)
\end{gather*}
Lo spettro del segnale corrisponde a:
\begin{gather*}
    X(f)=\displaystyle \delta(f)+20\,\tri \left(\frac{f}{20}\right)
\end{gather*}
Per cui la frequenza minima di campionamento corrisponde a $f_c>40\,\mathrm{Hz}$, mentre il tempo di campionamento massimo corrisponde a:
Non è presente aliasing quando la frequenza assume valori maggiori di:
\begin{gather*}
    f>\displaystyle\frac{2}{\frac{1}{20}}=40\,\mathrm{Hz}
\end{gather*}
Il tempo di campionamento deve essere:
\begin{gather}
    T>\displaystyle\frac{2}{f_{\max}}=\frac{2}{40\,\mathrm{Hz}}=25\,\mathrm{ms}
\end{gather}
Si suppone si stia campionando con un passo di campionamento doppio al valore massimo $\overline{T}=2T=50\,\mathrm{ms}$. Lo spettro del segnale campionato corrisponde a:
\begin{gather*}
    X_c(f)=\displaystyle \frac{1}{\overline{T}}\sum_{n=-\infty}^{+\infty}X\left(f-\frac{n}{\overline{T}}\right)=20\sum_{n=-\infty}^{+\infty}\delta(f-20n)+20\,\tri \left(\frac{f-20n}{20}\right)
\end{gather*}
Se il tempo di campionamento fosse pari a $T=50\times10^{-3}\mathrm{s}$, i triangoli in frequenza sarebbero sfasati tra di loro di esattamente metà della loro base, per cui la 
somma di tutti i triangoli assume valore costante in frequenza. 
Lo spettro del segnale campionato diventa una costante sommata ad un treno di delta:
\begin{gather}
    X_c(f)=400+20\displaystyle\sum_{n=-\infty}^{+\infty}\delta\left(f-\frac{n}{20}\right)
\end{gather}
Si effettua ora quest'analisi nel dominio del tempo:
\begin{gather*}
    x_c(t)=x(t)\pi(t)=x(t)\displaystyle\sum_{n=-\infty}^{+\infty}x(nT)\delta(t-nT)=\sum_{n=-\infty}^{+\infty}\left[1+400\,\sinc^2\left(20t\right)\right]\delta(t-50\times10^{-3}n)\\
    x_c(t)=\displaystyle\sum_{n=-\infty}^{+\infty}\delta(t-50\times10^{-3}n)+400\sum_{n=-\infty}^{+\infty}\sinc^2(20t)\delta(t-50\times10^{-3}n)\\
    x_c(t)=\displaystyle\sum_{n=-\infty}^{+\infty}\delta(t-50\times10^{-3}n)+400\delta(t)
\end{gather*}
La sua trasformata corrisponde a:
\begin{gather}
    X_c(f)=\displaystyle\sum_{n=-\infty}^{+\infty}\delta\left(f-\frac{n}{50\times10^{-3}}\right)\frac{1}{50\times10^{-3}}+400
\end{gather}

\subsubsection*{Esercizio 49}

Dato il segnale $x(t)$, si vuole calcolare il massimo intervallo di campionamento:
\begin{gather*}
    x(t)=\cos^2\left(\displaystyle\frac{2\pi t}{T_0}\right)=\frac{1}{2}+\frac{1}{2}\cos\left(\frac{4\pi t}{T_0}\right)
\end{gather*}

Il periodo del segnale corrisponde a $T=\displaystyle\frac{T_0}{2}$. La sua trasformata equivale a:
\begin{gather*}
    X(f)=\displaystyle\frac{1}{2}\delta(f)+\frac{1}{4}\delta\left(f+\frac{2}{T_0}\right)+\frac{1}{4}\delta\left(f-\frac{2}{T_0}\right)
\end{gather*}
Quindi la frequenza di Nyquist corrisponde a:
\begin{gather*}
    f<2\frac{2}{T_0}=\frac{4}{T_0}
\end{gather*}
Per cui il passo di campionamento del segnale è:
\begin{gather}
    T>\displaystyle\frac{1}{f_c}=\frac{T_0}{4}
\end{gather}

Si considera un passo di campionamento minore rispetto al valore necessario per non avere aliasing: 
\begin{gather*}
    T=\frac{4}{9}T_0
\end{gather*}
Si vuole calcolare lo spettro del segnale considerando questo tempo di campionamento:
\begin{gather*}
    X_c(f)=\displaystyle\frac{9}{4T_0}\sum_{n=-\infty}^{+\infty}X\left(f-\frac{9n}{4T_0}\right)\\
    X_c(f)=\sum_{n=-\infty}^{+\infty}\left[\frac{9}{8T_0}\delta\left(f-\frac{9n}{4T_0}\right)+\frac{9}{16T_0}\delta\left(f+\frac{2}{T_0}-\frac{9n}{4T_0}\right)+\frac{9}{16T_0}\delta\left(f-\frac{2}{T_0}-\frac{9n}{4T_0}\right)\right]
\end{gather*}
\begin{gather}
    X_c(f)=\displaystyle\sum_{n=-\infty}^{+\infty}\left[\frac{9}{8T_0}\delta\left(f-\frac{9n}{4T_0}\right)+\frac{9}{16T_0}\delta\left(f+\frac{8-9n}{4T_0}\right)+\frac{9}{16T_0}\delta\left(f-\frac{8+9n}{4T_0}\right)\right]
\end{gather}
%% grafico 

\subsubsection*{Esercizio 50}

Dato il segnale $x(t)$ calcolare il periodo di campionamento:
\begin{gather*}
    x(t)=2f_0\sinc(2f_0t)\cos(6\pi f_0t)\\
    X(f)=\displaystyle2\,\tri \left(\frac{f}{f_0}\right)*\left[\frac{1}{2}\delta\left(f-3f_0\right)+\frac{1}{2}\delta(f+3f_0)\right]\\
    X(f)=\displaystyle\tri \left(\frac{f-3f_0}{f_0}\right)+\tri \left(\frac{f+3f_0}{f_0}\right)
\end{gather*}
La frequenza di Nyquist corrisponde a $6f_0$, per cui il tempo di campionamento corrisponde a $T=\displaystyle\frac{1}{6f_0}$. Si considera il segnale campionato con un 
passo di campionamento:
\begin{gather*}
    T=\displaystyle\frac{1}{4f_0}
\end{gather*}

Per cui la trasformata del segnale campionato corrisponde a:
\begin{gather*}
    X_c(f)=\displaystyle\sum_{n=-\infty}^{+\infty}\tri \left(\frac{f-(4n+3)f_0}{f_0}\right)+\tri \left(\frac{f-(4n-3)f_0}{f_0}\right)
\end{gather*}
%% grafico 

Si ricostruisce il segnale inserendolo in un filtro ricostruttore DAC di funzione di trasferimento $H(f)$, filtro passa basso ideale, che si estende tra le 
frequenze $[-2f_0,+2f_0]$, si considera un tempo di campionamento pari ad $1$:
\begin{gather*}
    H(f)=\displaystyle\rect\left(\frac{f}{2f_0}\right)\\
    X(f)=X_c(f)H(f)=\displaystyle\tri \left(\frac{f-f_0}{f_0}\right)+\tri \left(\frac{f+f_0}{f_0}\right)\\
    x(t)=2f_0\sinc^2(f_0t)\cos(2\pi f_0t)\tageq
\end{gather*} 

\subsubsection*{Esercizio 51}

Dato un segnale che occupa una banda $[-6\,\mathrm{kHz}, 6\,\mathrm{kHz}]$. Il segnale viene campionato con passo $T$. Il segnale viene filtrato per essere ricostruito da un filtro 
la cui funzione di trasferimento assume una forma trapezoidale in frequenza:
\begin{gather*}
    H(f)=\begin{cases}
        0 &f<-2K_f\land f\geq 2K_f\\
        A\left(2-\left|\displaystyle\frac{f}{K_f}\right|\right) &-2K_f\leq f<-K_f\land K_f\leq f<2K_f\\
        A & -K_f\leq f<K_f
    \end{cases}
\end{gather*}
Calcolare il valore massimo di $T$, $K_f$ e $A$ affinché il segnale ricostruito coincida con il segnale di partenza. 
Il parametro $A$ affinché si ritorni al segnale originario deve essere necessariamente: 
\begin{gather}
    A=\displaystyle\frac{1}{T}
\end{gather}
Il massimo passo di campionamento se fosse presente un filtro ideale sarebbe $T=1/12\times10^{-3}\,\mathrm{s}$, ma per ricostruire il segnale originale con un filtro trapezoidale, bisognerebbe 
assegnare al valore $K_f=6\,\mathrm{kHz}$. Ma in questo modo il filtro del DAC considera altre repliche, poiché la sua banda arriva fino ad un valore di $2K_f=12\,\mathrm{kHz}$. La frequenza di 
campionamento deve essere quindi:
\begin{gather*}
    f\geq\displaystyle\frac{1}{T}+K_f=12+6=18\,\mathrm{kHz}
\end{gather*}
Il passo di campionamento è quindi:
\begin{gather}
    T=\displaystyle\frac{1}{f}=55\,\mu \mathrm{s}
\end{gather}

\subsubsection*{Esercizio 52}

Un segnale viene campionato con frequenza di campionamento (di Nyquist) $f_c=100\,\mathrm{Hz}$, dando origine ai seguenti campioni:
\begin{gather*}
    T_c=\displaystyle\frac{1}{100}\,s=10\,\mathrm{ms}\\
    f_{\max}=\frac{f_c}{2}=50\,\mathrm{Hz}\\
    x(nT)=\begin{cases}
        -1&n=-2,-1\\
        1&n=1,2\\
        0&\forall n\neq\pm1,\pm2
    \end{cases}
\end{gather*}
Il segnale viene ricostruito con un filtro ideale di funzione di trasferimento $H$:
\begin{gather*}
    H(f)=T\,\rect(Tf)\to  h(t)=\sinc\left(\frac{t}{T}\right)
\end{gather*} 
Calcolare il valore del segnale ricostruito al tempo $t=5\,\mathrm{ms}$. Il segnale campionato si esprime come:
\begin{gather*}
    x_c(t)=\displaystyle\sum_{n=-\infty}^{+\infty}x(nT)\delta(t-nT)\\
    x(2T)\delta(t-2T)+x(T)\delta(t-T)+x(-T)\delta(t+T)+x(-2T)\delta(t+2T)\\
    x_c(t)=\delta(t-2T)+\delta(t-T)-\delta(t+T)-\delta(t+2T)
\end{gather*}
Si calcola l'uscita del filtro DAC:
\begin{gather*}
    y(t)=x_c(t)*h(t)=\left[\delta(t-2T)+\delta(t-T)-\delta(t+T)-\delta(t+2T)\right]*\sinc\left(\displaystyle\frac{t}{T}\right)\\
    \displaystyle\sinc\left(\frac{t-2T}{T}\right)+\sinc\left(\frac{t-T}{T}\right)-\sinc\left(\frac{t+T}{T}\right)-\sinc\left(\frac{t+2T}{T}\right)\\
    y(5\,\mathrm{ms})=\displaystyle\sinc\left(\frac{5-20}{10}\right)+\sinc\left(\frac{5-10}{10}\right)-\sinc\left(\frac{5+10}{10}\right)-\sinc\left(\frac{5+20}{10}\right)\\
    \displaystyle\sinc\left(\frac{15}{10}\right)+\sinc\left(\frac{5}{10}\right)-\sinc\left(\frac{15}{10}\right)-\sinc\left(\frac{25}{10}\right)\\
    \displaystyle\sinc\left(\frac{1}{2}\right)-\sinc\left(\frac{5}{2}\right)=\frac{\sin\left(\frac{\pi}{2}\right)}{\frac{\pi}{2}}-\frac{\sin\left(\frac{5\pi}{2}\right)}{\frac{5\pi}{2}}=\frac{2}{\pi}-\frac{2}{5\pi}\\
    y(5\,\mathrm{ms})=\displaystyle\frac{8}{5\pi}\tageq
\end{gather*}

\subsubsection*{Esercizio 53}

Dato un segnale $x(t)$ di banda $[-W,W]$, campionato con un passo $T_c=\displaystyle\frac{1}{2W}$. Si considerano tre filtri, di risposta impulsiva:
\begin{gather*}
    h_1(t)=\sinc\left(\displaystyle\frac{t}{T}\right)\to H_1(f)=T\sinc(Tf)\\
    h_2(t)=\tri \left(\displaystyle\frac{t}{T}\right)\\
    h_3(t)=\tri \left(\displaystyle\frac{t}{2T}\right)
\end{gather*}
Il segnale campionato si ottiene come:
\begin{gather*}
    x_c(t)=\displaystyle\sum_{n=-\infty}^{+\infty}x(nT)\delta(t-nT)\\
    y_1(t)=x_c(t)*h_1(t)=\left[\displaystyle\sum_{n=-\infty}^{+\infty}x(nT)\delta(t-nT)\right]*\sinc\left(\frac{t}{T}\right)
\end{gather*}
Questo caso corrisponde al teorema del campionamento, poiché il passo di campionamento corrisponde esattamente al reciproco del doppio della frequenza massima del segnale:
\begin{gather}
    y_1(t)=\displaystyle\sum_{n=-\infty}^{+\infty}x(nT)\sinc\left(\frac{t-nT}{T}\right)=x(t)
\end{gather}
Per cui il segnale ricostruito corrisponde esattamente al segnale originario. 

Considerando il secondo filtro:
\begin{gather}
    y_2(t)=x_c(t)*h_2(t)=\displaystyle\sum_{n=-\infty}^{+\infty}x(nT)\tri \left(\frac{t-nT}{T}\right)
\end{gather}
Questo filtro ricostruisce il segnale, ma presenta degli errori alle alte frequenze. 

Considerando il terzo filtro:
\begin{gather}
    y_3(t)=x_c(t)*h_3(t)=\displaystyle\sum_{n=-\infty}^{+\infty}x(nT)\tri \left(\frac{t-nT}{2T}\right)
\end{gather}
Questo filtro non ricostruisce a pieno il segnale. 

\subsubsection*{Esercizio 54}

Dato il segnale $x(t)$:
\begin{gather*}
    x(t)=\tri \left(\displaystyle\frac{t}{T}\right)-\tri \left(\frac{t+T}{T}\right)\\
    x(t)=\begin{cases}
        \displaystyle\frac{\strut t}{\strut T} -2T\leq t<-T\\
        1+\displaystyle\frac{\strut 3t}{\strut T} -T\leq t<0\\
        1-\displaystyle\frac{\strut t}{\strut T} 0\leq t<2T
    \end{cases}
\end{gather*}
Viene campionato con intervallo di campionamento $T_c=T$. Il segnale campionato viene ricostruito con un filtro passa basso ideale di ampiezza $1$ di banda $[-1/2T_c,1/2T_c]$. 
Calcolare il segnale ricostruito nel tempo continuo $y(t)$ e la sua trasformata $Y(f)$. 

Sicuramente lo spettro del segnale originario non è limitato in banda, per cui la sua ricostruzione non potrà coincidere con il segnale originale. Il segnale è invece limitato 
nel tempo, per cui il suo campionamento produce una sequenza limitata, ovvero non nulla solo per un numero limitato di campioni:
\begin{gather*}
    x_c(t)=\displaystyle\sum_{n=-\infty}^{+\infty}x(nT)\delta(t-nT)=x(-T)\delta(t+T)+x(0)\delta(t)=\delta(t)-\delta(t+T)
\end{gather*}
Il segnale ricostruito è quindi:
\begin{gather*}
    y(t)=x_c(t)*h(t)=\left[\delta(t)-\delta(t+T)\right]*\frac{1}{T}\sinc\left(\frac{t}{T}\right)=\frac{1}{T}\sinc\left(\frac{t}{T}\right)-\frac{1}{T}\sinc\left(\frac{t+T}{T}\right)\\
    Y(f)=\rect(Tf)-\rect(Tf)e^{2i\pi fT}\tageq
\end{gather*}

\subsubsection*{Esercizio 55}

Dato il segnale $x(t)$, determinare il segnale campionato $x_c$ e la sua trasformata di Fourier $X_c$:
\begin{gather*}
    x(t)=\sinc\left(\displaystyle\frac{t}{2T}\right)+\cos\left(\frac{2\pi t}{T}\right)
\end{gather*}
Il segnale viene campionato con un passo $\overline{T}$ pari a quattro volte il periodo di Nyquist $T_c$:
\begin{gather*}
    \overline{T}=4T_c
\end{gather*}
Si considera la trasformata di Fourier del segnale $x$ per determinare la sua frequenza massima:
\begin{gather*}
    X(f)=2T\,\rect(2Tf)+\displaystyle\frac{1}{2}\delta\left(f-\frac{1}{T}\right)+\frac{1}{2}\delta\left(f+\frac{1}{T}\right)
\end{gather*}
La banda del segnale è limitata nell'intervallo $[-1/T, 1/T]$, per cui la frequenza di Nyquist corrisponde a $2/T$:
\begin{gather*}
    T_c=\displaystyle\frac{T}{2}\\
    \overline{T}=2T\\
    x_c(t)=\displaystyle\sum_{n=-\infty}^{+\infty}x(n\overline{T})\delta(t-n\overline{T})=\sum_{n=-\infty}^{+\infty}\left[\sinc\left(\frac{2nT}{2T}\right)+\cos\left(\frac{4\pi nT}{T}\right)\right]\delta(t-2nT)\\
    x_c(t)=\displaystyle\sum_{n=-\infty}^{+\infty}\left[\sinc(n)+\cancelto{1}{\cos(4\pi n)}\right]\delta(t-2nT)\\
    x(t)=\displaystyle\delta(t)+\sum_{n=-\infty}^{+\infty}\delta(t-2nT)\tageq
\end{gather*}
Si svolge ora la medesima analisi in frequenza:
\begin{gather*}
    X_c(f)=\displaystyle\frac{1}{2T}\sum_{n=-\infty}^{+\infty}X\left(f-\frac{n}{2T}\right)\\
    \displaystyle\frac{1}{2T}\sum_{n=-\infty}^{+\infty}2T\,\rect\left[2T\left(f-\frac{n}{2T}\right)\right]+\frac{1}{4T}\sum_{n=-\infty}^{+\infty}\delta\left(f-\frac{1}{T}-\frac{n}{T}\right)+\frac{1}{4T}\sum_{n=-\infty}^{+\infty}\delta\left(f+\frac{1}{T}-\frac{n}{T}\right)\\
    X_c(t)1+\displaystyle\frac{1}{2T}\sum_{n=-\infty}^{+\infty}\delta\left(f-\frac{n}{2T}\right)\tageq
\end{gather*}
Si ottiene lo stesso risultato della trasformata del segnale campionato nel tempo. 

\subsubsection*{Esercizio 56}

Dato il segnale $x(t)$:
\begin{equation*}
    x(t)=2\sinc(4t)\cos(6\pi t)
\end{equation*}
Calcolare la frequenza minima di campionamento per cui non sia presente aliasing. In seguito campionato con il tempo di campionamento massimo $T_c$, determinare 
l'espressione analitica di $x'(t)$ e $X'(f)$ del segnale ricostruito da un filtro $H(f)$ passa basso ideale definito sull'intervallo $[-3,3]$. 
Si determina la frequenza massima del segnale:
\begin{gather*}
    X(f)=\displaystyle\frac{1}{2}\rect\left(\frac{f}{4}\right)\left[\frac{1}{2}\delta(f-3)+\frac{1}{3}\delta(f+3)\right]\\
    X(f)=\rect\left(\displaystyle\frac{f-3}{4}\right)+\rect\left(\frac{f+3}{4}\right)
    f_{\max}=5\\
    f_c=2f_{\max}=10\to T_c=\displaystyle\frac{1}{10}\tageq
\end{gather*}
Si campiona con questo tempo $T_c$:
\begin{gather*}
    x_c(t)=\displaystyle\sum_{n=-\infty}^{+\infty}x\left(\frac{n}{10}\right)\delta\left(t-\frac{n}{10}\right)\\
    X_c(f)=10\displaystyle\sum_{n=-\infty}^{+\infty}X(f-10n)
\end{gather*}
Si filtra tramite $H(f)$:
\begin{gather*}
    H(f)=\displaystyle\rect\left(\frac{f}{6}\right)\\
    X'(f)=\left[10\displaystyle\sum_{n=-\infty}^{+\infty}X(f-10n)\right]\cdot\rect\left(\frac{f}{6}\right)\\
    \left[10\displaystyle\sum_{n=-\infty}^{+\infty}\rect\left(\displaystyle\frac{f-3-10n}{4}\right)+\rect\left(\frac{f+3-10n}{4}\right)\right]\cdot\rect\left(\frac{f}{6}\right)\\
    X'(f)=10\left[\displaystyle\rect\left(\frac{f-2}{2}\right)+\rect\left(\frac{f+2}{2}\right)\right]\tageq
\end{gather*}
Antitrasformando si ottiene il segnale ricostruito nel tempo:
\begin{equation}
    x'(t)=40\sinc(2t)\cos(4\pi t)
\end{equation}

\subsubsection*{Esercizio 57}

Il segnale $x(t)$ viene campionato con il massimo passo di campionamento tale da non provocare aliasing, producendo solamente due campioni diversi da zero:
\begin{gather*}
    x[n]=
    \begin{cases}
        1 &n=\pm1\\
        0 &n\neq\pm1
    \end{cases}
\end{gather*}
Viene poi ricostruito da un filtro passa basso ideale sull'intervallo $[-1/2T,1/2T]$ e di ampiezza $T$:
\begin{gather*}
    H(f)=\rect\displaystyle\left(Tf\right)
\end{gather*} 
Determinare il segnale ricostruito $x'(t)$ e la sua trasformata $X'(f)$. 
Si determina il segnale campionato:
\begin{gather*}
    x_c(t)=\displaystyle\sum_{n=-\infty}^{+\infty}x[n]\delta(t-nT)=x[1]\delta(t-T)+x[-1]\delta(t+T)\\
    X_c(f)=2\cos(2T\pi f)
\end{gather*}
Passando questo segnale attraverso il filtro si ottiene:
\begin{gather}
    X'(f)=X_c(f)\cdot H(f)=2\cos(2T\pi f)T\rect(Tf)
\end{gather}
Antitrasformando si ottiene il segnale ricostruito nel tempo:
\begin{equation}
    x'(t)=\displaystyle\sinc\left(\frac{t-T}{T}\right)+\sinc\left(\frac{t+T}{T}\right)
\end{equation}

\clearpage

\section{Fenomeni Aleatori}

\subsubsection*{Esercizio 58}

Si consideri una sorgente $S$ ed un destinatario $D$ che possono trasmettere e ricevere solamente $1$ o $0$. La probabilità che si ricevi uno $0$, 
quando viene trasmesso uno $0$ è dell'$80\%$, mentre la probabilità che si ricevi un $1$ quando viene trasmesso uno $0$ è del $20\%$:
\begin{gather*}
    \Pr(D=0|S=0)=0.8\\
    \Pr(D=1|S=1)=0.2
\end{gather*}
Mentre la probabilità di errore nella trasmissione di un $1$ è:
\begin{gather*}
    \Pr(D=1|S=1)=0.7\\
    \Pr(D=0|S=1)=0.3
\end{gather*}
Per cui si conosco le probabilità di errore del canale di trasmissione.
I simboli $0$ e $1$ non sono equiprobabili alla sorgente:
\begin{gather*}
    \Pr(S=0)=0.6\\
    \Pr(S=1)=0.4
\end{gather*}

Si vuole calcolare la probabilità sia stato trasmesso un $1$ dalla sorgente quando viene misurato un $1$:
\begin{gather*}
    \Pr(S=1|D=1)
\end{gather*}
Si esprime tramite il teorema di Bayes:
\begin{gather*}
    \Pr(S=1|D=1)=\displaystyle\frac{\Pr(D=1|S=1)\Pr(S=1)}{\Pr(D=1)}=\frac{0.7\cdot0.4}{\Pr(D=1)}
\end{gather*}

Per determinare la probabilità che il ricevitore misuri $1$ si determina tramite il teorema delle proprietà totali:
\begin{gather*}
    \Pr(D=1)=\Pr(D=1|S=1)\Pr(S=1)+\Pr(D=1|S=0)\Pr(S=0)=0.7\cdot0.4+0.3\cdot0.6
\end{gather*}
La probabilità che si misuri un $1$ quando viene effettivamente trasmesso un $1$ dalla sorgente risulta essere:
\begin{gather}
    \Pr(S=1|D=1)=\displaystyle\frac{0.7\cdot0.4}{0.7\cdot0.4+0.2\cdot0.6}=0.7
\end{gather}

\subsubsection*{Esercizio 59}

Sia data la probabilità di una variabile aleatoria $X$:
\begin{gather*}
    P_X(x)=\displaystyle\frac{e^{-\left(\frac{x}{\sqrt{2}{\sigma_x}}\right)^2}}{\sqrt{2\pi}\sigma_x}
\end{gather*}

Nota la sua relazione con un'altra variabile aleatoria $Y$:
\begin{gather*}
    y=ax+b
\end{gather*}
Si calcola la densità di probabilità della variabile aleatoria $Y$:
\begin{gather}
    P_Y(y)=\displaystyle\frac{e^{-\left(\frac{y-b}{a}\right)^2\frac{1}{2\sigma_x^2}}}{\sqrt{2\pi}\sigma_x}\left|\frac{1}{a}\right|
\end{gather}
Per cui se $X$ ha una statistica gaussiana, anche $Y$ ha una statistica gaussiana, ma non centrata nell'origine. 

\subsubsection*{Esercizio 60}

La densità di probabilità della variabile aleatoria $X$ è uniformemente distribuita sull'intervallo $[2,3]$. Sia data una variabile aleatoria $Y$ legata da una relazione 
lineare:
\begin{gather*}
    Y=aX+b
\end{gather*}
Determinare la distribuzione di probabilità della variabile aleatoria $Y$:
\begin{gather*}
    P_X(x)=\rect\left(x-\displaystyle\frac{5}{2}\right)\\
    P_Y(y)=P_X(g(y))|g'(y)|=\displaystyle\frac{1}{|a|}\rect\left(\frac{2y-2b-5a}{2a}\right)\tageq
\end{gather*}

\subsubsection*{Esercizio 61}

Data la densità di probabilità gaussiana di una variabile aleatoria $X$, calcolare la densità di probabilità di una variabile $Y$ legata dalla relazione:
\begin{gather*}
    y=x^3\\
    P_X(x)=\displaystyle\frac{e^{-\left(\frac{x}{\sqrt{2}{\sigma_x}}\right)^2}}{\sqrt{2\pi}\sigma_x}\\
    P_Y(y)=P_X(g(y))|g'(y)|=\displaystyle\frac{e^{-\left(\frac{\sqrt[3]{y}}{\sqrt{2}{\sigma_x}}\right)^2}}{\sqrt{2\pi}\sigma_x}\frac{1}{3\sqrt[3]{y^2}}
\end{gather*}

\subsubsection*{Esercizio 62}

Sia data una variabile aleatoria $X$ uniformemente distribuita sull'intervallo $[0,\pi]$ ed una variabile $Y$ dipendente dalla prima, tramite la seguente relazione:
\begin{gather*}
    y=\cos(x)\\
    P_X(x)=\displaystyle\frac{1}{\pi}\rect\left(\frac{x-\pi/2}{\pi}\right)
\end{gather*}
Determinare la densità di probabilità della variabile $Y$ $P_Y(y)$. Noto l'andamento della funzione coseno, la densità di probabilità di $y$ è una funzione 
compresa nell'intervallo $[-1,+1]$. Si applica quindi la formula su questo intervallo:
\begin{gather*}
    P_Y(y)=P_X(\arccos(y))\displaystyle\frac{\df}{\df y}\arccos(y)\\
    P_Y(y)=\displaystyle\frac{1}{\pi}\rect\left(\frac{\arccos(y)-\pi/2}{\pi}\right)\frac{1}{\sqrt{1-y^2}}\\
    P_Y(y)=\displaystyle\rect\left(\frac{y}{2}\right)\frac{1}{\sqrt{1-y^2}}\tageq
\end{gather*}

\subsubsection*{Esercizio 63}

Date due variabili aleatori legate dalla relazione $y=f(x)=\cos(x)$, con $X$ uniformemente distribuita sull'intervallo $[0,2\pi]$, calcolare la densità di probabilità 
della variabile $y$:
\begin{gather*}
    P_X(x)=\displaystyle\frac{1}{2\pi}\rect\left(\frac{x-\pi}{2\pi}\right)
\end{gather*}
Poiché il coseno non è invertibile sull'intero intervallo dove è definita la variabile $X$, si divide in $[0,\pi]$ e $[\pi,2\pi]$, e si definiscono due funzioni 
aggiuntive che legano le variabili aleatorie nei due intervalli:
\begin{gather*}
    f_1(x)=\cos(x)\\
    g_1(y)=\arccos(y)\\
    g_2(y)=-g_1(y)
\end{gather*}
La densità di probabilità sull'intervallo $[0,\pi]$, è già stata calcolata nell'esercizio precedente, in questo caso la variabile $X$ è distribuita uniformemente su un 
intervallo di ampiezza doppia, per cui si divide in un due funzioni centrate nei due intervalli:
\begin{gather*}
    P_X(x)=\displaystyle\frac{1}{2\pi}\left[\rect\left(\frac{x-\pi/2}{\pi}\right)+\rect\left(\frac{x-3\pi/2}{\pi}\right)\right]\\
    P_{Y,1}(y)=\displaystyle\frac{1}{2\pi}\rect\left(\frac{\arccos(y)-\pi/2}{\pi}\right)\frac{1}{\sqrt{1-y^2}}\\
    P_{Y,2}(y)=\displaystyle\frac{1}{2\pi}\rect\left(\frac{-\arccos(y)-3\pi/2}{\pi}\right)\frac{1}{\sqrt{1-y^2}}\\
    P_{Y,1}(y)+P_{Y,2}(y)=\displaystyle\frac{1}{2\pi}\frac{1}{\sqrt{1-y^2}}\left[\rect\left(\frac{y}{2}\right)+\rect\left(\frac{y}{2}\right)\right]=\frac{1}{\pi}\rect\left(\displaystyle\frac{y}{2}\right)\frac{1}{\sqrt{1-y^2}}=P_Y(y)\tageq
\end{gather*}
Si vuole calcolare il valore atteso, può essere calcolato sul dominio $y$ o $x$:
\begin{gather*}
    E[\cos (x)]=\displaystyle\int_{-\infty}^{+\infty}\cos (x)P_X(x)\df x =\int_0^{2\pi}\frac{1}{2\pi}\cos(x)\df x=0\\
    E[\cos(x)]=E[y]=\displaystyle\int_{-\infty}^{+\infty}yP_Y(y)\df y=\int_{-\infty}^{+\infty}y\,\rect\left(\frac{y}{2}\right)\frac{1}{\pi}\frac{1}{\sqrt{1-y^2}}\df y\\
    \displaystyle\int_{-1}^1\frac{y}{\sqrt{1-y^2}}\df y=0\\
    \mu_y=0\tageq
\end{gather*}

\subsubsection*{Esercizio 64}

Calcolare la densità di probabilità della variabile aleatoria $Y$, nota la densità di una variabile indipendente $X$ e la loro relazione:
\begin{gather*}
    y=f(x)=\begin{cases}
        x+2&-4\leq x<-1\\
        -x&-1\leq x<+1\\
        x-2&x\geq +1
    \end{cases}\\
    P_X(x)=\displaystyle\frac{1}{8}\rect\left(\frac{x}{8}\right)
\end{gather*}
Si studiano i tre casi separatamente, per cui si scompone la densità di probabilità della variabile $X$:
\begin{gather*}
    P_{X,1}(x)=\displaystyle\frac{1}{8}\rect\left(\frac{x+5/2}{3}\right)\\
    P_{X,2}(x)=\displaystyle\frac{1}{8}\rect\left(\frac{x}{2}\right)\\
    P_{X,3}(x)=\displaystyle\frac{1}{8}\rect\left(\frac{x-5/2}{3}\right)
\end{gather*}
Si ottengono quindi le seguenti densità di probabilità per la variabile $Y$:
\begin{gather*}
    P_{Y,1}(y)=\displaystyle\frac{1}{8}\rect\left(\frac{y-2+5/2}{3}\right)\\
    P_{Y,2}(y)=\displaystyle\frac{1}{8}\rect\left(\frac{-y}{3}\right)\\
    P_{Y,3}(y)=\displaystyle\frac{1}{8}\rect\left(\frac{y+2+5/2}{3}\right)
\end{gather*}
Per cui la densità della variabile $Y$ si ottiene dalla somma di quest'ultime:
\begin{equation}
    P_Y(y)=\displaystyle\frac{1}{8}\left(\rect\left(\frac{y-1/2}{3}\right)+\rect\left(\frac{y}{2}\right)+\rect\left(\frac{y+1/2}{3}\right)\right)
\end{equation}

\subsubsection*{Esercizio 65}

Sia data la densità di probabilità $P_X(x)$ e la relazione con la variabile aleatoria $Y$, determinare la densità di probabilità di $Y$:
\begin{gather*}
    P_X(x)=\displaystyle\frac{1}{2}\rect\left(\frac{x}{2}\right)\\
    y=f(x)=\begin{cases}
        -x&x<0\\
        x^2&x\geq 0
    \end{cases}
\end{gather*}
Si divide l'intervallo in $[-1/2,0]$ e $[0,1/2]$:
\begin{gather*}
    P_X(x)=\displaystyle\frac{1}{2}\rect\left(x+\frac{1}{2}\right)+\frac{1}{2}\rect\left(x-\frac{1}{2}\right)\\
    P_{Y,1}(y)=\displaystyle\frac{1}{2}\rect\left(-y+\frac{1}{2}\right)\\
    P_{Y,2}(y)=\displaystyle\frac{1}{2}\rect\left(\sqrt{y}-\frac{1}{2}\right)\frac{1}{2\sqrt{y}}\\
    P_Y(y)=\displaystyle\frac{1}{2}\rect\left(y+\frac{1}{2}\right)+\frac{1}{4}\rect\left(y-\frac{1}{2}\right)\frac{1}{\sqrt{y}}\tageq
\end{gather*}
Si considera la seguente densità di probabilità per la variabile $X$:
\begin{gather*}
    P_X(x)=\displaystyle\frac{1}{\sqrt{2\pi}\sigma_x}e^{-\left(\frac{x}{\sqrt2\sigma_x}\right)^2}\\
    P_Y(y)=\displaystyle\frac{1}{\sqrt{2\pi}\sigma_x}e^{-\left(\frac{y}{\sqrt2\sigma_x}\right)^2}u(y)+\frac{1}{2\sqrt{2\pi y} \sigma_x}e^{-\frac{y}{2\sigma_x^2}}u(y)\\
    P_Y(y)=\displaystyle\frac{1}{\sqrt{2\pi}\sigma_x}\left(e^{-\left(\frac{y}{\sqrt2\sigma_x}\right)^2}+\frac{1}{2\sqrt{y}}e^{-\frac{y}{2\sigma_x^2}}\right)u(y)\tageq
\end{gather*}

\subsubsection*{Esercizio 66}

Sia data una variabile aleatoria discreta $X$, la sua densità di probabilità e la relazione con la variabile aleatoria $Y$. Determinare la densità di probabilità di $Y$:
\begin{gather*}
    P_X(x)=\displaystyle\frac{1}{4}\delta(x+1)+\frac{1}{2}\delta(x)+\frac{1}{4}\delta(x-1)\\
    y=x^2\\
    P_Y(y)=\displaystyle\frac{1}{4}\delta(y-1)+\frac{1}{2}\delta(y)+\frac{1}{4}\delta(y-1)=\frac{1}{2}\left[\delta(y)+\delta(y-1)\right]\tageq
\end{gather*}

\subsubsection*{Esercizio 67}

Sia data la variabile aleatoria $X$ di statistica gaussiana centrata in $x=1$, e la relazione con la variabile aleatoria $Y$, tramite un rettificatore:
\begin{gather*}
    P_X(x)=\displaystyle\frac{1}{\sqrt{2\pi}\sigma_x}e^{-\left(\frac{x-1}{\sigma_x}\right)^2}\\
    y=\begin{cases}
        -1&x< 1\\
        2&x\geq 1
    \end{cases}
\end{gather*}
Poiché la statistica di $X$ è centrata in $1$, la probabilità che sia maggiore o minore di $1$ è pari a $1/2$, per cui la densità di probabilità di 
$Y$ corrisponde a due impulsi:
\begin{gather*}
    \displaystyle\int_{-\infty}^1P_X(x<1)\df x=\int_{1}^{+\infty}P(x>1)\df x=\frac{1}{2}\\
    P_Y(y)=\displaystyle\frac{1}{2}\delta(y-2)+\frac{1}{2}\delta(y+1)\tageq
\end{gather*}

\subsubsection*{Esercizio 68}

Sia data la distribuzione di probabilità della variabile $X$, e la funzione che la associa alla variabile aleatoria $Y$. Determinare la densità della probabilità $Y$:
\begin{gather*}
    P_X(x)=\displaystyle\frac{1}{2\pi}\rect\left(\frac{x-\pi}{2\pi}\right)
    y=\begin{cases}
        \pi &x<\pi\\
        x &x\geq\pi
    \end{cases}
\end{gather*}
La variabile aleatoria $Y$ è una variabile mista:
\begin{gather*}
    P_Y(y)=\displaystyle\int_0^{\pi}P_X(x\leq\pi)\df x\delta(y-\pi)+\int_{\pi}^{2\pi}P_X(x\geq\pi)\df x \frac{1}{2\pi}\rect\left(\frac{y-\pi-\pi/2}{\pi}\right)\\
    P_Y(y)=\displaystyle\frac{1}{2}\delta(y-\pi)+\frac{1}{4\pi}\rect\left(\frac{y-3\pi/2}{\pi}\right)\tageq
\end{gather*}

\subsubsection*{Esercizio 69}

Sia data la variabile aleatoria $X$ e la sua relazione con la variabile aleatoria $Y$, determinare la densità di probabilità di quest'ultima:
\begin{gather*}
    P_X(x)=\frac{1}{4a}\rect\left(\frac{x}{4a}\right)
    y=\begin{cases}
        -a&x<-a\\
        x&-a\geq x<a\\
        a&x\geq a
    \end{cases}\\
    P_Y(y)=\displaystyle\int_{-2a}^{-a}\frac{1}{4a}\df x\,\delta(y+a)+\int_{-a}^a\frac{1}{4a}\df x\frac{1}{2a}\,\rect\left(\frac{y}{2a}\right)+\int_{a}^{2a}\frac{1}{4a}\df x\,\delta(y-a)\\
    P_Y(y)=\displaystyle\frac{1}{4}\left[\delta(y+a)+\delta(y-a)+\frac{1}{a}\rect\left(\frac{y}{2a}\right)\right]\tageq
\end{gather*}

\subsubsection*{Esercizio 70}

Determinare il valore medio e deviazione standard della seguente densità di probabilità:
\begin{gather*}
    P_x(x)=\displaystyle\frac{1}{2}\rect\left({t+\frac{3}{2}}\right)+\frac{1}{2}\left({t-\frac{3}{2}}\right)\\
    \mu_x=\displaystyle\int_{-\infty}^{+\infty}x\left[\frac{1}{2}\rect\left({t+\frac{3}{2}}\right)+\frac{1}{2}\left({t-\frac{3}{2}}\right)\right]\df x\\
    \displaystyle\frac{1}{2}\int_{-2}^{-1}x\df x+\frac{1}{2}\int_{1}^2x\df x=-\frac{1}{2}\int_1^2x\df x+\frac{1}{2}\int_1^2x\df x=0\\
    \mu_x=0\tageq\\
    \sigma_x^2=\displaystyle\int_{-\infty}^{+\infty}x^2\left[\frac{1}{2}\rect\left({t+\frac{3}{2}}\right)+\frac{1}{2}\left({t-\frac{3}{2}}\right)\right]\df x\\
    \displaystyle\frac{1}{2}\int_{-2}^{-1}x^2\df x+\frac{1}{2}\int_1^2x^2\df x=\int_1^2x^2\df x\\
    \displaystyle\left[\frac{x^3}{3}\right]_1^2=\frac{8}{3}-\frac{1}{3}=\frac{7}{3}\\
    \sigma_x^2=\displaystyle\frac{7}{3}\tageq
\end{gather*}

\subsubsection*{Esercizio 71}

Date due variabili aleatorie indipendenti $X$ e $Y$ di statistica gaussiana, determinare la densità di probabilità della variabile dipendente $Z=X+Y$. 
Si può risolvere sia nel dominio di $z$ oppure in frequenza tramite la funzione caratteristica:
\begin{gather*}
    P_X(x)=\displaystyle\frac{1}{\sqrt{2\pi}\sigma_x}e^{-\frac{(x-\mu_x)^2}{2\sigma_x^2}}\\
    P_Y(y)=\displaystyle\frac{1}{\sqrt{2\pi}\sigma_y}e^{-\frac{(y-\mu_y)^2}{2\sigma_y^2}}\\
    P_Z(z)=P_X(x)*P_Y(y)=\displaystyle\frac{1}{\sqrt{2\pi(\sigma_x^2+\sigma_y^2)}}e^{-\frac{z-\mu_x-\mu_y}{2(\sigma_x^2+\sigma_y^2)}}\\
    P_Z(z)=\displaystyle\frac{1}{\sqrt{2\pi}\sigma_y}e^{-\frac{z-\mu_z}{2\sigma_z^2}}\tageq
\end{gather*} 

\subsubsection*{Esercizio 72}

Data la variabile aleatoria $Z$ data dalla combinazione lineare di due variabili aleatorie uniformemente distribuite sull'intervallo $[-1/2,1/2]$:
\begin{gather*}
    z=ax+by+c
\end{gather*}
Per ritornare ad un caso noto si applicano le seguenti sostituzioni:
\begin{gather*}
    x'=ax\\
    y'=by+c\\
    z=x'+y'
\end{gather*}
Si calcolano le densità di probabilità di queste nuove variabili aleatorie:
\begin{gather*}
    P_{X'}(x')=\rect\displaystyle\left(\frac{x'}{a}\right)\left|\frac{1}{a}\right|\\
    P_{Y'}(y')=\rect\displaystyle\left(\frac{y'-c}{b}\right)\left|\frac{1}{b}\right|\\
    P_Z(z)=P_{X'}(x')*P_{Y'}(y')=\displaystyle\left|\frac{1}{ab}\right|\rect\left(\frac{x'}{a}\right)*\rect\left(\frac{y'-c}{b}\right)\\
    P_Z(z)=\begin{cases}
        0 &z<c-\displaystyle\frac{\strut a+b}{\strut 2} \land z\geq c-\frac{\strut a+b}{\strut 2}\\
        \displaystyle \left|\frac{\strut 1}{\strut ab}\right|\frac{\strut 1}{\strut a}\left(\frac{\strut a+b}{\strut 2}-|z-c|\right)&\displaystyle c-\frac{\strut a+b}{2}\leq z< c-\frac{\strut b-a}{\strut 2}\land c+\frac{\strut b-a}{\strut 2}\leq z<c+\frac{\strut a+b}{\strut 2}\\
        \displaystyle\left|\frac{\strut 1}{\strut ab}\right|&c-\displaystyle\frac{\strut b-a}{\strut 2}\leq z<c+\frac{\strut b-a}{\strut 2}
    \end{cases}\tageq
\end{gather*}

\subsubsection*{Esercizio 73}

Data la variabile aleatoria dipendente $Y$, combinazione lineare di due variabili aleatorie $X_1$ e $X_2$. Determinare date le densità di probabilità 
di queste ultime la densità di probabilità di $Y$:
\begin{gather*}
    y=7x_1+x_2-3\\
    \Pr(x_1=-2)=0.6\\
    \Pr(x_1=2)=0.4\\
    \Pr(x_2=-1)=0.5\\
    \Pr(X_2=1)=0.5
\end{gather*}
Poiché sono possibili solo $4$ realizzazioni della variabile $Y$, è possibile creare una tabella caratterizzata dai quattro possibili valori:
\begin{center}
    \begin{tabular}{c|c|c|}
        &$x_1=-2$,$P=0.6$&$x_1=2$,$P=0.4$\\
        \hline\\
        $x_2=-1$,$P=0.5$&$-18$,$P=0.3$&$10$,$P=0.2$\\
        \hline\\
        $x_2=1$,$P=0.5$ &$-16$,$P=0.3$ &$12$,$P=0.2$\\
        \hline\\
    \end{tabular}
\end{center}
Si esprime quindi la densità di probabilità della variabile discreta $Y$:
\begin{equation}
    P_Y(y)=\displaystyle\frac{3}{10}\delta(y+18)+\frac{1}{5}\delta(y-10)+\frac{3}{10}\delta(y+16)+\frac{1}{5}\delta(y-12)
\end{equation}

Si può ottenere lo stesso risultato lavorando con le densità discrete delle due variabili $X_1$ e $X_2$:
\begin{gather*}
    P_{X_1}(x_1)=\displaystyle\frac{3}{5}\delta(x_1+2)+\frac{2}{5}\delta(x_1-2)\\
    P_{X_2}(x_2)=\displaystyle\frac{1}{2}\delta(x_2+1)+\frac{1}{2}\delta(x_2-1)
\end{gather*}
Si sostituiscono due variabili di appoggio:
\begin{gather*}
    x_1'=7x_1\\
    x_2'=x_2-3\\
    P_{X_1'}(x_1')=\displaystyle\frac{1}{7}\left[\frac{3}{5}\delta(x_1'/7+2)+\frac{2}{5}\delta(x_1'/7-2)\right]=\frac{3}{5}\delta(x_1'+14)+\frac{2}{5}\delta(x_1'-14)\\
    P_{X_2'}(x_2')=\displaystyle\frac{1}{2}\delta(x_2'+4)+\frac{1}{2}\delta(x_2'+2)\\
    P_Y(y)=P_{X_1'}(x_1')*P_{X_2'}(x_2')=\left[\displaystyle\frac{3}{5}\delta(x_1'+14)+\frac{2}{5}\delta(x_1'-14)\right]*\left[\frac{1}{2}\delta(x_2'+4)+\frac{1}{2}\delta(x_2'+2)\right]\\
    P_Y(y)=\displaystyle\frac{3}{10}\delta(y+18)+\frac{1}{5}\delta(y-10)+\frac{3}{10}\delta(y+16)+\frac{1}{5}\delta(y-12)
\end{gather*}

\subsubsection*{Esercizio 74}

Dato il processo aleatorio $x(t)$ stazionario, avente densità di probabilità uniforme sull'intervallo $[-2,10]$, ed una covarianza:
\begin{gather*}
    C_x(\tau)=A\mathrm{tri}\displaystyle\left(\frac{\tau}{2}\right)
\end{gather*}
Calcolare il valor medio e $A$:
\begin{gather*}
    \mu_x=\displaystyle\int_{-\infty}^{+\infty}xP_X(x)\df x=\int_{-2}^{10}\frac{x}{12}\df x=4\\
    C_x(\tau)=E[(x(t+\tau)-\mu_x)(x(t)-\mu_x)]\\
    C_x(0)=E[(x(t)-\mu_x)^2]=\sigma_x^2\\
    \mu_x^{(2)}=\displaystyle\int_{-\infty}^{+\infty}xP_X(x)\df x=\int_{-2}^{10}\frac{x^2}{12}\df x=28\\
    \sigma_x^2=\mu_x^{(2)}-\mu_x^2=28-16=12\\
    C_x(0)=A=12\tageq
\end{gather*}
Calcolare la funzione di autocorrelazione di $X$:
\begin{gather}
    R_x(\tau)=C_x(\tau)+\mu_x^2=12\mathrm{tri}\displaystyle\left(\frac{\tau}{2}\right)+16
\end{gather}
Sia: 
\begin{gather*}
    y(t)=x(t)+x(t-1)
\end{gather*}
Determinare il valore medio e la funzione di autocorrelazione di $y(t)$:
\begin{gather*}
    \mu_y=E[y(t)]=E[x(t)]+E[x(t-1)]=2\mu_x=8\tageq\\
    R_y(\tau)=E[t(t+\tau)y(t)]=E[(x(t+\tau)+x(t+\tau-1))(x(t)+x(t-1))]\\
    E[x(t+\tau)x(t)]+E[x(t+\tau)x(t-1)]+E[x(t+\tau-1)x(t)]+E[x(t+\tau-1)x(t-1)]\\
    R_x(\tau)+R_X(\tau-1)+R_x(\tau+1)+R_x(\tau)=2R_X(\tau)+2R_x(\tau+1)\\
    R_y(\tau)=24\mathrm{tri}\displaystyle\left(\frac{\tau}{2}\right)+32+24\mathrm{tri}\left(\frac{\tau+1}{2}\right)+32\\
    R_y(\tau)=24\mathrm{tri}\displaystyle\left(\frac{\tau}{2}\right)+24\mathrm{tri}\left(\frac{\tau+1}{2}\right)+64\tageq
\end{gather*}
Determinare la potenza di $Y$:
\begin{equation}
    P_y=R_y(0)=24+24\displaystyle\frac{1}{2}+64=100    
\end{equation}

\subsubsection*{Esercizio 75}

Date tre variabili aleatorie indipendenti tra di loro $A$, $B$ e $\theta$, determinare il valore medio e la densità spettrale di potenza del processo $x(t)$:
\begin{gather*}
    x(t)=(A+2B)\cos(300\pi t+\theta)+n(t)
\end{gather*}
Le tre variabili aleatorie hanno distribuzione uniforme rispettivamente in $[-1,+1]$, $[-2,+2]$ e $[0,4\pi]$:
\begin{gather*}
    P_A(a)=\displaystyle\frac{1}{2}\rect\left(\frac{a}{2}\right)\\
    P_B(b)=\displaystyle\frac{1}{4}\rect\left(\frac{b}{4}\right)\\
    P_\theta(\theta)=\displaystyle\frac{1}{4\pi}\rect\left(\frac{\theta-2\pi}{4\pi}\right)
\end{gather*}
Il rumore è un rumore bianco a valor medio nullo, è anch'esso indipendente dalle variabili aleatorie $A$, $B$ e $\theta$, ed ha una funzione di autocorrelazione:
\begin{gather*}
    R_n(\tau)=10\delta(\tau)
\end{gather*}
Il rumore ha quindi potenza:
\begin{gather*}
    P_n=R_n(0)=10
\end{gather*}

Si calcola il valore attesto tramite la definizione:
\begin{gather*}
    \mu_x=E[x(t)]=E[(A+2B)\cos(300\pi t+\theta)+n(t)]
\end{gather*}
Poiché sono variabili indipendenti, si può usufruire della proprietà di linearità della funzione ``expectation'':
\begin{gather*}
    (E[A]+2E[B])E[\cos(300\pi t+\theta)]+\cancelto{0}{E[n(t)]}
\end{gather*}
Poiché le variabili $A$ e $B$ sono distribuite uniformemente su un intervallo centrato nell'origine, i loro valori attesi sono nulli, per cui:
\begin{gather*}
    \mu_x=0\cdot E[\cos(300\pi t+\theta)]=0
\end{gather*}
Per calcolare la densità spettrale di potenza si calcola la funzione di autocorrelazione e si pone $\tau=0$:
\begin{gather*}
    R_x(\tau)=E[x(t+\tau)x^*(t)]\\
    E\left[\left\{(A+2B)\cos[300\pi (t+\tau)+\theta]+n(t+\tau)\right\}\left\{(A+2B)\cos[300\pi t+\theta]+n(t)\right\}\right]\\
    E[(A+2B)^2\cos\{300\pi(t+\tau)+\theta\}\cos(300\pi t+\theta)]+E[(A+2B)\cos\{300\pi (t+\tau)+\theta\}n(t)]\\
    +E[(A+2B)\cos(300\pi t+\theta)n(t+\tau)]+E[n(t+\tau)n(t)]
\end{gather*}
Poiché sono indipendenti tra di loro, si possono calcolare i contributi di ciascuna variabile indipendentemente:
\begin{gather*}
    E[(A+2B)^2]=E[A^2]+4E[B^2]+\cancelto{0}{2E[A]E[B]}=\displaystyle\int_{-1}^1\frac{a^2}{2}\df a+4\int_{-2}^{2}\frac{b^2}{4}\df b=\frac{1}{3}+4\frac{4}{3}=\frac{17}{3}\\
    E[\cos\{300\pi(t+\tau)+\theta\}\cos(300\pi t+\theta)]=\displaystyle\int_{0}^{4\pi}\frac{1}{4\pi}\cos[300\pi(t+\tau)+\theta]\cos(300\pi t+\theta)\df\theta\\
    \displaystyle\frac{1}{8\pi}\cancelto{0}{\int_{0}^{4\pi}\cos[300\pi(2t+\tau)-2\theta]\df\theta}+\frac{1}{8\pi}\int_0^{4\pi}\cos(300\pi \tau)\df\theta=\frac{1}{2}\cos(300\pi\tau)\\
    E[(A+2B)^2\cos[300\pi(t+\tau)+\theta]\cos(300\pi t+\theta)]=\displaystyle\frac{17}{3}\frac{1}{2}\cos(300\pi\tau)=\frac{17}{6}\cos(300\pi\tau)\\
    E[n(t)]=E[n(t+\tau)]=0\\
    E[(A+2B)\cos\{300\pi (t+\tau)+\theta\}n(t)]=E[(A+2B)\cos(300\pi t+\theta)n(t+\tau)]=0\\
    E[n(t+\tau)n(t)]=R_n(\tau)=10\delta(\tau)\\
\end{gather*}
Per cui la funzione di autocorrelazione del processo $x(t)$ risulta essere:
\begin{equation*}
    R_x(\tau)=\displaystyle\frac{17}{6}\cos(300\pi\tau)+10\delta(\tau)
\end{equation*}
Si applica quindi la trasformata di Fourier per ottenere la densità spettrale di potenza:
\begin{equation}
    G_x(f)=\displaystyle\frac{17}{12}\left[\delta(f-150)+\delta(f+150)\right]+10
\end{equation}

\subsubsection*{Esercizio 76}

Dato il segnale stocastico $x(t)$, sia $y(t)$ il segnale definito dalla seguente espressione:
\begin{gather*}
    y(t)=\displaystyle\int_{t-T}^{t+T}x(t')\df t'
\end{gather*}
Calcolare la densità spettrale di potenza di $y(t)$ in funzione di $x(t)$, e la potenza di $y(t)$. 
Il segnale $y(t)$ può essere espresso come un integrale su $[-\infty,+\infty]$, moltiplicato per una finestra:
\begin{gather*}
    y(t)=\displaystyle\int_{-\infty}^{+\infty}x(t')\rect\left(\frac{t'-t}{2T}\right)\df t'=x(t)*\rect\left(\frac{t}{2T}\right)
\end{gather*}
Per cui la finestra rappresenta la risposta impulsiva $h(t)$ di un filtro, per cui la densità spettrale di potenza dell'uscita di un filtro si ottiene come:
\begin{gather}
    G_y(f)=|H(f)|^2G_X(f)=4T^2\sinc^2(2Tf)G_x(f)
\end{gather}
Calcolare la potenza del segnale $y(t)$, nel caso in cui $x(t)$ sia un segnale bianco e densità spettrale di potenza:
\begin{gather*}
    G_x(f)=\displaystyle\frac{N_0}{2}
\end{gather*}
Si calcola la potenza integrando la densità spettrale di $y$, oppure si calcola la funzione di autocorrelazione di $y$ in $\tau=0$:
\begin{gather*}
    R_y(\tau)=\mathscr{F}^{-1}\left\{4T^2\sinc^2(2Tf)\displaystyle\frac{N_0}{2}\right\}=2T\mathrm{tri}\left(\frac{\tau}{2T}\right)\frac{N_0}{2}
\end{gather*}
La potenza è quindi:
\begin{equation}
    R_y(0)=N_0T=P_y
\end{equation}

\subsubsection*{Esercizio 77}

Dato un processo $y(t)$:
\begin{gather*}
    y(t)=4x(t-3)+4x(t+3)
\end{gather*}
Dove $x(t)$ è un processo stazionario a valor medio nullo, e di funzione di autocorrelazione:
\begin{gather*}
    R_x(\tau)=3e^{-|\tau|}
\end{gather*}
Calcolare la densità spettrale di potenza di $y(t)$, il suo valore atteso, la sua funzione di autocorrelazione e la sua potenza. 
Si calcola il valore atteso:
\begin{gather}
    \mu_y=E[y(t)]=E[4x(t-3)+4x(t+3)]=4E[x(t-3)]+4E[x(t+3)]=0
\end{gather}
Si calcola ora la funzione di autocorrelazione:
\begin{gather*}
    R_y(\tau)=E[y(t+\tau)y^*(t)]=E[\{4x(t+\tau-3)+4x(t+\tau+3)\}\{4x(t-3)+4x(+3)\}]\\
    16E[x(t+\tau-3)x(t-3)]+16E[x(t+\tau-3)x(t+3)]\\
    +16E[x(t+\tau+3)x(t-3)]+16E[x(t+\tau+3)x(t+3)]\\
    16R_x(\tau-6)+16R_x(\tau)+16R_x(\tau)+16R_X(\tau+6)\\
    R_y(\tau)=16R_x(\tau-6)+32R_x(\tau)+16R_x(\tau+6)\\
    R_y(\tau)=16\left(3e^{-|\tau-6|}+6e^{-|\tau|}+3e^{-|\tau+6|}\right)\tageq
\end{gather*}
Si calcola ora la potenza:
\begin{equation}
    P_y=R_y(0)=16(3e^{-6}+6+3e^{-6})=96+96e^{-6}
\end{equation}
Si calcola la densità spettrale di potenza come il transito attraverso un filtro, poiché il segnale $y(t)$ può essere espresso come:
\begin{gather*}
    y(t)=x(t)*[4\delta(t-3)+4\delta(t+3)]\\
    h(t)=4\delta(t-3)+4\delta(t+3)\\
    H(f)=\displaystyle8\cos(6\pi f)\\
    G_x(f)=\mathscr{F}\left\{3e^{-|\tau|}\right\}=\displaystyle\frac{3}{1+2i\pi f}+\frac{3}{1-2i\pi f}=\frac{6}{1+4\pi f}\\
    G_y(f)=64\cos^2(6\pi f)\displaystyle\frac{6}{1+4\pi f}\tageq
\end{gather*}

\subsubsection*{Esercizio 78}

Dato un processo $x(t)$ stazionario, avente potenza $P_X=19$ e covarianza $C_X(\tau)=3\tri(\tau)$, determinare il valor medio e la varianza: 
\begin{gather*}
    C_x(0)=E[(x(t)-\mu_x)^2]=\sigma_x^2=3\tageq\\
    R_x(0)=E[x^2(t)]=P_X=\mu_x^{(2)}=19\\
    \mu_x=\sqrt{\sigma_x^2-\mu_x^{(2)}}=\sqrt{16}=4\tageq
\end{gather*}
Si suppone il segnale attraversa un filtro di risposta impulsiva $h(t)=\rect(t/3)$, calcola il valore medio dell'uscita $y(t)$:
\begin{gather}
    \mu_y=E[y(t)]=E[x(t)*h(t)]=E[x(t)]*h(t)=\displaystyle\int_{-\infty}^{+\infty}\mu_x\rect\left(\frac{t}{3}\right)\df t=4\int_{-3}^3\df t=24
\end{gather}

\subsubsection*{Esercizio 79}

Dato un processo $x(t)$ stazionario, di valore medio $\mu_x=-3$, potenza $P_X=12$ e coefficiente di correlazione $\rho_X(\tau)=\sinc(\tau/4)$, calcolare la densità spettrale 
di potenza:
\begin{gather*}
    \sigma_x^2=P_X-\mu_X^2=12-9=3\\
    C_x(\tau)=\rho_x(\tau)\sigma_x^2=3\sinc\displaystyle\left(\frac{\tau}{4}\right)\\
    R_x(\tau)=C_x(\tau)+\mu_x^2=3\sinc\displaystyle\left(\frac{\tau}{4}\right)+9\\
    G_x(f)=\mathscr{F}\{R_X(\tau)\}=12\rect(4f)+9\delta(f)\tageq
\end{gather*}
Questo processo viene filtrato con un filtro avente una funzione di trasferimento $H(f)=\cos(4\pi f)$, calcolare la potenza 
di $y(t)$:
\begin{gather*}
    G_y(f)=|H(f)|^2G_x(f)=\cos^2(4\pi f)\left[12\rect(4f)+9\delta(f)\right]\\
    P_y=\displaystyle\int_{-\infty}^{+\infty}G_y(f)\df f=12\int_{-1/8}^{1/8}\cos^2(4\pi f)\df f+9=6\cancelto{0}{\int_{-1/8}^{1/8}\cos(8\pi f)\df f}+6\int_{-1/8}^{1/8}\df f+9\\
    \displaystyle6\frac{2}{8}+9\\
    P_y=\displaystyle\frac{21}{2}\tageq
\end{gather*}
Sia il processo $z(t)$:
\begin{gather*}
    z(t)=x(t)+x(t-6)
\end{gather*}
Determinare il valore medio, la varianza e la funzione di autocorrelazione di $z(t)$:
\begin{gather*}
    \mu_z=E[z(t)]=E[x(t)+x(t-6)]=2\mu_x=-6\tageq\\
    R_z(\tau)=E[z(t+\tau)z(t)]=E[x(t+\tau)x(t)]+E[x(t+\tau-6)x(t)]+E[x(t+\tau)x(t-6)]\\
    +E[x(t+\tau-6)x(t-6)]=R_x(\tau)+R_x(\tau-6)+R_x(\tau+6)+R_X(\tau)\\
    2R_x(\tau)+2R_x(\tau-6)+R_x(\tau-6)\\
    6\sinc\displaystyle\left(\frac{\tau}{4}\right)+18+3\sinc\left(\frac{\tau-6}{4}\right)+9+3\sinc\left(\frac{\tau+6}{4}\right)+9\\
    R_z(\tau)=6\sinc\left(\displaystyle\frac{\tau}{4}\right)+6\sin\left(\frac{\tau+6}{4}\right)+36\tageq\\
    \sigma_z^2=P_z-\mu_z^2=R_z(0)-36=42+6\sinc\displaystyle\left(\frac{3}{2}\right)-36=6\left[1+\sinc\left(\frac{3}{2}\right)\right]\tageq
\end{gather*}  

\subsubsection*{Esercizio 80}

Dato un processo $x(t)=A+n(t)$, il rumore è bianco, ha varianza $\sigma_n^2$ nota e valor medio nullo $\mu_n=0$, calcolare la potenza del processo $x$:
\begin{gather}
    P_x=E[x^2(t)]=E[A^2]+2A\cancelto{0}{E[n(t)]}+E[n^2(t)]=A^2+\sigma_n^2
\end{gather}
Il segnale viene filtrato in un filtro istantaneo:
\begin{gather*}
    y(t)=x(t)+x(t-1)
\end{gather*}
Calcolare la potenza del segnale in uscita dal filtro:
\begin{gather*}
    P_y=E[y^2(t)]=E[(x(t)+x(t-1))^2]=E[x^2(t)]+2E[x(t)x(t-1)]+E[x^2(t-1)]\\
    P_x+2R_x(1)+P_x=2A^2+2\sigma_n^2+2R_x(1)\\
    R_x(\tau)=E[(A+n(t+\tau))(A+n(t))]=A^2+A\cancelto{0}{E[n(t)]}+A\cancelto{0}{E[n(t+\tau)]}+R_n(\tau)=A^2+\sigma_n^2\delta(\tau)\\
    P_y=2A^2+\sigma_n^2+2A^2+2\sigma_n^2\cancelto{0}{\delta(1)}\\
    P_y=4A^2+2\sigma_n^2\tageq
\end{gather*}

\subsubsection*{Esercizio 81}

Dati due processi aleatori $x(t)$ e $y(t)$ statisticamente indipendenti, bianchi nella banda $[-W,W]$ con densità di probabilità uniforme nell'intervallo $[-A,A]$. Calcolare il 
valore medio $\mu_z$, la potenza $P_z$, la correlazione $R_z(\tau)$, la densità spettrale di potenza $G_z(f)$ e la densità di probabilità $P_z(z)$ del 
processo $z(t)$:
\begin{gather*}
    z(t)=x(t)+y(t)
\end{gather*}
Si calcola il valore medio, la potenza, la densità spettrale di potenza e la correlazione del processo $x(t)$:
\begin{gather*}
    P_X(x)=\displaystyle\frac{1}{2A}\rect\left(\frac{x}{2A}\right)\\
    \mu_x=\intinf xP_X(x)\df x=\frac{1}{2A}\int_{-A}^Ax\df x=0\\
    P_x=\intinf x^2P_X(x)\df x=\frac{1}{2A}\int_{-A}^Ax^2\df x=\frac{A^2}{3}\\
    G_x(f)=\displaystyle\frac{P_x}{2W}\rect\left(\frac{f}{2W}\right)=\frac{A^2}{6W}\rect\left(\frac{f}{2W}\right)\\
    R_x(\tau)=\displaystyle\frac{A^2}{3}\sinc(2W\tau)
\end{gather*}
Poiché il processo $y(t)$ è definito analogamente ad $x(t)$, i suoi parametri caratteristici sono anch'essi analoghi:
\begin{gather*}
    P_Y(f)=\displaystyle\frac{1}{2A}\rect\left(\frac{y}{2A}\right)\\
    \mu_y=0\\
    P_y=\displaystyle\frac{A^2}{3}\\
    G_y(f)=\displaystyle\frac{A^2}{6W}\rect\left(\frac{f}{2W}\right)\\
    R_y(\tau)=\displaystyle\frac{A^2}{3}\sinc(2W\tau)
\end{gather*}
Si calcola ora la densità di probabilità del processo $z(t)$, ed il suo valore medio: 
\begin{gather*}
    P_Z(z)=P_X(x)*P_Y(y)=\frac{1}{4A^2}\left[\rect\left(\frac{z}{2A}\right)*\rect\left(\frac{z}{2A}\right)\right]\\
    P_Z(z)=\displaystyle\frac{1}{4A^2}\tri\left(\frac{z}{2A}\right)\tageq
\end{gather*}
Poiché il triangolo è simmetrico, e centrato in $z=0$, il valore medio è nullo:
\begin{equation}
    \mu_z=0
\end{equation}
Si calcola la potenza e la correlazione di $z(t)$:
\begin{gather}
    P_z=E[z^2(t)]=E[(x(t)+y(t))^2]=P_x+P_y+2\mu_x\mu_y=\displaystyle\frac{2A^2}{3}\\
    R_z(\tau)=E[z(t+\tau)z(t)]=R_x(\tau)+R_y(\tau)+2\mu_x\mu_y=\displaystyle\frac{2A^2}{3}\sinc(2W\tau)
\end{gather}
Si calcola la densità spettrale del processo $z(t)$:
\begin{equation}
    G_z(f)=\displaystyle\frac{2A^3}{6W}\rect\left(\frac{f}{2W}\right)
\end{equation}

\subsubsection*{Esercizio 82}

Dato il processo $y(t)$:
\begin{gather*}
    y(t)=x(t-T)+x(t+T)
\end{gather*}
Dove $x(t)$ è un processo armonico:
\begin{gather*}
    x(t)=\displaystyle\cos\left(2\pi f_0t+\varphi\right)\\
    P_{\Phi}(\varphi)=\displaystyle\frac{1}{2\pi}\rect\left(\frac{\varphi-\pi}{2\pi}\right)
\end{gather*}
Calcolare il valore medio $\mu_y$ la potenza $P_y$, la funzione correlazione a la densità spettrale di potenza $G_y(f)$ del processo $y(t)$. 
Si considera il processo $y(t)$ l'uscita di un filtro di risposta impulsiva $h(t)$:
\begin{gather*}
    y(t)=x(t)[\delta(t-T)+\delta(t+T)]\\
    h(t)=\delta(t-T)+\delta(t+T)\\
    H(f)=2\cos\left(2\pi fT\right)
\end{gather*}
Si calcola il valore medio del processo $x(t)$:
\begin{gather*}
    \mu_x=E[\cos(2\pi f_0t+\varphi)]=0
\end{gather*}
Si calcola ora il valore medio del processo $y(t)$:
\begin{gather}
    \mu_y=E[y(t)]=E[x(t-T)]+E[x(t+T)]=2\mu_x
\end{gather}
Si considera la correlazione e la densità spettrale di potenza del processo armonico $x(t)$:
\begin{gather*}
    R_x(\tau)=\displaystyle\frac{1}{2}\cos(2\pi f_0\tau)\\
    G_x(f)=\displaystyle\frac{1}{4}\left[\delta(f-f_0)+\delta(f+f_0)\right]
\end{gather*}
Per le formule di passaggio attraverso un filtro si calcola la correlazione e la densità spettrale di energia del processo $y(t)$:
\begin{gather*}
    G_y(f)=G_x(f)|H(f)|^2=\displaystyle\frac{1}{4}\left[\delta(f-f_0)+\delta(f+f_0)\right]\cdot4\cos^2(2\pi fT)\\
    \cos^2(2\pi f_0T)\delta(f-f_0)+\cos^2(-2\pi f_0T)\delta(f+f_0)\\
    G_y(f)=\cos^2(2\pi f_0T)[\delta(f-f_0)+\delta(f+f_0)]\tageq
\end{gather*}
Antitrasformando la densità spettrale di energia si ottiene la funzione di autocorrelazione del processo $y(t)$:
\begin{equation}
    R_y(\tau)=2\cos^2(2\pi f_0T)\cos(2\pi f_0\tau)
\end{equation}
Si determina la potenza, calcolando la funzione di autocorrelazione in $\tau=0$:
\begin{equation}
    P_y=R_y(0)=2\cos^2(2\pi f_0T)
\end{equation}

\clearpage

\end{document}