\documentclass{article}

\usepackage{cancel}
\usepackage{tikz}
\usepackage{amsmath}
\usepackage[includehead,nomarginpar]{geometry}
\usepackage{graphicx}
\usepackage{amsfonts} 
\usepackage{verbatim}
\usepackage{mathrsfs}  
\usepackage{lmodern}
\usepackage{braket}
\usepackage{bookmark}
\usepackage{steinmetz}
\usepackage[italian]{babel}
\usepackage{pgfplots}
\usepackage{fancyhdr}
\usepackage{romanbarpagenumber}
\pgfplotsset{compat=1.16}

\setlength{\headheight}{12.0pt}
\addtolength{\topmargin}{-12.0pt}

%% inserire grafici

\hypersetup{
    colorlinks=true,
    linkcolor=black,
}
\tikzset{block/.style = {draw, fill=white, very thick, rectangle, minimum height=1cm, minimum width=2cm},  
         square/.style = {draw, fill=white, very thick, rectangle, minimum height=1cm, minimum width=1cm},
         sum/.style= {draw, fill=white, very thick, circle, node distance=0.5cm},  
         cross/.pic = {  \draw[rotate = 45] (-#1,0) -- (#1,0);
                         \draw[rotate = 45] (0,-#1) -- (0, #1);}}


\renewcommand{\contentsname}{Indice}

\title{Esercizi Svolti di Fondamenti di Telecomunicazioni}
\author{Giacomo Sturm}
\date{AA: 2023/2024 - Ing. Informatica}

\begin{document}

\pagenumbering{Roman}

\pagestyle{fancy}
\fancyhead{}\fancyfoot{}
\fancyhead[C]{Fondamenti di Telecomunicazioni - Giacomo Sturm}
\fancyfoot[C]{\thepage}

\maketitle

\vspace{10mm}

\begin{center}
    Sorgente del file LaTeX disponibile su \url{https://github.com/00Darxk/Fondamenti-di-Telecomunicazioni}
\end{center}

\clearpage

\tableofcontents

\clearpage

\pagenumbering{arabic}

\section{Convoluzioni, Correlazioni e Sistemi Ingresso-Uscita}

Calcolare la convoluzione tra un esponenziale unilatero ed un gradino, 
Si considera $\alpha\in\mathbb{R}^+$. 
\begin{equation*}
    z(t)=e^{-\alpha t}u(t)*u(t)=\displaystyle\int_{-\infty}^{+\infty}e^{-\alpha\tau}u(\tau)u(t-\tau)d\tau
\end{equation*}
Poiché è presente un gradino nell'integrale, la convoluzione assumerà valori nulli per $t<0$, ciò si può anche individure graficamente, poiché per gli stessi valori i due 
graficie del gradino e delò'esponenziale unilatero non si sovrappongono. All'aumentare del valore di $t$, il valore del segnale $z(t)$ aumenta sempre più lentamente, poiché i 
contributi dell'esponenziale vanno diminuendo. Poiché entrambi i gradini nell'integrale assumono valori unitari per $t>0$, il valore del segnale in questo intervallo 
dipende dalla funzione integrale dell'esponenziale, da $0$ al valore di $t$ corrente:
\begin{equation*}
    z(t)=\begin{cases}
        0&t<0\\
        \displaystyle\int_0^te^{-\alpha \tau}d\tau&t\geq0
    \end{cases}
\end{equation*}
Risolvendo l'integrale si ottiene:
\begin{equation*}
    \displaystyle\int_0^te^{-\alpha \tau}d\tau=\left|-\frac{e^{-\alpha\tau}}{\alpha}\right|^t_0=\frac{1-e^{-\alpha t}}{\alpha}
\end{equation*}
Per cui il segnale convoluzione in forma analitica risulta:
\begin{equation}
    z(t)=\begin{cases}
        0&t<0\\
        \displaystyle\frac{1-e^{-\alpha t}}{\alpha}&t\geq0
    \end{cases}
\end{equation}
Questo segnale tende asintoticamente a $1/\alpha$ per $t\to+\infty$. 

\begin{center}
    \begin{tikzpicture}[scale=2]
        \draw[->,very thick](-1,0)--(1.5,0)node[above]{$\tau$};
        \draw[->,very thick](0,-0.5)--(0,1.5)node[right]{$x(\tau)$};
        \node[below right]at(0,0){$0$};
        \node[above right]at(0,1){$1$};

        \draw[-,very thick]plot[smooth, domain=0:1.5](\x,{e^(-\x)});

        \draw[->,very thick](-1,-2.1)--(1.5,-2.1)node[above]{$\tau$};
        \draw[->,very thick](0,-2.6)--(0,-0.6)node[left]{$y(t-\tau)$};
        \node[below right]at(0,-2.1){$0$};
        \node[above right]at(0,-1.1){$1$};

        \draw[-,very thick](-1,-1.1)--(0.4,-1.1)--(0.4,-2.1)node[below right]{$t$};
        \draw[dashed,very thick](0.4,-1.1)--(0.4,0.67);


        \draw[->,very thick](2,-1)--(4,-1)node[above]{$t$};
        \draw[->,very thick](2.5,-1.5)--(2.5,0.5)node[right]{$z(t)$};
        \node[below right]at(2.5,-1){$0$};
        \node[left]at(2.5,0){$\displaystyle\frac{1}{\alpha}$};
        \draw[dashed](2.5,0)--(4,0);
        \draw[-,very thick]plot[smooth, domain=2.5:4](\x,{-e^(-2*(\x-2.5))});
    \end{tikzpicture}
\end{center}

Calcolare l'autoconvoluzione di un esponenziale unilatero
\begin{equation*}
    z(t)=e^{-\alpha t}u(t)*e^{-\alpha t}u(t)=\displaystyle\int_{-\infty}^{+\infty}e^{-\alpha\tau}u(\tau)e^{-\alpha(t-\tau)}u(t-\tau)d\tau
\end{equation*}
Poiché sono unilateri, non si sovrapporrano per $t<0$, quindi la convoluzione è nulla per quei valori di tempo. Può essere spiegato tramite la prorpietà del gradino di 
cambiare i limiti di integrazione, per cui invece di integrare da $-\infty$ a $+\infty$, si integra nell'intervallo dove si trova il gradino $u(\tau)$, ovvero da $0$ a $+\infty$: 
\begin{equation*}
    z(t)=\displaystyle\int_0^{+\infty}e^{-\alpha t}u(t-\tau)d\tau=e^{-\alpha t}\int_0^{+\infty}u(t-\tau)d\tau
\end{equation*}
L'area sottesa da un gradino ribaltato e traslato di un fattore $t>0$ da $0$ a $+\infty$ equivale all'area di un rettangolo di altezza $1$ e di base $t$:
\begin{equation*}
    z(t)=e^{-\alpha t}\displaystyle\int_0^td\tau=te^{-\alpha t}
\end{equation*}

In forma analitica risulta:
\begin{equation}
    z(t)=\begin{cases}
        0&t<0\\
        te^{-\alpha t}&t\geq0
    \end{cases}
\end{equation}
 

\begin{center}
    \begin{tikzpicture}[scale=2]
        \draw[->,very thick](-1,0)--(1.5,0)node[above]{$\tau$};
        \draw[->,very thick](0,-0.5)--(0,1.5)node[right]{$x(\tau)$};
        \node[below right]at(0,0){$0$};
        \node[above right]at(0,1){$1$};

        \draw[-,very thick]plot[smooth, domain=0:1.5](\x,{e^(-\x)});

        \draw[->,very thick](-1,-2.1)--(1.5,-2.1)node[above]{$\tau$};
        \draw[->,very thick](0,-2.6)--(0,-0.6)node[left]{$y(t-\tau)$};
        \node[below right]at(0,-2.1){$0$};
        \node[above right]at(0,-1.1){$1$};

        \draw[-,very thick]plot[smooth, domain=-1:0.4](\x,{e^(\x-0.67)-2.1});
        \draw[-,very thick](0.4,-2.1)node[below]{$t$}--(0.4,-1.337);
        \draw[dashed,very thick](0.4,-1.337)--(0.4,0.67);


        \draw[->,very thick](2,-1)--(4,-1)node[below]{$t$};
        \draw[->,very thick](2.5,-1.5)--(2.5,0)node[right]{$z(t)$};
        \node[below right]at(2.5,-1){$0$};
        \draw[-,very thick]plot[smooth, domain=2.5:4](\x,{2*(\x-2.5)*e^(-2*(\x-2.5))-1});
    \end{tikzpicture}
\end{center}

Calcolare la convoluzione tra due finestre. 
Si considerano due casi, dove le due finestra hanno base uguale, ed un caso dove hanno base differente. Si considera il caso $T_1=T_2$:
\begin{equation*}
    z(t)=\mbox{rect}\left(\displaystyle\frac{t}{T}\right)*\mbox{rect}\left(\frac{t}{T}\right)
\end{equation*}

La convoluzione assume valori nulli quando non si sovrappongono, ovvero per un valore $t+\displaystyle\frac{T}{2}<-\frac{T}{2}\to t<-T$. L'area sottesa dal prodotto 
di queste due finestre aumenta linearmente fino a quando non si sovrappongono per $t=0$, dove l'area assume valore massimo $T\cdot 1$, dopo il quale 
decresce linearmente fino a $t=T$. Se una delle due fosse stata traslata di un fattore $t_0$, l'intera convoluzione sarebbe stata traslata dello stesso fattore. Da notare 
che la convoluzione di due segnali pari genera un segnale pari.  
\begin{equation*}
    z(t)=\begin{cases}
        0&t<-T\\
        \displaystyle\int_{-T}^td\tau&-T\leq t<0\\
        \displaystyle\int_t^Td\tau&0\leq t<T\\
        0&t>T
    \end{cases}=\begin{cases}
        0&t<-T\land t>T\\
        T-|t|&-T\leq t<T\\
    \end{cases}
\end{equation*}
\begin{equation}
    z(t)=T\,\mbox{tri}\left(\displaystyle\frac{t}{T}\right)
\end{equation}
La convoluzione tra due finestre di base uguale risulta in un triangolo di base doppia $2T$, e scalata di un fattore pari alla base $T$. 

\begin{center}
    \begin{tikzpicture}[scale=2]
        \draw[->,very thick](-1,0)--(1.5,0)node[above]{$\tau$};
        \draw[->,very thick](0,-0.5)--(0,1.5)node[right]{$x(\tau)$};
        \node[below right]at(0,0){$0$};
        \node[above right]at(0,1){$1$};

        \draw[-,very thick](-0.5,0)node[below]{$-\displaystyle\frac{T}{2}$}--(-0.5,1)--(0.5,1)--(0.5,0)node[below]{$\displaystyle\frac{T}{2}$};

        \draw[->,very thick](-1,-2.1)--(1.5,-2.1)node[above]{$\tau$};
        \draw[->,very thick](0,-2.6)--(0,-0.6)node[left]{$y(t-\tau)$};
        \node[below right]at(0,-2.1){$0$};
        \node[above right]at(0,-1.1){$1$};

        \draw[-,very thick](0.3,-2.1)node[below]{$t+\displaystyle\frac{T}{2}$}--(0.3,-1.1)--(-0.7,-1.1)--(-0.7,-2.1)node[below]{$t-\displaystyle\frac{T}{2}$};
        \draw[dashed,very thick](0.3,-1.1)--(0.3,1);


        \draw[->,very thick](2,-1)--(5,-1)node[below]{$t$};
        \draw[->,very thick](3.5,-1.5)--(3.5,0.5)node[right]{$z(t)$};
        \node[below right]at(3.5,-1){$0$};
        \draw[-,very thick](2.5,-1)node[below]{$-T$}--(3.5,0)node[above right]{$T$}--(4.5,-1)node[below]{$T$};
    \end{tikzpicture}
\end{center}

Si considera ora il caso dove le due finestre hanno basi $T_1>T_2$:
\begin{equation*}
    z(t)=\mbox{rect}\left(\displaystyle\frac{t}{T_1}\right)*\mbox{rect}\left(\frac{t}{T_2}\right)
\end{equation*}

Poiché le due finestre sono simmetriche, si analizzano solo i casi per $t<0$, per poi aggiungere per simmetria l'equazione analitica per $t\geq0$. Le due finestre non si 
sovrappongono per $t+\displaystyle\frac{T_2}{2}<-\frac{T_1}{2}\to t<-\frac{1}{2}(T_1+T_2)$, per cui in quell'intervallo la convoluzione assume valore nullo. Il valore 
della convoluzione aumenta linearmente fino a quando la finestra più piccola trasla fino ad essere completamente interna alla finestra di base $T_1$ per un valore 
$t-\displaystyle\frac{T_2}{2}<-\frac{T_1}{2}\to t<-\frac{1}{2}(T_1-T_2)$. Quando le due finestre si sovrappongono, il valore della convoluzione è costante, e corrisponde 
all'area di un rettangolo di base, base della finestra più piccola $T_2$ ed altezza unitaria $T_2\cdot1$ fino a raggiungere $t_0$. Ribaltando questo segnale ottenuto 
si ottiene il segnale per valori $t\geq0$:
\begin{gather*}
    z(t)=\begin{cases}
        0&t<-\displaystyle\frac{1}{2}(T_1+T_2)\\
        \displaystyle\int_{-\frac{1}{2}(T_1+T_2)}^td\tau&-\frac{1}{2}(T_1+T_2)\leq t <-\frac{1}{2}(T_1-T_2)\\
        T_2&-\frac{1}{2}(T_1-T_2)\leq t<\frac{1}{2}(T_1-T_2)\\
        -\displaystyle\int_{-\frac{1}{2}(T_1+T_2)}^td\tau&\frac{1}{2}(T_1-T_2)\leq t <\frac{1}{2}(T_1+T_2)\\
        0&t\geq\displaystyle\frac{1}{2}(T_1+T_2)
    \end{cases}
\end{gather*}
\begin{gather}
    z(t)=\begin{cases}
        0&t<-\displaystyle\frac{1}{2}(T_1+T_2)\land t\geq\frac{1}{2}(T_1+T_2)\\
        \displaystyle\frac{1}{2}(T_1+T_2)-|t| &-\frac{1}{2}(T_1+T_2)\leq t <-\frac{1}{2}(T_1-T_2)\land \frac{1}{2}(T_1-T_2)\leq t <\frac{1}{2}(T_1+T_2)\\
        T_2&-\displaystyle\frac{1}{2}(T_1-T_2)\leq t<\frac{1}{2}(T_1-T_2)
    \end{cases}
\end{gather}

\begin{center}
    \begin{tikzpicture}[scale=2]
        \draw[->,very thick](-1,0)--(1.5,0)node[above]{$\tau$};
        \draw[->,very thick](0,-0.5)--(0,1.5)node[right]{$x(\tau)$};
        \node[below right]at(0,0){$0$};
        \node[above right]at(0,1){$1$};

        \draw[-,very thick](-0.7,0)node[below]{$-\displaystyle\frac{T_1}{2}$}--(-0.7,1)--(0.7,1)--(0.7,0)node[below]{$\displaystyle\frac{T_1}{2}$};

        \draw[->,very thick](-1,-2.1)--(1.5,-2.1)node[above]{$\tau$};
        \draw[->,very thick](0,-2.6)--(0,-0.6)node[left]{$y(t-\tau)$};
        \node[below right]at(0,-2.1){$0$};
        \node[above right]at(0,-1.1){$1$};

        \draw[-,very thick](0.3,-2.1)node[below]{$t+\displaystyle\frac{T_2}{2}$}--(0.3,-1.1)--(-0.8,-1.1)--(-0.8,-2.1)node[below]{$t-\displaystyle\frac{T_2}{2}$};
        \draw[dashed,very thick](0.3,-1.1)--(0.3,1);


        \draw[->,very thick](2,-1)--(5,-1)node[below]{$t$};
        \draw[->,very thick](3.5,-1.5)--(3.5,0.5)node[right]{$z(t)$};
        \node[above right]at(3.5,-1){$0$};
        \draw[-,very thick](2.5,-1)node[below left]{$-\displaystyle\frac{1}{2}(T_1+T_2)$}--(3,0)--(3.5,0)node[above right]{$T_2$}--(4,0)--(4.5,-1)node[below right]{$\displaystyle\frac{1}{2}(T_1+T_2)$};
        \draw[dashed, very thick](3,0)--(3,-1)node[below]{$\displaystyle-\frac{1}{2}(T_1-T_2)$};
        \draw[dashed, very thick](4,0)--(4,-1)node[below]{$\displaystyle\frac{1}{2}(T_1-T_2)$};
    \end{tikzpicture}
\end{center}

Il grafico di questa convoluzione rappresenta un trapezio di altezza $T_2$, base maggiore $T_1+T_2$ e base minore $T_1-T_2$ Si nota come l'autoconvoluzione di due finestre 
di base uguali rappresenta un caso speciale di questo trapezio, avente base minore nulla $T-T=0$ e base doppia rispetto alla finestra $T+T=2T$, ciò equivale ad un triangolo 
di altezza $T$ e base $2T$. 



Calcolare la convoluzione tra i segnali $x$ e $y$:
\begin{gather*}
    x(t)=\displaystyle t\,\mbox{rect}\left(\frac{t-T}{2T}\right)\\
    y(t)=\displaystyle\mbox{rect}\left(\frac{t+T}{2T}\right)\\
    z(t)=x(t)*y(t)=\displaystyle\int_{-\infty}^{+\infty}\tau\,\mbox{rect}\left(\frac{\tau-T}{2T}\right)\mbox{rect}\left(\frac{t-\tau+T}{2T}\right)d\tau\\
    z(t)=\begin{cases}
        0&t<-2T\\
        \displaystyle\int_{0}^{t+2T}\tau d\tau&-2T\leq t<0\\
        \displaystyle\int_{t}^{2T}\tau d\tau&0\leq t<2T\\
        0&t\geq2T
    \end{cases}
\end{gather*}
\begin{gather}
    z(t)=\begin{cases}
        0&t<-2T\land t\geq2T\\
        \displaystyle\frac{1}{2}t^2+2Tt+2T^2&-2T\leq t<0\\
        \displaystyle2T^2-\frac{1}{2}t^2&0\leq t<2T
    \end{cases}
\end{gather}


Calcolare la convoluzione tra queste due finestre:
\begin{gather*}
    z(t)=z(t)*y(t)\\
    x(t)=\displaystyle\mbox{rect}\left(\frac{t-3T}{4T}\right)\\
    y(t)=-\displaystyle\mbox{rect}\left(\frac{t+2T}{2T}\right)\\
    z(t)=\begin{cases}
        0&t<-2T\\
        -\displaystyle\int^{3T+t}-{T}d\tau&-2T\leq t<0\\
        -2T&0\geq t<2T\\
        -\displaystyle\int_{T+t}^{5T}d\tau&2T\leq t<4T\\
        0&t\geq 4T
    \end{cases}
\end{gather*}
\begin{gather}
    z(t)=\begin{cases}
        0&t<-2T \land t\geq 4T\\
        -2T-t&-2T\leq t<0\\
        -2T&0\leq t<2T\\
        -4T+t&2T\leq t<4T
    \end{cases}
\end{gather}


Calcolare la correlazione tra due finestre:
\begin{equation*}
    \mbox{rect}\displaystyle\left(\frac{t-\frac{T_1}{2}}{T_1}\right)\otimes\mbox{rect}\left(\frac{t-\frac{T_2}{2}}{T_2}\right)
\end{equation*}

Si considera la prima finestra di base maggiore $T_1>T_2$. Per facilitare il calcolo si esprime come una convoluzione. I due segnali non si sovrappongono per valori 
$T_2-t<0\to t\geq T_2$, per cui la correlazione è nulla. Cominciano a sovrapporsi da $t<T_2$ fino a quando la finestra più piccola non si trova interamente nella prima 
finestra per $-t<0\to t\geq0$. La finestra più piccola si trova interamente in quella più grande, risultando in un'area di $T_2$, fino ad un valore 
$T_2-t<T_1\to t\geq -(T_1-T_2)$. L'area comincia a scendere fino ad un valore $-t<T_1\to t\geq -T_1$. Per valori più piccoli di $-T_1$ le due finestre non si sovrappongono e la 
correlazione risulta nulla.
\begin{equation*}
    R_{xy}(t)=\begin{cases}
        0&t\geq T_2\\
        \displaystyle\int_t^{T_2}dt& 0\leq t<T_2\\
        T_2& -(T_1+T_2)\leq t<0\\
        \displaystyle\int_{-T_1}^tdt& -T_1\leq t<-(T_1-T_2)\\
        0&t<-T_1
    \end{cases}
\end{equation*}
\begin{equation}
    R_{xy}(t)=\begin{cases}
        0& t<-T_1\land t\geq T_2\\
        T_1+t& T_1\leq t<-(T_1-T_2)\\
        T_2& -(T_1-T_2)\leq t<0\\
        T_2-t &  0\leq t<T_2
    \end{cases}
\end{equation}


\begin{center}
    \begin{tikzpicture}[scale=2]
        \draw[->,very thick](-1,0)--(1.5,0)node[above]{$\tau$};
        \draw[->,very thick](0,-0.5)--(0,1.5)node[right]{$x(\tau)$};
        \node[below right]at(0,0){$0$};
        \node[above right]at(0,1){$1$};

        \draw[-,very thick](0,0)--(0,1)--(1.2,1)--(1.2,0)node[below]{$T_1$};

        \draw[->,very thick](-1,-2.1)--(1.5,-2.1)node[above]{$\tau$};
        \draw[->,very thick](0,-2.6)--(0,-0.6)node[left]{$y(t+\tau)$};
        \node[below left]at(0,-2.1){$0$};
        \node[above right]at(0,-1.1){$1$};

        \draw[-,very thick](0.3,-2.1)node[below]{$T_2-t$}--(0.3,-1.1)--(-0.8,-1.1)--(-0.8,-2.1)node[below]{$-t$};
        \draw[dashed,very thick](0.3,-1.1)--(0.3,1);


        \draw[->,very thick](2,-1)--(4.5,-1)node[below]{$t$};
        \draw[->,very thick](3.5,-1.5)--(3.5,0.5)node[right]{$z(t)$};
        \node[above right]at(3.5,-1){$0$};
        \draw[-,very thick](2.5,-1)node[below left]{$-T_1$}--(3,0)--(3.5,0)--(4,-1)node[below]{$T_2$};
        \draw[dashed, very thick](3,0)--(3,-1)node[below]{$-(T_1-T_2)$};
    \end{tikzpicture}
\end{center}

Dato il sistema definito dall'equazione $y$:
\begin{gather*}
    y(t)=x(t)+3u(t-1)
\end{gather*}
Si dimostra che non è lineare:
\begin{gather*}
    x_1(t)\to y_1(t)=x_1(t)+3u(t-1)\\
    x_2(t)\to y_2(y)=x_2(t)+3u(t-1)\\
    ax_1(t)+bx_2(t)\to ax_1(t)+bx_2(t)+3u(t-1)\neq ay_1(t)+by_2(t)
\end{gather*}
Si dimostra che non è tempo invariante:
\begin{gather*}
    x(t-\tau)\to x(t-\tau)+3u(t-1)\neq y(t-\tau)=x(t-\tau)+3u(t-\tau-1)
\end{gather*}
Sicuramente non è un filtro. Mentre è causale, poiché il gradino è ritardato, e quindi l'uscita non dipende da valori futuri.


Calcolare l'uscita di un filtro discreto di risposta impulsiva $h[n]$, con un entrata parametrizzata $x[n]$:
\begin{gather*}
    h[n]=\displaystyle\frac{1}{4}\delta[n+1]+\frac{1}{2}\delta[n]+\frac{1}{4}\delta[n-1]\\
    x[n]=\cos(2\pi\phi n)\\
    \phi_1=0\to x_1[n]=1\\
    y_1[n]=h[n]*x_1[n]=\displaystyle\left[\frac{1}{4}\delta[n+1]+\frac{1}{2}\delta[n]+\frac{1}{4}\delta[n-1]\right]*x_1[n]\\
    \displaystyle\frac{1}{4}x_1[n+1]+\frac{1}{2}x_1[n]+\frac{1}{4}x_1[n-1]=\frac{1}{4}+\frac{1}{2}+\frac{1}{4}=1
\end{gather*}
\begin{equation}
    y_1[n]=1
\end{equation}
\begin{gather*}
    \phi_2=\displaystyle\frac{1}{4}\to x_2[n]=\cos\left(\displaystyle\frac{\pi n}{2}\right)\\
    y_2[n]=\left[\displaystyle\frac{1}{4}\delta[n+1]+\frac{1}{2}\delta[n]+\frac{1}{4}\delta[n-1]\right]*\cos\left(\frac{\pi n}{2}\right)\\
    \displaystyle\frac{1}{4}\cos\left(\frac{\pi (n+1)}{2}\right)+\frac{1}{2}\cos\left(\frac{\pi n}{2}\right)+\frac{1}{4}\cos\left(\frac{\pi (n-1)}{2}\right)\\
    \displaystyle\frac{1}{4}\cos\left(\frac{\pi (n+1)}{2}\right)=-\frac{1}{4}\cos\left(\frac{\pi (n-1)}{2}\right)\,\forall k\in\mathbb{Z}
\end{gather*}
\begin{equation}
    y_2[n]=\displaystyle\frac{1}{2}\cos\left(\frac{\pi n}{2}\right)
\end{equation}
\begin{gather*}
    \phi_3=\displaystyle\frac{1}{2}\to x_3[n]=\cos\left(\displaystyle{\pi n}\right)
\end{gather*}
\begin{equation}
    y[n]=\displaystyle\frac{1}{4}\cos(\pi (n+1))+\frac{1}{2}\cos(\pi n)+\frac{1}{4}\cos(\pi(n-1))=0
\end{equation}

\clearpage

\section{Serie di Fourier}

Dato il segnale:
\begin{equation*}
    x(t)=\left|\cos\left(\displaystyle\frac{2\pi t}{T}\right)\right|
\end{equation*}

Si determinano i coefficienti dell'espansione di Fourier tramite la definizione:
\begin{gather*}
    c_k=\displaystyle\frac{2}{T}\int_{-\frac{T}{4}}^{\frac{T}{4}}\cos\left(\displaystyle\frac{2\pi t}{T}\right)e^{-i\frac{4\pi kt}{T}}dt\\
    \displaystyle\frac{2}{T}\int_{-\frac{T}{4}}^{\frac{T}{4}}\cos\left(\displaystyle\frac{2\pi t}{T}\right)\left[\cos\left(\frac{4\pi kt}{T}\right)-i\sin\left(\frac{4\pi kt}{T}\right)\right]dt\\
    \displaystyle\frac{2}{T}\int_{-\frac{T}{4}}^{\frac{T}{4}}\cos\left(\frac{2\pi t}{T}\right)\cos\left(\frac{4\pi kt}{T}\right)dt
    {-i\int_{-\frac{T}{4}}^{\frac{T}{4}}\cos\left(\frac{2\pi t}{T}\right)\sin\left(\frac{4\pi kt}{T}\right)dt}
\end{gather*}
Poiché è una funzione pari, il primo integrale corrisponde al doppio dell'integrale su solo una metà dell'intervallo, mentre il secondo integrale, essendo una funzione 
dispari, assume valore nullo su un intervallo simmetrico:
\begin{equation*}
    \displaystyle\frac{4}{T}\int_{0}^{\frac{T}{4}}\cos\left(\frac{2\pi t}{T}\right)\cos\left(\frac{4\pi kt}{T}\right)dt-i\cdot0
\end{equation*}


Per le formule di prostaferesi si ottiene una somma di due coseni. Integrandoli separatamente si ottiene:
\begin{gather*}
    c_k=\displaystyle\frac{4}{T}\frac{1}{2}\int_{-\frac{T}{4}}^{\frac{T}{4}}\cos\left(\frac{2(2k+1)\pi t}{T}\right)+\cos\left(\frac{2(2k-1)\pi t}{T}\right)dt\\
    \displaystyle\left[\frac{2}{T}\frac{T}{2(2k+1)\pi}\sin\left(\frac{2(2k+1)\pi t}{T}\right)+
    \frac{2}{T}\frac{T}{2(2k-1)\pi}\sin\left(\frac{2(2k-1)\pi t}{T}\right)\right]_{0}^{\frac{T}{4}}\\
    \left[\displaystyle\frac{\sin(2(2k+1)\pi t/T)}{(2k+1)\pi}+\frac{\sin(2(2k-1)\pi t/T)}{(2k-1)\pi}\right]_0^{\frac{T}{4}}\\
    c_k=\displaystyle\frac{\sin(2(2k+1)\pi/2)}{(2k+1)\pi}+\frac{\sin(2(2k-1)\pi/2)}{(2k-1)\pi}
\end{gather*}

L'argomento del seno, lo rende tale che assume solo valori pari a $\pm1$. Per valori pari di $k$, il primo seno assume valore di $1$, mentre l'altro $-1$, accade l'opposto per 
valori dispari di $k$. 
I due seni assumono sempre valori discordi e unitari, quindi si possono esprimere come: 
\begin{gather}
    c_k=\displaystyle\frac{1}{\pi}\frac{(-1)^k}{2k+1}-\frac{1}{\pi}\frac{(-1)^k}{2k-1}=-\frac{2}{\pi}\frac{1}{4k^2-1}(-1)^k
\end{gather}

Dato un segnale periodico:
\begin{gather*}
    x(t)=2\cos\displaystyle\left(\frac{2\pi t}{T}+\frac{\pi}{4}\right)+6\sin\left(\frac{2\pi t}{3T}\right)
\end{gather*}
Il primo passaggio nella rappresentazione di Fourier corrisponde all'individuazione del periodo del segnale $x$. Il periodo della somma di due seganli periodici corrisponde 
al minimo comune multiplo tra il periodo dei due segnali. In questo caso il periodo del segnale $x$ è $T'=3T$. Per cui si può riscrivere come:
\begin{gather*}
    x(t)=\displaystyle2\cos\left(\frac{6\pi t}{T'}+\frac{\pi}{4}\right)+6\sin\left(\frac{2\pi t}{T'}\right)\\
    \displaystyle e^{i\left(\frac{6\pi t}{T'}+\frac{\pi}{4}\right)}+e^{-i\left(\frac{6\pi t}{T'}+\frac{\pi}{4}\right)}+\frac{3}{i}e^{i\frac{2\pi t}{T'}}-\frac{3}{i}e^{-i\frac{2\pi t}{T'}}
\end{gather*} 
Per confronto diretto si ottengono i seguenti coefficienti:
\begin{gather}
    c_k=
    \begin{cases}
        0&\forall k\neq\pm1\land\pm3\\
        \displaystyle\frac{3}{i}=-3i=-3e^{\frac{\pi}{2}} &k=1\\
        -\displaystyle\frac{3}{i}=3i=3e^{\frac{\pi}{2}}&k=-1\\
        \displaystyle e^{\frac{\pi}{4}}=\frac{1}{\sqrt{2}}(1+i)&k=3\\
        \displaystyle e^{-\frac{\pi}{4}}=\frac{1}{\sqrt{2}}(1-i)&k=-3
    \end{cases}
\end{gather}


Dato un segnale:
\begin{gather*}
    x(t)=\sin\displaystyle(2\pi t)+\cos(4\pi t+\phi)
\end{gather*}
Questo segnale ha un periodo $T=1$. Per confronto diretto si ottengono dei coefficienti:
\begin{gather*}
    x(t)=\displaystyle \frac{e^{2i\pi t}-e^{-2i\pi t}}{2i}+\frac{e^{i\phi}e^{4i\pi t}+e^{-i\phi}e^{-4i\pi t}}{2}
\end{gather*}
\begin{gather}
    c_k=
    \begin{cases}
        0&\forall k\neq\pm1\land\pm2\\
        \displaystyle\frac{1}{2i}=-\frac{i}{2} &k=1\\
        -\displaystyle\frac{1}{2i}=\frac{i}{2}&k=-1\\
        \displaystyle\frac{e^{i\phi}}{2}&k=2\\
        \displaystyle\frac{e^{-i\phi}}{2}&k=-2
    \end{cases}
\end{gather}

Dato il segnale:
\begin{gather*}
    x(t)=\sin^2\displaystyle\left(\frac{2\pi t}{T}+\phi\right)=\frac{1}{2}\left(1-\cos\left(\frac{4\pi t}{T}+2\phi\right)\right)
\end{gather*}
Questo segnale ha periodo $T/2$. Per confronto diretto si ottengono i coefficienti:
\begin{gather*}
    x(t)=\displaystyle\frac{1}{2}-\frac{1}{4}e^{2i\phi}e^{i\frac{4\pi t}{T}}-\frac{1}{4}e^{-2i\phi}e^{-i\frac{4\pi t}{T}}
\end{gather*}
\begin{gather}
    c_k=\begin{cases}
        0&\forall k\neq0\land\pm2\\
        \displaystyle\frac{1}{2}&k=0\\
        -\displaystyle\frac{1}{4}e^{2i\phi}&k=1\\
        -\displaystyle\frac{1}{4}e^{-2i\phi}&k=-1
    \end{cases}
\end{gather}


Dato il segnale dente di sega:
\begin{center}
    \begin{tikzpicture}[scale=2]
        \draw[->](-2.5,0)--(2.5,0)node[above]{$t$};
        \draw[->](0,-1)--(0,1)node[right]{$x(t)$};
        \draw[-](-2.5,-0.5)--(-1.5,0.5)--(-1.5,-0.5)--(-0.5,0.5)--(-0.5,0)node[above right]{$\displaystyle-\frac{T}{2}$}--(-0.5,-0.5)--(0.5,0.5)--(0.5,0)node[above right]{$\displaystyle\frac{T}{2}$}--(0.5,-0.5)--(1.5,0.5)--(1.5,-0.5)--(2.5,0.5);
        \draw[dashed](-2.5,0.5)--(0,0.5)node[above right]{$\displaystyle\frac{T}{2}$}--(2.5,0.5);
        \draw[dashed](-2.5,-0.5)--(0,-0.5)node[below right]{$-\displaystyle\frac{T}{2}$}--(2.5,-0.5);
    \end{tikzpicture}
\end{center}

Si calcolano i coefficienti di Fourier tramite la definizione, e si calcola l'integrale per parti: 
\begin{gather*}
    c_k=\displaystyle\frac{1}{T}\int_{-\frac{T}{2}}^{\frac{T}{2}}te^{-i\frac{2\pi kt}{T}}dt=\displaystyle\left[-\frac{1}{T}\frac{T}{2i\pi k}te^{-i\frac{2\pi kt}{T}}\right]_{-\frac{T}{2}}^{\frac{T}{2}}-
    \frac{1}{T}\int_{-\frac{T}{2}}^{\frac{T}{2}}-\frac{T}{2i\pi k}e^{-i\frac{2\pi kt}{T}dt}\\
    \displaystyle-\frac{T}{2i\pi k}\left(\frac{e^{-i\pi k}+e^{i\pi k}}{2}\right)+\frac{1}{2i\pi k}\left[\frac{T}{2i\pi k}e^{-i\frac{2\pi kt}{T}}\right]_{-\frac{T}{2}}^{\frac{T}{2}}\\
    \displaystyle-\frac{T}{2i\pi k}\left(\frac{e^{i(-\pi k)}+e^{-i(-\pi k)}}{2}\right)-\frac{2iT}{4\pi^2k^2}\left(\frac{e^{i(-\pi k)}-e^{-i(-\pi k)}}{2i}\right)\\
    \displaystyle-\frac{T}{2i\pi k}\cos\left(-\pi k\right)+\frac{2iT}{4\pi^2k^2}\sin(-\pi k)
\end{gather*}
La componente sinusoidale è nulla per ogni valore di $k$, per cui il coefficiente generico si esprime come:
\begin{gather}
    c_k=\displaystyle-\frac{T}{2i\pi k}\cos(-\pi k)=\frac{Ti}{2\pi k}\cos(\pi k)=\frac{Ti}{2\pi k}(-1)^k \;\;k\neq0
\end{gather}
Il coefficiente $c_0$ è nullo, poiché rappresenta il valor medio del periodo. In questo caso essendo un segnale dispari, il valor medio nel periodo è nullo:
\begin{equation}
    c_0=\displaystyle\frac{1}{T}\int_{-\frac{T}{2}}^{\frac{T}{2}}t\,dt=0
\end{equation}

Dato il seguente segnale dente di sega:
\begin{center}
    \begin{tikzpicture}[scale=2]
        \draw[->](-3,-0.5)--(2,-0.5)node[above]{$t$};
        \draw[->](-0.5,-1.25)--(-0.5,0.75)node[right]{$x(t)$};
        \draw[-](-2.5,-0.5)--(-1.5,0.5)--(-1.5,-0.5)--(-0.5,0.5)--(-0.5,0)--(-0.5,-0.5)--(0.5,0.5)--(0.5,0)--(0.5,-0.5)node[below right]{$T$}--(1.5,0.5)--(1.5,-0.5);
        \draw[dashed](-3,0.5)--(0,0.5)node[above right]{$T$}--(2,0.5);
    \end{tikzpicture}
\end{center}
Si può esprimere come il precedente segnale analizzato, traslato nel tempo e nello spazio di uno stesso fattore $T/2$. La traslazione nello spazio può essere sia un ritardo 
che un anticipo, poiché si ottiene lo stesso segnale risultante:
\begin{equation*}
    y(t)=x\left(t\pm\frac{T}{2}\right)+\displaystyle\frac{T}{2}
\end{equation*}
Si considera in questo caso un ritardo di $T/2$. Conoscendo i coefficienti di $c_k$, si può attuare una traslazione nel tempo, per ottenere i coefficienti della serie di Fourier 
del segnale $y$:
\begin{gather}
    d_k=\begin{cases}
        \displaystyle c_0+\frac{T}{2}=\frac{T}{2} &k=0\\
        c_ke^{-i\frac{2\pi k}{T}\frac{T}{2}}=\displaystyle\frac{T}{2i\pi k}(-1)^ke^{-i\pi k}&k\neq0
    \end{cases}
\end{gather}



Dato un segnale treno di triangoli:
\begin{equation*}
    x(t)=\displaystyle\sum_{k=-\infty}^{+\infty}\mbox{tri}\left(\frac{2(t-kT)}{\tau}\right)
\end{equation*}
\begin{center}
    \begin{tikzpicture}[scale=2]
        \draw[->](-2.5,0)--(2.5,0)node[above]{$t$};
        \draw[->](0,-0.5)--(0,1.5)node[right]{$x(t)$};

        \draw[-](-2,0)--(-1.5,1)--(-1,0)--(-0.5,0)node[below]{$\displaystyle-\frac{\tau}{2}$}--(0,1)node[above right]{$T$}--(0.5,0)node[below]{$\displaystyle\frac{\tau}{2}$}--(1,0)--(1.5,1)--(2,0);
        \draw[dashed](-2.5,1)--(2.5,1);
    \end{tikzpicture}
\end{center}
Si calcolano i coefficienti tramite la definizione. Poiché è presente un solo triangolo nell'intervallo di intergrazione, si può restringere l'intervallo: 
\begin{gather*}
    c_k=\displaystyle\frac{1}{T}\int_{-\frac{T}{2}}^{\frac{T}{2}}\mbox{tri}\left(\frac{2t}{\tau}\right)e^{-i\frac{2\pi kt}{T}}dt=\frac{1}{T}\int_{-\frac{\tau}{2}}^{\frac{\tau}{2}}\left(1-\left|\frac{2t}{\tau}\right|\right)e^{-i\frac{2\pi kt}{T}}dt\\
    \displaystyle\frac{1}{T}\int_{-\frac{\tau}{2}}^{\frac{\tau}{2}}e^{-i\frac{2\pi kt}{T}}dt+\frac{1}{T}\int_{-\frac{\tau}{2}}^{\frac{\tau}{2}}-\left|\frac{2t}{\tau}\right|e^{-i\frac{2\pi kt}{T}}dt\\
    \displaystyle\frac{1}{T}\int_{-\frac{\tau}{2}}^{\frac{\tau}{2}}e^{-i\frac{2\pi kt}{T}}dt-\frac{1}{T}\left(\int_{-\frac{\tau}{2}}^{0}-\frac{2t}{\tau}e^{-i\frac{2\pi kt}{T}}dt+\int_{0}^{\frac{\tau}{2}}\frac{2t}{\tau}e^{-i\frac{2\pi kt}{T}}dt\right)
\end{gather*}
Il primo integrale corrisponde all'integrale ai coefficienti di una finestra, ed è stato già precedentemente calcolato, per cui si omette la risoluzione. Si considera la 
sostituzione $t=-t'$ nel secondo integrale:
\begin{gather*}
    \displaystyle\frac{\tau}{T}\mbox{sinc}\left(\frac{\tau k}{T}\right)-\frac{1}{T}\left(\int_{-\frac{-\tau}{2}}^{0}-\frac{2(-t)}{\tau}e^{-i\frac{2\pi k(-t)}{T}}d(-t)+\int_{0}^{\frac{\tau}{2}}\frac{2t}{\tau}e^{-i\frac{2\pi kt}{T}}dt\right)\\
    \displaystyle\frac{\tau}{T}\mbox{sinc}\left(\frac{\tau k}{T}\right)-\frac{1}{T}\left(-\int_{\frac{\tau}{2}}^{0}\frac{2t}{\tau}e^{i\frac{2\pi kt}{T}}dt+\int_{0}^{\frac{\tau}{2}}\frac{2t}{\tau}e^{-i\frac{2\pi kt}{T}}dt\right)\\
    \displaystyle\frac{\tau}{T}\mbox{sinc}\left(\frac{\tau k}{T}\right)-\frac{1}{T}\left(\int^{\frac{\tau}{2}}_{0}\frac{2t}{\tau}e^{i\frac{2\pi kt}{T}}dt+\int_{0}^{\frac{\tau}{2}}\frac{2t}{\tau}e^{-i\frac{2\pi kt}{T}}dt\right)\\
    \displaystyle\frac{\tau}{T}\mbox{sinc}\left(\frac{\tau k}{T}\right)-\frac{1}{T}\int^{\frac{\tau}{2}}_{0}\frac{4t}{\tau}\frac{e^{i\frac{2\pi kt}{T}}+e^{-i\frac{2\pi kt}{T}}}{2}dt\\
    \displaystyle\frac{\tau}{T}\mbox{sinc}\left(\frac{\tau k}{T}\right)-\frac{4}{\tau T}\int_{0}^{\frac{\tau}{2}}t\cos\left(\frac{2\pi kt}{T}\right)dt
\end{gather*}
Quest'ultimo integrale così ottenuto si risolve mediante integrazione per parti:
\begin{gather*}
    \displaystyle\frac{\tau}{T}\mbox{sinc}\left(\frac{\tau k}{T}\right)-\frac{4}{\tau T}\left[\frac{T}{2\pi k}t\sin\left(\frac{2\pi kt}{T}\right)\right]^{\frac{\tau}{2}}_0
    +\frac{4}{\tau T}\int_{0}^{\frac{\tau}{2}}\frac{T}{2\pi k}\sin\left(\frac{2\pi kt}{T}\right)dt\\
    \displaystyle\frac{\tau}{T}\mbox{sinc}\left(\frac{\tau k}{T}\right)-\frac{\tau}{T}\frac{T}{\pi k\tau}\sin\left(\frac{\pi k\tau}{T}\right)+0+
    \frac{4}{\tau T}\left[-\left(\frac{T}{2\pi k}\right)^2\cos\left(\frac{2\pi kt}{T}\right)\right]^{\frac{\tau}{2}}_0\\
    \displaystyle\frac{\tau}{T}\mbox{sinc}\left(\frac{\tau k}{T}\right)-\displaystyle\frac{\tau}{T}\mbox{sinc}\left(\frac{\tau k}{T}\right)-
    \frac{T}{\pi^2k^2\tau}\left[\cos\left(\frac{\pi k\tau}{T}\right)-1\right]\\
    \displaystyle\frac{2T}{\pi^2k^2\tau}\sin^2\left(\frac{\pi kt}{2T}\right)=
    \frac{\tau}{2T}\left[\frac{2T}{\pi k\tau}\sin\left(\frac{\pi kt}{2T}\right)\right]\left[\frac{2T}{\pi k\tau}\sin\left(\frac{\pi kt}{2T}\right)\right]\\
    \displaystyle\frac{\tau}{2T}\left[\mbox{sinc}\left(\frac{k\tau}{2T}\right)\right]\left[\mbox{sinc}\left(\frac{k\tau}{2T}\right)\right]
\end{gather*}
\begin{gather}
    c_k=\displaystyle\frac{\tau}{2T}\mbox{sinc}^2\left(\frac{k\tau }{2T}\right)
\end{gather}
I coefficienti di un treno di triangoli risultano essere dei seni cardinali quadrati. 

\clearpage

\section{Trasformata di Fourier}
Calcolare l'energia del dato segnale:
\begin{gather*}
    x(t)=\displaystyle\frac{\sin(\pi t)}{\pi t}\cos(2\pi \alpha t)=\frac{1}{2}\mbox{sinc}(t)(e^{2i\pi\alpha t}+e^{-2i\pi\alpha t})\\
\end{gather*}
Per il teroema della traslazioen in frequenza, la trasformata del segnale $x$ risulta essere:
\begin{gather*}
    X(f)=\displaystyle\frac{1}{2}\mbox{rect}(f+\alpha)+\frac{1}{2}\mbox{rect}(f-\alpha)
\end{gather*}
La sua energia si calcola come:
\begin{gather*}
    E_x=\displaystyle\int_{-\infty}^{+\infty}|X(f)|^2df\\
    \displaystyle\frac{1}{4}\left(\int_{-\infty}^{+\infty}\left[\mbox{rect}(f+\alpha)+\mbox{rect}(f-\alpha)\right]^2df\right)\\
    E_x=\begin{cases}
        \displaystyle\frac{1}{4}\int_{-\alpha-1/2}^{-\alpha+1/2}df+\frac{1}{4}\int_{\alpha-1/2}^{\alpha+1/2}df& \alpha\geq1/2\\
        \displaystyle\frac{1}{4}\int_{-\alpha-1/2}^{\alpha-1/2}df+\frac{1}{2}\int_{\alpha-1/2}^{-\alpha+1/2}df+\frac{1}{4}\int_{-\alpha+1/2}^{\alpha+1/2}df&\alpha<1/2
    \end{cases}
\end{gather*}
\begin{gather}
    E_x=\begin{cases}
        \displaystyle\frac{1}{2}&\alpha\geq1/2\\
        \displaystyle\frac{2\alpha}{4}+\frac{2\alpha}{4}+1-2\alpha=1-\alpha&\alpha<1/2
    \end{cases}
\end{gather}


Calcolare la trasformata del segnale $x$, tramite la proprietà di ribaltamento e di scala: 
\begin{gather*}
    x(t)=e^{3t}u(-t)\\
    x_1(t)=e^{-3t}u(t)\\
    X_1(f)=X(-f)=\displaystyle\frac{1}{3+2i\pi f}
\end{gather*}
\begin{gather}
    X(f)=\displaystyle\frac{1}{3-2i\pi f}
\end{gather}


Calcolare la trasformata del segnale $x$, per la proprietà della traslazione del tempo:
\begin{gather*}
    x(t)=e^{-2t+4}u(t-2)=e^{-2*(t-2)}u(t-2)\\
    x_1(t)=e^{-2t}u(t)\\
    X_1(f)=\displaystyle\frac{1}{2+2i\pi f}
\end{gather*}
\begin{gather}
    X(f)=\displaystyle\frac{1}{2+2i\pi f}e^{-4i\pi f}
\end{gather}

Calcolare la trasformata del segnale $x$, si applica la proprietà di modulazione: 
\begin{gather*}
    x(t)=e^{-t/2}\cos(100\pi t)u(t)\\
    x_1(t)=e^{-t/2}u(t)\\
    X_1(f)=\displaystyle\frac{1}{1/2+2i\pi f}\\
    X(f)=\displaystyle\frac{1}{2}\left(X_1(f-50)+X_1(f+50)\right)\\
    X(f)=\displaystyle\frac{1}{2}\frac{1}{1/2+2i\pi (f-50)}+\frac{1}{2}\frac{1}{1/2+2i\pi (f+50)}
\end{gather*}
\begin{gather}
    X(f)=\displaystyle\frac{1}{1+4i\pi (f-50)}+\frac{1}{1+4i\pi (f+50)}
\end{gather}


Calcolare l'autoconvoluzione del segnale $x$, passando per la sua trasformata: 
\begin{gather*}
    x(t)=A\,\mbox{rect}\displaystyle\left(\frac{2(t-3T/4)}{T}\right)-A\,\mbox{rect}\displaystyle\left(\frac{2(t-T/4)}{T}\right)\\
    X(f)=\displaystyle\frac{AT}{2}\,\mbox{sinc}\left(\frac{Tf}{2}\right)e^{-3i\pi fT/2}-\displaystyle\frac{AT}{2}\,\mbox{sinc}\left(\frac{T}{2}f\right)e^{-i\pi fT/2}\\
    X(f)=\displaystyle\frac{AT}{2}\,\mbox{sinc}\left(\frac{Tf}{2}\right)\left(e^{-3i\pi fT/2}-e^{-i\pi fT/2}\right)\\
    X^2(f)=\displaystyle\frac{A^2T^2}{4}\,\mbox{sinc}^2\left(\frac{Tf}{2}\right)\left(e^{-3i\pi fT}+e^{-i\pi fT}-2e^{-2i\pi fT}\right)\\
    x(t)*x(t)=\displaystyle\int_{-\infty}^{+\infty}X^2(f)e^{2i\pi ft}df\\
    \displaystyle\int_{-\infty}^{+\infty}\left[\frac{A^2T^2}{4}\,\mbox{sinc}^2\left(\frac{Tf}{2}\right)\left(e^{-3i\pi fT}+e^{-i\pi fT}-2e^{-2i\pi fT}\right)\right]e^{2i\pi ft}df
\end{gather*}
Un prodotto nel domino delle frequenze corrisponde ad una convoluzione nel domino del tempo, per cui si può esprimere l'autoconvoluzione come la convoluzione tra un segnale 
triangolo e una combinazione lineare di impulsi:
\begin{gather*}
    X_1(f)=\displaystyle\frac{A^2T^2}{4}\,\mbox{sinc}^2\left(\frac{Tf}{2}\right)\\
    x_1(t)=\displaystyle\frac{A^2T^2}{4}\frac{2}{T}\,\mbox{tri}\left(\frac{2t}{T}\right)\\
    X_2(f)=e^{-3i\pi fT}+e^{-i\pi fT}-2e^{-2i\pi fT}\\
    x_2(t)=\delta(t-3T/2)+\delta(t-T/2)-2\delta(t-T)\\
    x(t)*x(t)=\displaystyle\int_{-\infty}^{+\infty}X(f)^2e^{2i\pi ft}df=\int_{-\infty}^{+\infty}X_1(f)X_2(f)e^{2i\pi ft}df=x_1(t)*x_2(t)\\
    \displaystyle\frac{A^2T^2}{4}\frac{2}{T}\,\mbox{tri}\left(\frac{2t}{T}\right)*\left[\delta(t-3T/2)+\delta(t-T/2)-2\delta(t-T)\right]
\end{gather*}
\begin{gather}
    x(t)*x(t)=\displaystyle\frac{A^2T^2}{4}\left(\mbox{tri}\left(\frac{2}{T}(t-3T/2)\right)+\mbox{tri}\left(\frac{2}{T}(t-T/2)\right)-2\,\mbox{tri}\left(\frac{2}{T}(t-T)\right)\right)
\end{gather}
%% grafo segnale convoluzione 

Calcolare la trasformata del segnale $x$: 
\begin{gather*}
    x(t)=\mbox{rect}(3t-1/2)=\mbox{rect}(3(t-1/6))
\end{gather*}
Per la proprietà di scala e di traslazione nel tempo:
\begin{gather}
    X(f)=\displaystyle\frac{1}{3}\mbox{sinc}\left(\frac{f}{3}\right)e^{-i\pi f/3}
\end{gather}


Calcolare la trasformata del segnale $x$, esprimibile come un segnale triangolo moltiplicato per un gradino, oppure una retta moltiplicata per una finestra, si sceglie 
quest'ulima rappresentazione per facilitare il calcolo:
\begin{gather*}
    x(t)=A\,\displaystyle\mbox{tri}\left(\frac{t+T/2}{T}\right)u(t+T/2)\lor x(t)=A\left(\frac{1}{2}-\frac{t}{T}\right)\mbox{rect}\left(\frac{t}{T}\right)\\
    \displaystyle\frac{A}{2}\mbox{rect}\left(\frac{t}{T}\right)-\frac{A}{T}t\,\mbox{rect}\left(\frac{t}{T}\right)
\end{gather*}
%% grafo segnale
Per la prorpietà duale alla derivazione, si calcola il secondo componente: 
\begin{gather*}
    x_1(t)=-2i\pi tx_2(t)\\
    X_1(f)=\displaystyle\frac{dX_2}{df}\\
    x_1(t)=-\displaystyle\frac{A}{T}t\,\mbox{rect}\left(\frac{t}{T}\right)\\
    x_2(t)=\displaystyle\frac{1}{2i\pi}\frac{A}{T}\mbox{rect}\left(\frac{t}{T}\right)\\
    X_2(f)=\displaystyle\frac{1}{2i\pi}\frac{A}{T}T\,\mbox{sinc}(Tf)=\frac{A}{\pi T}\frac{\sin(\pi Tf)}{f}\\
    X_1(f)=\displaystyle\frac{dX_2}{df}=\frac{1}{2i\pi}\frac{A}{\pi T}\frac{\cos(\pi Tf)\pi Tf-\sin(\pi Tf)}{f^2}\\
\end{gather*}
Per cui la trasformata complessiva è:
\begin{gather*}
    X(f)=\displaystyle\frac{AT}{2}\mbox{sinc}(Tf)+\frac{A}{\pi Tf^2(2\pi i)}\left[\cos(\pi Tf)\pi Tf-\sin (\pi Tf)\right]\\
    \displaystyle\frac{AT}{2\pi Tf}\frac{i}{i}\sin(\pi Tf)+\frac{A}{2i\pi f}\cos(\pi Tf)-\frac{A}{2i\pi f}\frac{\sin(\pi Tf)}{\pi Tf}\\
    \displaystyle\frac{A}{2i\pi f}\frac{i}{i}\left[\cos(\pi Tf)+i\sin(\pi Tf)\right]-\frac{A}{2i\pi f}\frac{i}{i}\mbox{sinc}( Tf)
\end{gather*}
\begin{gather}
    X(f)=\displaystyle-\frac{Ai}{2\pi f}\left[e^{i\pi fT}-\mbox{sinc}(fT)\right]
\end{gather}


Calcolare ora questa trasformata usando la proprietà alla derivazione:
\begin{gather*}
    x_1(t)=\displaystyle\frac{A}{2}\mbox{rect}\left(\frac{t}{T}\right)-\frac{A}{T}t\,\mbox{rect}\left(\frac{t}{T}\right)\\
    \displaystyle\frac{dx_1}{dt}=x_2(t)
\end{gather*}
Il segnale $x_1$ presente una discontinuità in $-T/2$, e rimane costante fino al valore $T/2$:
\begin{gather*}
    x_2(t)=A\delta(t+T/2)-\displaystyle\frac{A}{T}\mbox{rect}\left(\frac{t}{T}\right)\\
    X_2(f)=Ae^{i\pi fT}-\displaystyle\frac{A}{T}T\mbox{sinc}(Tf)\\
    X_2(f)=2i\pi fX_1(f)\\
    X_1(f)=\displaystyle\frac{A}{2i\pi f}\frac{i}{i}\left[e^{i\pi fT}-\mbox{sinc}(Tf)\right]
\end{gather*}
\begin{gather}
    X_1(f)=\displaystyle-\frac{Ai}{2\pi f}\left[e^{i\pi fT}-\mbox{sinc}(Tf)\right]
\end{gather}


Calcolare l'antitrasformata di $X_1$:
\begin{gather*}
    X_1(f)=\displaystyle\frac{1}{2}\left[X(f-f_0)+X(f+f_0)\right]\cos(2\pi f/f_0)
\end{gather*}
Per l'inverso della proprietà di modulazione, e per la proprietà di convoluzione:
\begin{gather*}
    x_1(t)=\left[x(t)\cos(2\pi tf_0)\right]*\left[\displaystyle\frac{1}{2}\delta(t+1/f_0)+\frac{1}{2}\delta(t-1/f_0)\right]\\
    x_1(t)=\displaystyle\frac{1}{2}x\left(t+\frac{1}{f_0}\right)\cos\left[2\pi f_0\left(t+\frac{1}{f_0}\right)\right]+\frac{1}{2}x\left(t-\frac{1}{f_0}\right)\cos\left[2\pi f_0\left(t-\frac{1}{f_0}\right)\right]
\end{gather*}
Poiché il coseno è traslato di un periodo $1/f_0$: 
\begin{gather}
    x_1(t)=\left[\displaystyle\frac{1}{2}x\left(t+\frac{1}{f_0}\right)+\frac{1}{2}x\left(t-\frac{1}{f_0}\right)\right]\cos(2\pi tf_0)
\end{gather}


Calcolare la trasformata del segnale $x$:
\begin{gather*}
    x(t)=2\,\mbox{rect}(2t)-\mbox{rect}(t)=\mbox{rect}(2t)-\mbox{rect}\left[\displaystyle4\left(t-\frac{3}{2}\right)\right]-\mbox{rect}\left[\displaystyle4\left(t-\frac{3}{8}\right)\right]\\
    X(f)=\displaystyle\frac{1}{2}\mbox{sinc}\left(\frac{f}{2}\right)-\frac{1}{4}\mbox{sinc}\left(\frac{f}{4}\right)e^{-3i\pi f/4}-\frac{1}{4}\mbox{sinc}\left(\frac{f}{4}\right)e^{-3i\pi f}
\end{gather*}
\begin{gather}
    X(f)=\displaystyle\frac{1}{2}\left[\mbox{sinc}\left(\frac{f}{2}\right)-\mbox{sinc}\left(\frac{f}{4}\right)\cos\left(\frac{3\pi f}{4}\right)\right]
\end{gather}


Calcolare la trasformata del seguente segnale onda quadra di base $\tau$ e di periodo $T$:
\begin{gather*}
    x(t)=\displaystyle\sum_{k=-\infty}^{+\infty}\mbox{rect}\left(\frac{t-kT}{\tau}\right)
\end{gather*}
\begin{gather}
    X(f)=\displaystyle\sum_{k=-\infty}^{+\infty}\frac{\tau}{T}\mbox{sinc}\left(\frac{\tau k}{T}\right)\delta\left(f-\frac{k}{T}\right)\\
    c_k=\displaystyle\frac{\tau}{T}\mbox{sinc}\left(\frac{\tau k}{T}\right)
\end{gather}

Calcolare la trasformata del treno campionatore, o segnale pettine:
\begin{gather*}
    \pi(t)=\displaystyle\sum_{k=-\infty}^{+\infty}\delta(t-kT)\\
    c_k=\displaystyle\frac{1}{k}\\
    \pi(t)=\displaystyle\sum_{k=-\infty}^{+\infty}\frac{1}{T}e^{2i\pi kt/T}
\end{gather*}
\begin{gather}
    \Pi(f)=\displaystyle\sum_{k=-\infty}^{+\infty}\frac{1}{T}\delta\left(f-\frac{k}{T}\right)
\end{gather}
Questo è uno dei pochi segnali oltre alla gaussiana che si autotrasforma. 


Calcolare la trasformata di un treno di trapezi di base maggiore $2T$, di base minore $T$ e di altezza $T$, di periodo $\overline T=5T/2$:
%% grafo segnale 
\begin{gather*}
    X(f)=\displaystyle\sum_{k=-\infty}^{+\infty}c_k\delta\left(f-\frac{k}{\overline T}\right)
\end{gather*}
Per calcolare i coefficienti della serie, si può esprimere un singolo trapezio come l'autoconvoluzione di un segnale finestra. Inoltre è sufficiente calcoalre la 
trasformata di un singolo trapezio per ottenere i coefficienti:
\begin{gather*}
    x_1(t)=2\displaystyle\mbox{rect}\left(\frac{2t}{3T}\right)\\
    x_2(t)=\displaystyle\mbox{rect}\left(\frac{2t}{T}\right)\\
    c_k=\displaystyle\frac{1}{\overline T}\int_{-\overline T/2}^{\overline T/2}x_1(t)*x_2(t)e^{-2i\pi kt/\overline T}dt\\
    c_k=\displaystyle\frac{1}{\overline T}X_1\left(\frac{k}{ T}\right)X_2\left(\frac{k}{ T}\right)\\
    c_k=\displaystyle\frac{2}{\overline T}\left[\frac{3T}{2}\mbox{sinc}\left(\frac{3T}{2}\frac{k}{ T}\right)\frac{T}{2}\mbox{sinc}\left(\frac{T}{2}\frac{k}{ T}\right)\right]\\
    c_k=\displaystyle\frac{6T^2}{4}\frac{2}{5T}\mbox{sinc}\left(\frac{3T}{2}\frac{k}{ T}\right)\mbox{sinc}\left(\frac{T}{2}\frac{k}{ T}\right)\\
    c_k=\displaystyle\frac{3T}{5}\mbox{sinc}\left(\frac{3k}{2}\right)\mbox{sinc}\left(\frac{k}{2}\right)
\end{gather*}
\begin{gather}
    X(f)=\displaystyle\frac{2}{5T}\sum_{n=-\infty}^{+\infty}\left[\frac{3T}{5}\mbox{sinc}\left(\frac{3k}{2}\right)\mbox{sinc}\left(\frac{k}{2}\right)\right]\delta\left(f-\frac{2k}{5T}\right)
\end{gather}

Calcolare la seguente convoluzione:
\begin{gather*}
    x(t)=\left[\cos(2\pi f_0t)\sin(2\pi f_0t)\right]*\displaystyle\frac{\sin(3\pi f_0t)}{\pi t}\\
    \left[\cos(2\pi f_0t)\sin(2\pi f_0t)\right]*\displaystyle3f_0\frac{\sin(3\pi f_0t)}{3\pi f_0t}\\
    \left[\cos(2\pi f_0t)\sin(2\pi f_0t)\right]*3f_0\mbox{sinc}(3f_0 t)
\end{gather*}
Si applica la formula di duplicazione del seno, e si passa per il dominio della frequenza:
\begin{gather*}
    x(t)=\displaystyle\frac{3f_0}{2}\sin(4\pi f_0t)*\mbox{sinc}(3f_0t)\\
    X(f)=\displaystyle\frac{1}{4i}\left[\delta(f-2f_0)-\delta(f+2f_0)\right]\frac{3f_0}{3f_0}\mbox{rect}\left(\frac{f}{3f_0}\right)\\
    \displaystyle\frac{1}{4i}\left[\mbox{rect}\left(-\frac{2f_0}{3f_0}\right)\delta(f-2f_0)-\mbox{rect}\left(\frac{2f_0}{3f_0}\right)\delta(f+2f_0)\right]\\
    \displaystyle\frac{1}{4i}\mbox{rect}\left(\frac{2}{3}\right)\left[\delta(f-2f_0)-\delta(f+2f_0)\right]
\end{gather*}
Poiché il valore $2/3$ è esterno all'intervallo di valori non nulli del segnale finestra $[-1/2,1/2]$, la trasformata assume valore nullo. 
\begin{gather}
    x(t)=0
\end{gather}


Calcolare il seganle $x$, passando per il dominio della frequenza:
\begin{gather*}
    x(t)=\cos(2\pi f_0t)\left[\sin(2\pi f_0t)*\displaystyle\frac{\sin(3\pi f_0t)}{\pi t}\right]\\
    x_1(t)=\sin(2\pi f_0t)*\displaystyle\frac{\sin(3\pi f_0t)}{\pi t}\\
    X_1(f)=\displaystyle\frac{1}{2i}\left[\delta(f-f_0)-\delta(f+f_0)\right]\cdot \frac{3f_0}{3f_0}\mbox{rect}\left(\frac{f}{3f_0}\right)\\
    \displaystyle\frac{1}{2i}\delta(f-f_0)\mbox{rect}\left(-\frac{f_0}{3f_0}\right)-\frac{1}{2i}\delta(f+f_0)\mbox{rect}\left(\frac{f_0}{3f_0}\right)\\
    \displaystyle\frac{1}{2i}\mbox{rect}\left(\frac{1}{3}\right)\left[\delta(f-f_0)-\delta(f+f_0)\right]\\
    X_1(f)=\displaystyle\frac{1}{2i}\left[\delta(f-f_0)-\delta(f+f_0)\right]\\
    x_1(t)=\sin(2\pi f_0t)
\end{gather*}
\begin{gather}
    x(t)=\cos(2\pi f_0t)\sin(2\pi f_0t)=\displaystyle\frac{\sin(4\pi f_0t)}{2}
\end{gather}
Per la formula di duplicazione del seno. 

Dato il segnale $x$, calcolare il periodo, i coefficienti di Fourier, la potenza, e la sua trasformata:
\begin{gather*}
    x(t)=2e^{2i\pi t/T_0}-e^{-i\pi t/2T_0}=2e^{2i\pi t/T_0}-e^{-2i\pi t/4T_0}
\end{gather*}
Il periodo di un segnale combinazione lineare di altri segnali è il minimo comune multiplo tra i periodi dei segnali considerati:
\begin{gather}
    T=4T_0
\end{gather}
Si calcola la potenza del segnale:
\begin{gather*}
    P_x=\displaystyle\frac{1}{4T_0}\int_{-2T_0}^{2T_0}\left|2e^{2i\pi t/T_0}-e^{-2i\pi t/4T_0}\right|^2dt\\
    \displaystyle\frac{1}{4T_0}\int_{-2T_0}^{2T_0}\left(4+1+4\cos\left[\frac{2\pi t}{T_0}\left(1+\frac{1}{4}\right)\right]\right)\\
    \displaystyle\frac{20T_0}{4T_0}+\frac{2T_0}{5\pi }\cancelto{0}{\left[\sin\left(\frac{5\pi t}{2T_0}\right)\right]_{-2T_0}^{2T_0}}
\end{gather*}
\begin{gather}
    P_x=5
\end{gather}
Si determinano i coefficienti di Fourier:
\begin{gather*}
    c_k=\displaystyle\frac{1}{4T_0}\int_{-2T_0}^{2T_0}\left(2e^{2i\pi t/T_0}-e^{-2i\pi t/4T_0}\right)e^{-2i\pi kt/4T_0}dt\\
    c_k=\displaystyle\frac{2}{4T_0}\int_{-2T_0}^{2T_0}e^{2i\pi t(1-k/4)/T_0}dt+\frac{1}{4T_0}\int_{-2T_0}^{2T_0}2e^{-2i\pi t(1+k)/4T_0}dt
\end{gather*}
\begin{gather}
    c_k=\begin{cases}
        0 &\forall k\neq4\lor-1\\
        \displaystyle\frac{1}{4T_0}& k=-1\\
        \displaystyle\frac{1}{2T_0}&k=4
    \end{cases}
\end{gather}
Si ottiene lo stesso risultato, più velocemente, per confronto diretto. La sua trasformata si ottiente, sia trasformando la sua serie di Fourier, sia attuando l'integrale 
di trasformazione:
\begin{gather}
    X(f)=\displaystyle\frac{1}{4T_0}\delta\left(f+\frac{1}{4T_0}\right)+\frac{1}{2T_0}\delta\left(f-\frac{1}{T_0}\right)
\end{gather}



Caclolare la trasformata del segnale $x$:
\begin{gather*}
    x(t)=\cos\left[\displaystyle\frac{\pi(t-T)}{2T}\right]\cos\left(\frac{2\pi t}{T}\right)=\cos\left[\displaystyle\frac{2\pi(t-T)}{4T}\right]\cos\left(\frac{2\pi t}{T}\right)
\end{gather*}
Questo segnale ha periodo $4T$. Espandendo l'argomento del primo coseno si ottiene:
\begin{gather*}
    x(t)=\displaystyle\cos\left[\frac{2\pi t}{4T}-\frac{\pi}{2}\right]\cos\left(\frac{2\pi t}{T}\right)\\
    \sin\left(\displaystyle\frac{2\pi t}{4T}\right)\cos\left(\frac{2\pi t}{T}\right)
\end{gather*}
Si rappresenta il segnale in forma esponenziale:
\begin{gather*}
    x(t)=\displaystyle\frac{1}{2i}\left(e^{2i\pi t/4T}-e^{-2i\pi t/4T}\right)\cdot\displaystyle\frac{1}{2}\left(e^{2i\pi t/T}+e^{-2i\pi t/T}\right)\\
    \displaystyle\frac{1}{4i}\left[e^{\frac{2\pi t}{T}\left(\frac{1}{4}+1\right)}-e^{\frac{2\pi t}{T}\left(1-\frac{1}{4}\right)}+e^{\frac{2\pi t}{T}\left(\frac{1}{4}-1\right)}-e^{-\frac{2\pi t}{T}\left(1+\frac{1}{4}\right)}\right]\\
    \displaystyle\frac{1}{4i}\left[e^{\frac{2\pi t}{T}\frac{5}{4}}-e^{\frac{2\pi t}{T}\frac{3}{4}}+e^{-\frac{2\pi t}{T}\frac{3}{4}}-e^{-\frac{2\pi t}{T}\frac{5}{4}}\right]\\
    X(f)=\displaystyle\frac{1}{4i}\left[\delta\left(f+\frac{5}{4T}\right)-\delta\left(f+\frac{3}{4T}\right)+\delta\left(f-\frac{3}{4T}\right)-\delta\left(f-\frac{5}{4T}\right)\right]
\end{gather*}
\begin{gather}
    X(f)=\displaystyle\frac{1}{4i}\left[\delta\left(f+\frac{5}{4T}\right)-\delta\left(f-\frac{5}{4T}\right)\right]-\frac{1}{4i}\left[\delta\left(f+\frac{3}{4T}\right)-\delta\left(f-\frac{3}{4T}\right)\right]
\end{gather}
Per cui il segnale orginale corrisponde ad una differenza di due seni, considerando operazioni trigonometriche del segnale originario. Alternametivamente, per la 
proprietà della convoluzione, si può calcolare come convoluzione tra le due trasformate dei segnali moltiplicati:
\begin{gather*}
    x(t)=\cos\left[\displaystyle\frac{\pi(t-T)}{2T}\right]\cos\left(\frac{2\pi t}{T}\right)=x_1(t)\cdot x_2(t)\\
    X(f)=X_1(f)*X_2(f)\\
    X_1(f)=\displaystyle\frac{1}{2}\left[\delta\left(f+\frac{1}{4T}\right)+\delta\left(f-\frac{1}{4T}\right)\right]e^{-2i\pi fT}\\
    \displaystyle\frac{1}{2}e^{-2i\pi/4}\delta\left(f+\frac{1}{4T}\right)+\frac{1}{2}e^{2i\pi /4}\delta\left(f-\frac{1}{4T}\right)\\
    \displaystyle-\frac{i}{2}\delta\left(f+\frac{1}{4T}\right)+\frac{i}{2}\delta\left(f-\frac{1}{4T}\right)\\
    X_2(f)=\displaystyle\frac{1}{2}\left[\delta\left(f+\frac{1}{T}\right)+\delta\left(f-\frac{1}{T}\right)\right]
\end{gather*}
Si attua quindi una convoluzione nel dominio della frequenza per ottenere la trasformata del segnale orginale:
\begin{gather*}
    X(f)=\displaystyle\frac{i}{2}\left[\delta\left(f-\frac{1}{4T}\right)-\delta\left(f+\frac{1}{4T}\right)\right]*\frac{1}{2}\left[\delta\left(f+\frac{1}{T}\right)+\delta\left(f-\frac{1}{T}\right)\right]
\end{gather*}
\begin{gather}
    X(f)=\displaystyle\frac{i}{4}\left[-\delta\left(f+\frac{5}{4T}\right)+\delta\left(f+\frac{3}{4T}\right)-\delta\left(f-\frac{3}{4T}\right)+\delta\left(f-\frac{5}{4T}\right)\right]
\end{gather}

Dato un filtro di risposta impulsiva $h(t)$, calcolare la sua funzione di trasferimento, e la la risposta ad un gradino:
\begin{gather*}
    h(t)=\delta(t-2T)+\delta(t)-\delta(t-T)
\end{gather*}
\begin{gather}
    H(f)=e^{-4i\pi fT}+1-e^{-2i\pi fT}\\
    h(t)*u(t)=\delta(t-2T)+\delta(t)-\delta(t-T)*u(t)=\delta(t-2T)u(t-2T)+\delta(t)u(t)-\delta(t-T)u(t-T)
\end{gather}
\begin{center}
    \begin{tikzpicture}[scale=2]
        \draw[->](-0.5,0)--(2,0)node[above]{$t$};
        \draw[->](0,-0.25)--(0,1.5)node[right]{$y(t)$};
        \draw[-](0,1)node[left]{$1$}--(0.5,1)--(0.5,0)node[below]{$T$}--(1,0)node[below]{$2T$}--(1,1)--(2,1);
    \end{tikzpicture}
\end{center}

\clearpage

\section{Trasformata di Fourier Tempo Discreto}
Data una sequenza di valore costante:
\begin{gather*}
    x[n]=1
\end{gather*}
Si determina la trasformata della sequenza tramite la definizione:
\begin{gather}
    X_C(f)=\displaystyle\sum_{n=-\infty}^{+\infty}e^{-2i\pi nfT}
\end{gather}
Altrimenti si considera il segnale da cui si è campionata la sequenza:
\begin{gather*}
    x(t)=1\\
    X(f)=\delta(f)
\end{gather*}
Per cui la trasformata del segnale campionato si può esprimere come un treno di delta:
\begin{gather}
    X_C(f)=\displaystyle\frac{1}{T}\sum_{n=-\infty}^{+\infty}\delta\left(f-\frac{n}{T}\right)
\end{gather}
Poiché la sequenza considerata coincide con il treno campionatore $x[n]=\pi(t)$. 

Calcolare la trasformata della sequenza $x[n]$:
\begin{gather*}
    x[n]=\left[\displaystyle\frac{\sin(2\pi f_CnT)}{\pi n}\right]^2=4f_C^2T^2\mbox{sinc}^2(2f_CnT)
\end{gather*}
Questa sequenza si può considerare estratta dal segnale tempo continuo $x$:
\begin{gather*}
    x(t)=4f_C^2T^2\mbox{sinc}(2f_Ct)\\
    X(f)=\displaystyle 2f_CT^2\mbox{tri}\left(\frac{f}{2f_C}\right)
\end{gather*}

Per cui si esprime la trasformata della sequenza come:
\begin{gather}
    X_C(f)=\displaystyle\frac{1}{T}\sum_{n=-\infty}^{+\infty}X\left(f-\frac{n}{T}\right)=2f_CT\sum_{n=-\infty}^{+\infty}\mbox{tri}\left(\frac{f-\frac{n}{T}}{2f_C}\right)
\end{gather}
Se $T$ il passo di campionamento non è abbastanza grande, i triangoli si sovrappongono tra di loro. 



Calcolare la convoluzione $y$ tempo discreta tra $x$ e $h$, esponenziali unilateri: 
\begin{gather*}
    y[n]=x[n]*h[n]\\
    x[n]=a^nu[n]\\
    h[n]=b^nu[n]\\
    y[n]=\displaystyle\sum_{k=-\infty}^{+\infty}x[k]h[n-k]=\sum_{k=-\infty}^{+\infty}a^ku[k]b^{n-k}u[n-k]=\sum_{k=0}^na^kb^{n-k}\\
    \displaystyle b^n\sum_{k=0}^n\left(\frac{a}{b}\right)^k=b^n\frac{1-\left(\frac{a}{b}\right)^{n+1}}{1-\frac{a}{b}}
\end{gather*}
\begin{gather}
    y[n]=\displaystyle\frac{b^{n+1}-a^{n+1}}{b-a}u[n]
\end{gather}
Si può calcolare anlogamente nel dominio della frequenza, poi attuando un antitrasformata: 
\begin{gather*}
    Y_C(f)=X_C(f)H_C(f)\\
    X_C(f)=\displaystyle\sum_{n=0}^{+\infty}a^ne^{-2i\pi nfT}=\sum_{n=0}^{+\infty}\left(ae^{-2i\pi fT}\right)^n=\frac{1}{1-ae^{-2i\pi fT}}\\
    H_C(f)=\displaystyle\frac{1}{1-be^{-2i\pi fT}}
\end{gather*}
Poiché le due sommatorie diventano delle serie geometriche. Si considera ora il loro prodotto:
\begin{gather*}
    Y_C(f)=X_C(f)H_C(f)=\displaystyle\frac{1}{1-ae^{-2i\pi fT}}\frac{1}{1-be^{-2i\pi fT}}
\end{gather*}
Si può decomporre in fratti semplici come:
\begin{gather*}
    Y_C(f)=\displaystyle\frac{A}{1-ae^{-2i\pi fT}}+\frac{B}{1-be^{-2i\pi fT}}\\
    \begin{cases}
        A+B=1\\
        Ab+Ba=0
    \end{cases}\to
    \begin{cases}
        A=1-B\\
        b-Bb+Ba=0
    \end{cases}\to\begin{cases}
        A=\displaystyle\frac{a}{a-b}\\
        B=\displaystyle\frac{b}{b-a}
    \end{cases}
\end{gather*}
Per cui il segnale convoluzione si esprime come:
\begin{gather*}
    Y_C(f)=\displaystyle\frac{a}{a-b}\frac{1}{1-ae^{-2i\pi fT}}+\frac{b}{b-a}\frac{1}{1-be^{-2i\pi fT}}
\end{gather*}
\begin{gather}
    y[n]=\displaystyle\frac{a}{a-b}a^nu[n]+\displaystyle\frac{b}{b-a}b^nu[n]
\end{gather}


Calcolare la trasforamta di Fourier della sequenza $x$:
\begin{gather*}
    x[n]=\displaystyle\frac{\sin(\pi n/3)}{\pi n}\cos\left(\frac{2\pi n}{6}\right)
\end{gather*}
Si considera $T=1$, e si esprime il segnale da cui è stata campionata la sequenza:
\begin{gather*}
    x(t)=\frac{1}{3}\mbox{sinc}\left(\frac{t}{3}\right)\cos\left(\frac{2\pi t}{6}\right)
\end{gather*}
Si determina la trasformata di questo segnale tempo continuo:
\begin{gather*}
    X(f)=\displaystyle\mbox{rect}(3f)*\frac{1}{2}\left[\delta\left(f-\frac{1}{6}\right)+delta\left(f+\frac{1}{6}\right)\right]\\
    X(f)=\displaystyle\frac{1}{2}\mbox{rect}\left(3f-\frac{1}{2}\right)+\frac{1}{2}\mbox{rect}\left(3f+\frac{1}{2}\right)
\end{gather*}
\begin{gather}
    X_C(f)=\displaystyle\frac{1}{T}\sum_{n=-\infty}^{+\infty}X\left(f-\frac{n}{T}\right)=\frac{1}{T}\sum_{n=-\infty}^{+\infty}\left[\frac{1}{2}\mbox{rect}\left(3f-\frac{1}{2}-3n\right)+\frac{1}{2}\mbox{rect}\left(3f+\frac{1}{2}-3n\right)\right]
\end{gather}



Calcolare la trasformata di Fourier della sequenza $x$:
\begin{gather*}
    x[n]=\left[\displaystyle\frac{\sin(\pi n/3)}{\pi n}\right]^2=\frac{1}{9}\mbox{sinc}^2\left(\frac{n}{3}\right)\\
    T=1\\
    x(t)=\displaystyle\frac{1}{9}\mbox{sinc}^2\left(\frac{t}{3}\right)\\
    X(f)=\displaystyle \frac{1}{3}\mbox{tri}(3f)
\end{gather*}
\begin{gather}
    X_C(f)=\displaystyle\frac{1}{3}\sum_{n=-\infty}^{+\infty}\mbox{tri}(3f-n)
\end{gather}

\clearpage

\section{Campionamento}

Dato un segnale coseno di frequenza $f_0$ si determina la frequenza di Nyquist:
\begin{gather*}
    x(t)=\cos(2\pi f_0t)\\
    X(f)=\displaystyle\frac{1}{2}\left[\delta(f-f_0)+\delta(f+f_0)\right]
\end{gather*}
Per cui la frequenza massima corrisponde a $f_0$, per cui la frequenza di Nyquist si ottiene come:
\begin{gather}
    f_C=2f_0
\end{gather}
Il segnale campionato corrisponde a:
\begin{gather*}
    x_C(t)=\displaystyle\sum_{n=-\infty}^{+\infty}x(nT)\delta(t-nT)=\sum_{n=-\infty}^{+\infty}\cos(\pi n)\delta\left(t-\frac{n}{2f_0}\right)
\end{gather*}
Ma utilizzando la frequenza di Nyquist come la frequenza di campionamento gli impulsi coincidono per $f=2f_0$, per cui questa sequenza non individua univocamente il segnale 
coseno, per cui è necessaria una frequenza di campionamento maggiore. Si considera allora:
\begin{gather*}
    f_C=4f_0
\end{gather*}
Si suppone sia presente aliasing nel segnale campionato in frequenza. Per cui si ha una frequenza $f_C<2f_0$, e sono presenti degli impulsi in più nell'intervallo $[-f_0,f_0]$, 
corrispondenti alle repliche antecedenti e posteriori alla replica centrata in $f=0$. Lo spettro si sovrappone, quindi ricomponendo il segnale non si ricostruice il 
segnale coseno originiario. 
Il segnale ricostruito si ottiene antitrasformando il segnale ottenuto in frequenza:
\begin{gather*}
    Y(f)=X_C(f)H(f)=\displaystyle\frac{1}{2}\left(f-\frac{1}{T}-f_0\right)+\frac{1}{2}\left(f-\frac{1}{T}+f_0\right)\\
    y(t)=\cos\left[2\pi \left(f_0-\frac{1}{T}\right)t\right]
\end{gather*}





Dato il segnale campionato $x(t)$, determinare la frequenza di Nyquist: 
\begin{gather*}
    x(t)=1+\left[\displaystyle\frac{\sin(20\pi t)}{\pi t}\right]^2=1+400\,\mbox{sinc}^2\left(20\pi t\right)
\end{gather*}
Lo spettro del segnale corrisponde a:
\begin{gather*}
    X(f)=\displaystyle \delta(f)+20\,\mbox{tri}\left(\frac{f}{20}\right)
\end{gather*}
Per cui la frequenza minima di campiomaneto corrisponde a $f_C>40\,Hz$, mentre il tempo di campionamento massimo corrisponde a:
Non è presente aliasing quando la frequenza assume valori maggiori di:
\begin{gather*}
    f>\displaystyle\frac{2}{\frac{1}{20}}=40\,Hz
\end{gather*}
Il tempo di campionamento deve essere:
\begin{gather}
    T>\displaystyle\frac{2}{f_{max}}=\frac{2}{40\,Hz}=25\,ms
\end{gather}
Si suppone si stia campionando con un passo di campionamento doppio al valore massio $\overline{T}=2T=50\,ms$. Lo spettro del segnale campionato corrisponde a:
\begin{gather*}
    X_C(f)=\displaystyle \frac{1}{\overline{T}}\sum_{n=-\infty}^{+\infty}X\left(f-\frac{n}{\overline{T}}\right)=20\sum_{n=-\infty}^{+\infty}\delta(f-20n)+20\,\mbox{tri}\left(\frac{f-20n}{20}\right)
\end{gather*}
Se il tempo di campionamento fosse pari a $T=50\times10^{-3}s$, i triangoli in frequenza sarebbero sfasati tra di loro di esattamente metà della loro base, per cui la 
somma di tutti i triangoli assume valore costante in frequenza. 
Lo spettro del segnale campionato diventa una costante sommata ad un treno di delta:
\begin{gather}
    X_C(f)=400+20\displaystyle\sum_{n=-\infty}^{+\infty}\delta\left(f-\frac{n}{20}\right)
\end{gather}
Si effettua ora quest'analisi nel dominio del tempo:
\begin{gather*}
    x_C(t)=x(t)\pi(t)=x(t)\displaystyle\sum_{n=-\infty}^{+\infty}x(nT)\delta(t-nT)=\sum_{n=-\infty}^{+\infty}\left[1+400\,\mbox{sinc}^2\left(20t\right)\right]\delta(t-50\times10^{-3}n)\\
    x_C(t)=\displaystyle\sum_{n=-\infty}^{+\infty}\delta(t-50\times10^{-3}n)+400\sum_{n=-\infty}^{+\infty}\mbox{sinc}^2(20t)\delta(t-50\times10^{-3}n)\\
    x_C(t)=\displaystyle\sum_{n=-\infty}^{+\infty}\delta(t-50\times10^{-3}n)+400\delta(t)
\end{gather*}
La sua trasformata corrisponde a:
\begin{gather*}
    X_C(f)=\displaystyle\sum_{n=-\infty}^{+\infty}\delta\left(f-\frac{n}{50\times10^{-3}}\right)\frac{1}{50\times10^{-3}}+400
\end{gather*}


Dato il segnale $x$, si vuole calcolare il massimo intervallo di campionamento:
\begin{gather*}
    x(t)=\cos^2\left(\displaystyle\frac{2\pi t}{T_0}\right)=\frac{1}{2}+\frac{1}{2}\cos\left(\frac{4\pi t}{T_0}\right)
\end{gather*}

Il periodo del segnale corrisponde a $T=\displaystyle\frac{T_0}{2}$. La sua trasformata equivale a:
\begin{gather*}
    X(f)=\displaystyle\frac{1}{2}\delta(f)+\frac{1}{4}\delta\left(f+\frac{2}{T_0}\right)+\frac{1}{4}\delta\left(f-\frac{2}{T_0}\right)
\end{gather*}
Quindi la frequenza di Nyquist corrisponde a:
\begin{gather*}
    f<2\frac{2}{T_0}=\frac{4}{T_0}
\end{gather*}
Per cui il passo di campionamento del segnale è:
\begin{gather}
    T>\displaystyle\frac{1}{f_C}=\frac{T_0}{4}
\end{gather}

Si considera un passo di campionamento minore rispetto al valore necessario per non avere aliasing: 
\begin{gather*}
    T=\frac{4}{9}T_0
\end{gather*}
Si vuole calcolare lo spettro del segnale considerando questo tempo di campionamento:
\begin{gather*}
    X_C(f)=\displaystyle\frac{9}{4T_0}\sum_{n=-\infty}^{+\infty}X\left(f-\frac{9n}{4T_0}\right)\\
    X_C(f)=\sum_{n=-\infty}^{+\infty}\left[\frac{9}{8T_0}\delta\left(f-\frac{9n}{4T_0}\right)+\frac{9}{16T_0}\delta\left(f+\frac{2}{T_0}-\frac{9n}{4T_0}\right)+\frac{9}{16T_0}\delta\left(f-\frac{2}{T_0}-\frac{9n}{4T_0}\right)\right]
\end{gather*}
\begin{gather}
    X_C(f)=\displaystyle\sum_{n=-\infty}^{+\infty}\left[\frac{9}{8T_0}\delta\left(f-\frac{9n}{4T_0}\right)+\frac{9}{16T_0}\delta\left(f+\frac{8-9n}{4T_0}\right)+\frac{9}{16T_0}\delta\left(f-\frac{8+9n}{4T_0}\right)\right]
\end{gather}
%% grafico 

Dato il segnale $x$ calcolare il periodo di campionamento:
\begin{gather*}
    x(t)=2f_0\mbox{sinc}(2f_0t)\cos(6\pi f_0t)\\
    X(f)=\displaystyle2\,\mbox{tri}\left(\frac{f}{f_0}\right)*\left[\frac{1}{2}\delta\left(f-3f_0\right)+\frac{1}{2}\delta(f+3f_0)\right]\\
    X(f)=\displaystyle\mbox{tri}\left(\frac{f-3f_0}{f_0}\right)+\mbox{tri}\left(\frac{f+3f_0}{f_0}\right)
\end{gather*}
La frequenza di Nyquist corrisponde a $6f_0$, per cui il tempo di campionamento corrisponde a $T=\displaystyle\frac{1}{6f_0}$. Si considera il segnale campionato con un 
passo di campionamento:
\begin{gather*}
    T=\displaystyle\frac{1}{4f_0}
\end{gather*}

Per cui la trasformata del segnale campionato corrisponde a:
\begin{gather*}
    X_C(f)=\displaystyle\sum_{n=-\infty}^{+\infty}\mbox{tri}\left(\frac{f-(4n+3)f_0}{f_0}\right)+\mbox{tri}\left(\frac{f-(4n-3)f_0}{f_0}\right)
\end{gather*}
%% grafico 

Si ricostruisce il segnale inserendolo in un filtro ricostruttore DAC di funzione di trasferimento $H(f)$, filtro passa basso ideale, che si estende tra le 
frequenze $[-2f_0,+2f_0]$, si considera un tempo di campionamento pari ad $1$:
\begin{gather*}
    H(f)=\displaystyle\mbox{rect}\left(\frac{f}{2f_0}\right)\\
    X(f)=X_C(f)H(f)=\displaystyle\mbox{tri}\left(\frac{f-f_0}{f_0}\right)+\mbox{tri}\left(\frac{f+f_0}{f_0}\right)
\end{gather*}
\begin{gather}
    x(t)=2f_0\mbox{sinc}^2(f_0t)\cos(2\pi f_0t)
\end{gather} 



Dato un segnale che occupa una banda $[-6\,kHz, 6\,kHz]$. Il segnale viene campionato con passo $T$. Il segnale viene filtrato per essere ricostruito da un filtro 
la cui funzione di trasferimento assume una forma trapezoidale in frequenza:
\begin{gather*}
    H(f)=\begin{cases}
        0 &f<-2K_f\land f\geq 2K_f\\
        A\left(2-\left|\displaystyle\frac{f}{K_f}\right|\right) &-2K_f\leq f<-K_f\land K_f\leq f<2K_f\\
        A & -K_f\leq f<K_f
    \end{cases}
\end{gather*}
Calcolare il valore massimo di $T$, $K_f$ e $A$ affinché il segnale ricostruito coincida con il segnale di partenza. 
Il parametro $A$ affinché si ritorni al segnale orginiario deve essere necessariamente: 
\begin{gather}
    A=\displaystyle\frac{1}{T}
\end{gather}
Il massimo passo di campionamento se fosse presente un filtro ideale sarebbe $T=1/12\times10^{-3}\,s$, ma per ricostruire il segnale originale con un filtro trapezoidale, bisognerebbe 
assegnare al valore $K_f=6\,kHz$. Ma in questo modo il filtro del DAC considera altre repliche, poiché la sua banda arriva fino ad un valore di $2K_f=12\,kHz$. La frequenza di 
campionameto deve essere quindi:
\begin{gather*}
    f\geq\displaystyle\frac{1}{T}+K_f=12+6=18\,kHz
\end{gather*}
Il passo di campionamento è quindi:
\begin{gather}
    T=\displaystyle\frac{1}{f}=55\,\mu Hz
\end{gather}



Un segnale viene campionato con frequenza di campionamento (di Nyquist) $f_C=100\,Hz$, dando origine ai seguenti campioni:
\begin{gather*}
    T_C=\displaystyle\frac{1}{100}\,s=10\,ms\\
    f_{max}=\frac{f_C}{2}=50\,Hz\\
    x(nT)=\begin{cases}
        -1&n=-2,-1\\
        1&n=1,2\\
        0&\forall n\neq\pm1,\pm2
    \end{cases}
\end{gather*}
Il segnale viene ricostruito con un filtro ideale di funzione di trasferimento $H$:
\begin{gather*}
    H(f)=T\,\mbox{rect}(Tf)\to  h(t)=\mbox{sinc}\left(\frac{t}{T}\right)
\end{gather*} 
Calcolare il valore del segnale ricostruito al tempo $t=5\,ms$. Il segnale campionato si esprime come:
\begin{gather*}
    x_C(t)=\displaystyle\sum_{n=-\infty}^{+\infty}x(nT)\delta(t-nT)\\
    x(2T)\delta(t-2T)+x(T)\delta(t-T)+x(-T)\delta(t+T)+x(-2T)\delta(t+2T)\\
    x_C(t)=\delta(t-2T)+\delta(t-T)-\delta(t+T)-\delta(t+2T)
\end{gather*}
Si calcola l'uscita del filtro DAC:
\begin{gather*}
    y(t)=x_C(t)*h(t)=\left[\delta(t-2T)+\delta(t-T)-\delta(t+T)-\delta(t+2T)\right]*\mbox{sinc}\left(\displaystyle\frac{t}{T}\right)\\
    \displaystyle\mbox{sinc}\left(\frac{t-2T}{T}\right)+\mbox{sinc}\left(\frac{t-T}{T}\right)-\mbox{sinc}\left(\frac{t+T}{T}\right)-\mbox{sinc}\left(\frac{t+2T}{T}\right)\\
    y(5\,ms)=\displaystyle\mbox{sinc}\left(\frac{5-20}{10}\right)+\mbox{sinc}\left(\frac{5-10}{10}\right)-\mbox{sinc}\left(\frac{5+10}{10}\right)-\mbox{sinc}\left(\frac{5+20}{10}\right)\\
    \displaystyle\mbox{sinc}\left(\frac{15}{10}\right)+\mbox{sinc}\left(\frac{5}{10}\right)-\mbox{sinc}\left(\frac{15}{10}\right)-\mbox{sinc}\left(\frac{25}{10}\right)\\
    \displaystyle\mbox{sinc}\left(\frac{1}{2}\right)-\mbox{sinc}\left(\frac{5}{2}\right)=\frac{\sin\left(\frac{\pi}{2}\right)}{\frac{\pi}{2}}-\frac{\sin\left(\frac{5\pi}{2}\right)}{\frac{5\pi}{2}}=\frac{2}{\pi}-\frac{2}{5\pi}
\end{gather*}
\begin{gather}
    y(5\,ms)=\displaystyle\frac{8}{5\pi}
\end{gather}


Dato un segnale $x$ di banda $[-\omega,\omega]$, campionato con un passo $T_C=\displaystyle\frac{1}{2\omega}$. Si considerano tre filtri, di risposta impulsiva:
\begin{gather*}
    h_1(t)=\mbox{sinc}\left(\displaystyle\frac{t}{T}\right)\to H_1(f)=T\mbox{sinc}(Tf)\\
    h_2(t)=\mbox{tri}\left(\displaystyle\frac{t}{T}\right)\\
    h_3(t)=\mbox{tri}\left(\displaystyle\frac{t}{2T}\right)
\end{gather*}
Il segnale campionato si ottiene come:
\begin{gather*}
    x_C(t)=\displaystyle\sum_{n=-\infty}^{+\infty}x(nT)\delta(t-nT)\\
    y_1(t)=x_C(t)*h_1(t)=\left[\displaystyle\sum_{n=-\infty}^{+\infty}x(nT)\delta(t-nT)\right]*\mbox{sinc}\left(\frac{t}{T}\right)
\end{gather*}
Questo caso corrisponde al teorema del campionamento, poiché il passo di campionamento corrisponde esattamente al reciproco del doppio della frequenza massima del segnale:
\begin{gather}
    y_1(t)=\displaystyle\sum_{n=-\infty}^{+\infty}x(nT)\mbox{sinc}\left(\frac{t-nT}{T}\right)=x(t)
\end{gather}
Per cui il segnale ricostruito corrisponde esattamente al segnale orginiario. 

Considerando il secondo filtro:
\begin{gather}
    y_2(t)=x_C(t)*h_2(t)=\displaystyle\sum_{n=-\infty}^{+\infty}x(nT)\mbox{tri}\left(\frac{t-nT}{T}\right)
\end{gather}
Questo filtro ricostruisce il segnale, ma presenta degli errori alle alte frequenze. 

Considerando il terzo filtro:
\begin{gather}
    y_3(t)=x_C(t)*h_3(t)=\displaystyle\sum_{n=-\infty}^{+\infty}x(nT)\mbox{tri}\left(\frac{t-nT}{2T}\right)
\end{gather}
Questo filtro non ricostruisce a pieno il segnale. 


Dato il segnale $x$:
\begin{gather*}
    x(t)=\mbox{tri}\left(\displaystyle\frac{t}{T}\right)-\mbox{tri}\left(\frac{t+T}{T}\right)\\
    x(t)=\begin{cases}
        \displaystyle\frac{t}{T} -2T\leq t<-T\\
        1+\displaystyle\frac{3t}{T} -T\leq t<0\\
        1-\displaystyle\frac{t}{T} 0\leq t<2T
    \end{cases}
\end{gather*}
Viene campionato con intervallo di campionamento $T_C=T$. Il segnale campionato viene ricostruito con un filtro passa basso ideale di ampiezza $1$ di banda $[-1/2T_C,1/2T_C]$. 
Calcolare il segnale ricostruito nel tempo continuo $y(t)$ e la sua trasformata $Y(f)$. 

Sicuramente lo spettro del segnale originario non è limitato in banda, per cui la sua ricostruzione non potrà coincidere con il segnale originale. Il segnale è invece limitato 
nel tempo, per cui il suo campionamento produce una sequenza limitata, ovvero non nulla solo per un numero limitato di campioni:
\begin{gather*}
    x_C(t)=\displaystyle\sum_{n=-\infty}^{+\infty}x(nT)\delta(t-nT)=x(-T)\delta(t+T)+x(0)\delta(t)=\delta(t)-\delta(t+T)
\end{gather*}
Il segnale ricostruito è quindi:
\begin{gather}
    y(t)=x_C(t)*h(t)=\left[\delta(t)-\delta(t+T)\right]*\frac{1}{T}\mbox{sinc}\left(\frac{t}{T}\right)=\frac{1}{T}\mbox{sinc}\left(\frac{t}{T}\right)-\frac{1}{T}\mbox{sinc}\left(\frac{t+T}{T}\right)\\
    Y(f)=\mbox{rect}(Tf)-\mbox{rect}(Tf)e^{2i\pi fT}
\end{gather}


Dato il segnale $x$:
\begin{gather*}
    x(t)=\mbox{sinc}\left(\displaystyle\frac{t}{2T}\right)+\cos\left(\frac{2\pi t}{T}\right)
\end{gather*}
Il segnale viene campionato con un passo $\overline{T}$ pari a quattro volte il periodo di Nyquist $T_C$:
\begin{gather*}
    \overline{T}=4T_C
\end{gather*}
Si considera la trasformata di Fourier del segnale $x$ per determinare la sua frequenza massima:
\begin{gather*}
    X(f)=2T\,\mbox{rect}(2Tf)+\displaystyle\frac{1}{2}\delta\left(f-\frac{1}{T}\right)+\frac{1}{2}\delta\left(f+\frac{1}{T}\right)
\end{gather*}
La banda del segnale è limitata nell'intervallo $[-1/T, 1/T]$, per cui la frequenza di Nyquist corrisponde a $2/T$:
\begin{gather*}
    T_C=\displaystyle\frac{T}{2}\\
    \overline{T}=2T
\end{gather*}

Determinare il segnale campionato $x_C$ e la sua trasformata di Fourier $X_C$:
\begin{gather*}
    x_C(t)=\displaystyle\sum_{n=-\infty}^{+\infty}x(n\overline{T})\delta(t-n\overline{T})=\sum_{n=-\infty}^{+\infty}\left[\mbox{sinc}\left(\frac{2nT}{2T}\right)+\cos\left(\frac{4\pi nT}{T}\right)\right]\delta(t-2nT)\\
    x_C(t)=\displaystyle\sum_{n=-\infty}^{+\infty}\left[\mbox{sinc}(n)+\cancelto{1}{\cos(4\pi n)}\right]\delta(t-2nT)
\end{gather*}
\begin{equation}
    x(t)=\displaystyle\delta(t)+\sum_{n=-\infty}^{+\infty}\delta(t-2nT)
\end{equation}
Si svolge ora la medesima analisi in frequenza:
\begin{gather*}
    X_C(f)=\displaystyle\frac{1}{2T}\sum_{n=-\infty}^{+\infty}X\left(f-\frac{n}{2T}\right)\\
    \displaystyle\frac{1}{2T}\sum_{n=-\infty}^{+\infty}2T\,\mbox{rect}\left[2T\left(f-\frac{n}{2T}\right)\right]+\frac{1}{4T}\sum_{n=-\infty}^{+\infty}\delta\left(f-\frac{1}{T}-\frac{n}{T}\right)+\frac{1}{4T}\sum_{n=-\infty}^{+\infty}\delta\left(f+\frac{1}{T}-\frac{n}{T}\right)
\end{gather*}
\begin{gather}
    X_C(t)1+\displaystyle\frac{1}{2T}\sum_{n=-\infty}^{+\infty}\delta\left(f-\frac{n}{2T}\right)
\end{gather}
Si ottiene lo stesso risultato della trasformata del segnale campionato nel tempo. 

\clearpage

\section{Fenomeni Aleatori}

Si consideri una sorgente $S$ ed un destinatario $D$ che possono trasmettere e ricevere solamente $1$ o $0$. La probabilità che si ricevi uno $0$, 
quando viene trasmesso uno $0$ è dell'$80\%$, mentre la probabilità che si ricevi un $1$ quando viene trasmesso uno $0$ è del $20\%$:
\begin{gather*}
    P(D=0|S=0)=0.8\\
    P(D=1|S=1)=0.2
\end{gather*}
Mentre la probabilità di errore nella trasmissione di un $1$ è:
\begin{gather*}
    P(D=1|S=1)=0.7\\
    P(D=0|S=1)=0.3
\end{gather*}
Per cui si conosco le probabilità di errore del canale di trasmissione.
I simboli $0$ e $1$ non sono equiprobabili alla sorgente:
\begin{gather*}
    P(S=0)=0.6\\
    P(S=1)=0.4
\end{gather*}

Si vuole calcolare la probabilità sia stato trasmesso un $1$ dalla sorgente quando viene misurato un $1$:
\begin{gather*}
    P(S=1|D=1)
\end{gather*}
Si esprime tramite il teorema di Bayes:
\begin{gather*}
    P(S=1|D=1)=\displaystyle\frac{P(D=1|S=1)P(S=1)}{P(D=1)}=\frac{0.7\cdot0.4}{P(D=1)}
\end{gather*}

Per determinare la probabilità che il ricevitore misuri $1$ si determina tramite il teorema delle proprietà totali:
\begin{gather*}
    P(D=1)=P(D=1|S=1)P(S=1)+P(D=1|S=0)P(S=0)=0.7\cdot0.4+0.3\cdot0.6
\end{gather*}
La probabilità che si misuri un $1$ quando viene effettivamente trasmesso un $1$ dalla sorgente risulta essere:
\begin{gather}
    P(S=1|D=1)=\displaystyle\frac{0.7\cdot0.4}{0.7\cdot0.4+0.2\cdot0.6}=0.7
\end{gather}

\end{document}