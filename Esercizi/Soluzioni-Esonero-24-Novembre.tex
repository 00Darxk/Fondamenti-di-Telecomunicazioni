\documentclass{article}

\usepackage{cancel}
\usepackage{tikz}
\usepackage{amsmath}
\usepackage{geometry}
\usepackage{graphicx}
\usepackage{amsfonts} 
\usepackage{verbatim}
\usepackage{mathrsfs}  
\usepackage{lmodern}
\usepackage{braket}
\usepackage{steinmetz}
\usepackage{bookmark}
\usepackage{pgfplots}
\usepackage[italian]{babel}
\pgfplotsset{compat=1.16}


%% inserire grafici

\hypersetup{
    colorlinks=true,
    linkcolor=black,
}
\tikzset{block/.style = {draw, fill=white, very thick, rectangle, minimum height=1cm, minimum width=2cm},  
         square/.style = {draw, fill=white, very thick, rectangle, minimum height=1cm, minimum width=1cm},
         sum/.style= {draw, fill=white, very thick, circle, node distance=0.5cm},  
         cross/.pic = {  \draw[rotate = 45] (-#1,0) -- (#1,0);
                         \draw[rotate = 45] (0,-#1) -- (0, #1);}}


\renewcommand{\contentsname}{Indice}

\title{Soluzioni Esonero del 24 Novembre}
\author{Giacomo Sturm}
\date{AA: 2023/2024 - Ing. Informatica}

\begin{document}

\maketitle

\vspace{10mm}

\begin{center}
    Sorgente del file LaTeX disponibile su \url{https://github.com/00Darxk/Fondamenti-di-Telecomunicazioni}
\end{center}

\clearpage

\tableofcontents

\clearpage

\section{Soluzioni Svolte}

\subsection{Serie di Fourier}
Calcolare i coefficienti di Fourier del segnale $x$, la sua potenza e la sua trasformata:
\begin{gather*}
    x(t)=\displaystyle\frac{1}{2}-\cos^2\left[\frac{\pi(t-2T_0)}{4T_0}\right]
\end{gather*}
Per la proprietà di bisezione del coseno e per il coseno della differenza si ottiene la seguente espresisone equivalente:
\begin{gather*}
    x(t)=\displaystyle\frac{1}{2}-\frac{1}{2}-\cos\left[\frac{\pi(t-2T_0)}{2T_0}\right]=-\frac{1}{2}\cos\left(\frac{\pi t}{2T_0}\right)\cos\left(\frac{-2T_0\pi}{2T_0}\right)=\frac{1}{2}\cos\left(\frac{\pi t}{2T_0}\right)\\
    T=4T_0
\end{gather*}
Si esprime il coseno in forma esponenziale per determinare i coefficienti per confronto diretto:
\begin{gather*}
    x(t)=\displaystyle\frac{1}{4}\left(e^{\frac{i\pi t}{2T_0}}-e^{-\frac{i\pi t}{2T_0}}\right)
\end{gather*}
\begin{gather}
    c_1=c_{-1}=\displaystyle\frac{1}{4}\\
    c_n=0\,\forall n\neq\pm1
\end{gather}
Si calcola la potenza tramite Parceval:
\begin{gather}
    P_x=\displaystyle\sum_{n=-\infty}^{+\infty}|c_k|^2=\frac{1}{8}
\end{gather}
Dato un segnale periodico la sua trasformata si può esprimere come:
\begin{gather}
    X(f)=\displaystyle\sum_{n=-\infty}^{+\infty}c_n\delta\left(f-\frac{n}{T}\right)=\frac{1}{4}\left(f-\frac{1}{4T_0}\right)+\frac{1}{4}\delta\left(f+\frac{1}{4T_0}\right)
\end{gather}


Calcolare la trasformata del segnale $x$:
\begin{gather*}
    x(t)=\begin{cases}
        \cos\displaystyle\left(\frac{\pi t}{T}\right) &-\displaystyle\frac{T}{4}\leq t<\frac{T}{4}\\
        0&n\notin\left[\right.\displaystyle-\frac{T}{4},\frac{T}{4}\left.\right)
    \end{cases}
\end{gather*}
Si può esprimere il segnale replicato come un coseno moltiplicato per una finestra:
\begin{gather*}
    \dot x(t)=\cos\displaystyle\left(\frac{\pi t}{T}\right)\mbox{rect}\left(\frac{2t}{T}\right)
\end{gather*}
Si individua il periodo $\overline{T}$, e si determinano i coefficienti tramite la definizione:
\begin{gather*}
    \overline{T}=2T\\
    c_n=\displaystyle\frac{1}{T}\int_{-\frac{\overline{T}}{2}}^{\frac{\overline{T}}{2}}x(t)e^{-2i\pi nt/\overline{T}}dt
\end{gather*}
Altrimenti si determinano rispetto alla trasformata di Fourier del segnale replicato $\dot x(t)$:
\begin{gather*}
    c_n=\displaystyle\frac{1}{\overline{T}}\dot X\left(\frac{n}{\overline{T}}\right)
\end{gather*}

Si considera la trasformata di $\dot x$:
\begin{gather*}
    \dot X(f)=\displaystyle\frac{T}{2}\mbox{sinc}\left(\frac{T}{2}f\right)*\left[\frac{1}{2}\left(f-\frac{1}{2T}\right)+\frac{1}{2}\left(f+\frac{1}{2T}\right)\right]\\
    \dot X(f)=\displaystyle\frac{T}{4}\mbox{sinc}\left[\frac{T}{2}\left(f-\frac{1}{2T}\right)\right]+\frac{T}{4}\mbox{sinc}\left[\frac{T}{2}\left(f+\frac{1}{2T}\right)\right]
\end{gather*}
I coefficienti sono quindi:
\begin{gather*}
    c_n=\frac{1}{\overline{T}}\dot X\left(\frac{n}{\overline{T}}\right)\frac{T}{4}\mbox{sinc}\left[\frac{T}{2}\left(\frac{n}{\overline{T}}-\frac{1}{2T}\right)\right]+\frac{T}{4}\mbox{sinc}\left[\frac{T}{2}\left(\frac{n}{\overline{T}}+\frac{1}{2T}\right)\right]\\
    \overline{T}=2T
\end{gather*}
\begin{gather}
    c_n=\displaystyle\frac{1}{8}\mbox{sinc}\left(\frac{n-1}{4}\right)+\frac{1}{8}\mbox{sinc}\left(\frac{n+1}{4}\right)
\end{gather}
La trasformata del segnale completo $x$ si esprime come:
\begin{gather}
    X(f)=\displaystyle\sum_{n=-\infty}^{+\infty}\left[\frac{1}{8}\mbox{sinc}\left(\frac{n-1}{4}\right)+\frac{1}{8}\mbox{sinc}\left(\frac{n+1}{4}\right)\right]\delta\left(f-\frac{n}{\overline{T}}\right)
\end{gather}


Calcolare i coefficienti del segnale $x$, la trasofrmata e la potenza:
\begin{gather*}
    x(t)=\displaystyle\frac{2}{3}-i\cos\left(\frac{\pi t}{3T_0}\right)\\
    T=6T_0
\end{gather*}
Per confronto diretto si ottengono i coefficienti:
\begin{gather*}
    x(t)=\displaystyle\frac{2}{3}-\frac{i}{2}e^{i\frac{\pi t}{3T_0}}-\frac{i}{2}e^{-i\frac{\pi t}{3T_0}}
\end{gather*}
\begin{gather}
    c_0=\displaystyle\frac{2}{3}\\
    c_{1}=c_{-1}=\displaystyle-\frac{i}{2}
\end{gather}
La trasformata di ottiene come:
\begin{gather}
    X(f)=\displaystyle\sum_{n=-\infty}^{+\infty}c_n\delta\left(f-\frac{n}{T}\right)=\frac{2}{3}\delta(f)-\frac{i}{2}\delta\left(f-\frac{1}{6T_0}\right)-\frac{i}{2}\delta\left(f+\frac{1}{6T_0}\right)
\end{gather}
La potenza si calcola con il teorema di Parceval:
\begin{gather}
    P_x=\displaystyle\sum_{n=-\infty}^{+\infty}|c_n|^2=\frac{4}{9}+\frac{1}{4}+\frac{1}{4}=\frac{17}{18}
\end{gather}

Calcolare i coefficienti e la trasformata di Fourier del segnale periodico $x$, di periodo $T$:
\begin{gather*}
    x(t)=\begin{cases}
        t&\displaystyle-\frac{T}{4}\leq t<\frac{T}{4}\\
        0&t\notin\left[\displaystyle-\frac{T}{4},\frac{T}{4}\right]
    \end{cases}
\end{gather*}
I coefficienti si possono calcolare con la definizione, altrimenti si calcola rispetto al segnale replicato $\dot x$:
\begin{gather*}
    c_n=\displaystyle\frac{1}{\overline{T}}\dot X\left(\frac{n}{\overline{T}}\right)\\
    \dot x(t)=t\,\mbox{rect}\left(\displaystyle\frac{2t}{T}\right)
\end{gather*}
Questa trasformata si può calcolare mediante la proprietà della derivazione della trasformata:
\begin{gather*}
    \dot x(t)=t\,\mbox{rect}\displaystyle\left(\frac{2t}{T}\right)\\
    \dot X(f)=\displaystyle-\frac{1}{2\pi i}X_1(f)\\
    x_1(t)=\mbox{rect}\displaystyle\left(\frac{2t}{T}\right)
    X_1(f)=\displaystyle\frac{T}{2}\mbox{sinc}\left(\frac{T}{2}f\right)\\
    \dot X(f)=-\displaystyle\frac{1}{2i\pi} \frac{d}{df}\left[\frac{T}{2}\mbox{sinc}\left(\frac{T}{2}f\right)\right]=-\frac{1}{2i\pi}\frac{d}{df}\left[\frac{T}{2}\frac{\sin\left(\frac{T}{2}\pi f\right)}{\frac{T}{2}\pi f}\right]\\
    \dot X(f)=\displaystyle\frac{i}{2\pi^2}\frac{\cos\left(\frac{T}{2}\pi f\right)\frac{T}{2}\pi f-\sin\left(\frac{T}{2}\pi f\right)}{f^2}=\frac{iT}{4\pi f}\cos\left(\frac{\pi fT}{2}\right)-\frac{i}{2\pi^2f^2}\sin\left(\frac{\pi fT}{2}\right)
\end{gather*}
I coefficienti di Fourier sono allora:
\begin{gather*}
    c_n=\displaystyle\frac{1}{T}\frac{iT}{4\pi \frac{n}{T}}\cos\left(\frac{\pi \frac{n}{T}T}{2}\right)-\frac{i}{2\pi^2\left(\frac{n}{T}\right)^2}\sin\left(\frac{\pi \frac{n}{T}T}{2}\right)\
\end{gather*}
\begin{gather}
    c_n=\displaystyle\frac{iT}{4\pi n}\cos\left(\frac{\pi n}{2}\right)-\frac{iT^2}{2\pi^2n^2}\sin\left(\frac{\pi n}{2}\right)
\end{gather}
Per cui per $n=2k$ si annulla il coseno, mentre per $=2k-1$ si annulla il seno. Mentre la trasformata del segnale risulta essere:
\begin{gather}
    X(f)=\displaystyle\sum_{n=-\infty}^{+\infty}\left[\frac{iT}{4\pi n}\cos\left(\frac{\pi n}{2}\right)-\frac{iT^2}{2\pi^2n^2}\sin\left(\frac{\pi n}{2}\right)\right]\delta\left(f-\frac{n}{T}\right)
\end{gather}

\subsection{Convoluzioni}

Calcolare l'autoconvoluzione del segnale $x$:
\begin{gather*}
    x[n]=\delta[n]-\delta[n-2]+\delta[n+2]\\
    x[n]*x[n]=(\delta[n]-\delta[n-2]+\delta[n+2])*(\delta[n]-\delta[n-2]+\delta[n+2])\\
    \delta[n]-\delta[n-2]+\delta[n+2]-\delta[n-2]+\delta[n-4]-\delta[n]+\delta[n+2]-\delta[n]+\delta[n+4]
\end{gather*}
\begin{equation}
    x[n]*x[n]=-\delta[n]-2\delta[n-2]+2\delta[n+2]+\delta[n-4]+\delta[n+4]
\end{equation}




Calcolare l'autoconvoluzione di $x$:
\begin{gather*}
    x[n]=2\delta[n]-\delta[n-1]-\delta[n-1]\\
    x[n]*x[n]=(2\delta[n]-\delta[n-1]-\delta[n+1])*(2\delta[n]-\delta[n-1]-\delta[n+1])\\
    4\delta[n]-2\delta[n-1]-2\delta[n+1]-2\delta[n-1]+\delta[n-2]+\delta[n]-2\delta[n+1]+\delta[n]+\delta[n+2]
\end{gather*}
\begin{equation}
    x[n]*x[n]=6\delta[n]-4\delta[n-1]-\delta[n+1]+\delta[n-2]+\delta[n-2]
\end{equation}



Calcolare la convoluzione tra i segnali $x$ e $y$:
\begin{gather*}
    x(t)=u(t)\\
    y(t)=t\,\mbox{rect}\displaystyle\left(\frac{t}{2T}\right)\\
    x(t)*y(t)=\displaystyle\int_{-\infty}^{+\infty}u(\tau)y(t-\tau)d\tau=\begin{cases}
        0&t<-T\\
        \displaystyle\int_{-T}^{t}\tau d\tau=\frac{t^2-T^2}{2}&-T\leq t<T\\
        0& t\geq T
    \end{cases}
\end{gather*}
\begin{equation}
    x(t)*y(t)=\begin{cases}
        0& t<-T\land t\geq T\\
        \displaystyle\frac{t^2-T^2}{2}&-T\leq t<T
    \end{cases}
\end{equation}



Calcolare la convoluzione tra i segnali $x$ e $y$:
\begin{gather*}
    x(t)=u(t)-u(-t)\\
    y(t)=\mbox{rect}\displaystyle\left(\frac{t}{2T}\right)\\
    x(t)*y(t)=\begin{cases}
        -2T&t<-T\\
        \displaystyle-\int_{t-T}^{0}d\tau+\int_{0}^{t+T}d\tau &-T\leq t<T\\
        2T &t\geq T
    \end{cases}
\end{gather*}
\begin{equation}
    x(t)*y(t)=\begin{cases}
        -2T&t<-T\\
        2t&-T\leq t<T\\
        2T &t\geq T
    \end{cases}
\end{equation}

\subsection{Proprietà Sistemi}

Dati i sistemi seguenti, definite le loro proprietà: 
\begin{gather*}
    y(t)=x^2(t)-x(t+1)
\end{gather*}
Non lineare, invariante nel tempo, non causale

\begin{gather*}
    y(t)=x(t-1)u(t-1)
\end{gather*}
Lineare, non invariante nel tempo e causale. 

\begin{gather*}
    y(t)=x(t+1)+u(t-1)
\end{gather*}
Non lineare, non invariante nel tempo non è causale

\begin{gather*}
    y(t)=x(t)x(t-1)
\end{gather*}
Non lineare, invariante nel tempo, causale

\subsection{Sistemi Ingresso-Uscita}

Data la funzione di trasferimento $H$ calcolare la risposta impulsiva del sistema:
\begin{gather*}
    H(f)=\left[\mbox{sinc}\displaystyle(2Tf)-2\mbox{sinc}(4Tf)\right]e^{-2i\pi ft}
\end{gather*}
\begin{equation}
    h(t)=\displaystyle\frac{1}{2T}\mbox{rect}\left(\frac{t-T}{2T}\right)-\frac{1}{2T}\mbox{rect}\left(\frac{t-T}{4T}\right)
\end{equation}

%% grafico

L'uscita del sistema con un gradino in entrata risulta:
\begin{gather*}
    y(t)=h(t)*u(t)=\left[\displaystyle\frac{1}{2T}\mbox{rect}\left(\frac{t-T}{2T}\right)-\frac{1}{2T}\mbox{rect}\left(\frac{t-T}{4T}\right)\right]*u(t)\\
    y(t)=\begin{cases}
        0&t<-T\\
        \displaystyle\int_{-T}^t-\frac{1}{2T}d\tau&-T\leq t<0\\
        -\displaystyle\frac{1}{2}&0\leq t<2T\\
        \displaystyle-\frac{1}{2}-\int_{2T}^t\frac{1}{2T}d\tau =\displaystyle-\frac{1}{2}-\frac{t-2T}{2T}&2T\leq t<3T\\
        -1&t\geq 3T
    \end{cases}
\end{gather*}
\begin{equation}
    y(t)=\begin{cases}
        0&t<-T\\
        \displaystyle-\frac{t+T}{2T}&-T\leq t<0\\
        \displaystyle-\frac{1}{2}&0\leq t<2T\\
        1-\displaystyle\frac{t}{2T}&2T\leq t<3T\\
        -1 &t\geq 3T
    \end{cases}
\end{equation}




Data la risposta impulsiva $h$ calcolare la funzione di trasferimento:
\begin{gather*}
    h(t)=e^{-t}\cos(2\pi f_0t)u(t)\\
    H(f)=\displaystyle\frac{1}{1+2i\pi f}*\left[\frac{1}{2}\delta(f-f_0)+\frac{1}{2}\delta(f+f_0)\right]
\end{gather*}
\begin{equation}
    H(f)=\displaystyle\frac{1}{2+4i\pi (f-f_0)}+\frac{1}{2+4i\pi (f+f_0)}
\end{equation}

%% grafico

Calcolare l'uscita del sistema con una delta centrata in $1/f_0$ in entrata:
\begin{gather*}
    y(t)=h(t)*\delta\left(t-\displaystyle\frac{1}{f_0}\right)=e^{-t+\frac{1}{f_0}}\cos\left(2\pi f_0t+2\pi\right)u\left(t-\frac{1}{f_0}\right)
\end{gather*}
\begin{equation}
    y(t)=e^{-t+\frac{1}{f_0}}\cos\left(2\pi f_0t\right)u\left(t-\frac{1}{f_0}\right)
\end{equation}



Data la risposta impulsiva $h$, determinare la funzione di trasferimento del sistema:
\begin{gather*}
    h(t)=\displaystyle\frac{\sin^2(2\pi f_0t)}{\pi t}=\sin(2\pi f_0t)2T_0\mbox{sinc}(2f_0t)\\
    H(f)=\mbox{rect}\left(\displaystyle\frac{f}{2f_0}\right)*\left[\frac{1}{2i}\delta\left(f-f_0\right)-\frac{1}{2i}\delta(f+f_0)\right]
\end{gather*}
\begin{equation}
    H(f)=\displaystyle\frac{1}{2i}\mbox{rect}\left(\frac{f-f_0}{2f_0}\right)-\frac{1}{2i}\mbox{rect}\left(\frac{f+f_0}{2f_0}\right)
\end{equation}

%% grafico 

Calcolare l'uscita del sistema con una delta centrata in $1/f_0$ in entrata:
\begin{gather*}
    y(t)=h(t)*\delta\left(\displaystyle t-\frac{1}{f_0}\right)=\frac{\sin^2\left(2\pi f_0t-2\pi\right)}{\pi \left(t-\frac{1}{f_0}\right)}
\end{gather*}
\begin{equation}
    y(t)=\displaystyle\frac{\sin^2(2\pi f_0t)}{\pi\left(1-\frac{1}{f_0}\right)}
\end{equation}


Data la risposta impulsiva $h$, calcolare la funzione di trasferimento del sistema:
\begin{gather*}
    h(t)=\mbox{rect}\left(\displaystyle\frac{t-2T}{2T}\right)-\mbox{rect}\left(\frac{t+2T}{2T}\right)\\
    H(f)=2T\mbox{sinc}(2fT)e^{-4i\pi fT}-2T\mbox{sinc}(2fT)e^{4i\pi fT}
\end{gather*}
\begin{equation}
    H(f)=-4iT\mbox{sinc}(2fT)\sin(4\pi fT)
\end{equation}

%% grafico

Calcolare l'uscita del ssitema quando è presente un gradino in entrata:
\begin{gather*}
    y(t)=h(t)*u(t)=\left[\mbox{rect}\left(\displaystyle\frac{t-2T}{2T}\right)-\mbox{rect}\left(\frac{t+2T}{2T}\right)\right]*u(t)\\
    y(t)=\begin{cases}
        0 &t<-3T\\
        \displaystyle-\int_{-3T}^td\tau &-3T\leq t<-T\\
        -2T &-T\leq t<T\\
        -2T+\displaystyle\int_{T}^{t}d\tau &T\leq t<3T\\
        0& t\geq 3T
    \end{cases}
\end{gather*}
\begin{equation}
    y(t)=\begin{cases}
        0 &t<-3T\\
        -t-3T&-3T\leq t<-T\\
        -2T &-T\leq t<T\\
        t-3T &T\leq t<3T\\
        0& t\geq 3T
    \end{cases}
\end{equation}

\end{document}